


\iffalse
% !TeX spellcheck = en_US
%%%%%%%%%%%%%%%%%%%%%%%%%%%%%%%%%%%%%%%%%%%%%%%%%%%%%%%%%%%%%%%%%%%%%%%%
%
% This is the template file for the 6th International conference
% NONLINEAR ANALYSIS AND EXTREMAL PROBLEMS
% June 25-30, 2018
% Irkutsk, Russia
%
%%%%%%%%%%%%%%%%%%%%%%%%%%%%%%%%%%%%%%%%%%%%%%%%%%%%%%%%%%%%%%%%%%%%%%%%
% The preparation of the article is based on the standard llncs class
% (Lecture Notes in Computer Sciences), which is adjusted with style
% file of the conference.
%
% There are two ways of compilation of the file into PDF
% 1. Use pdfLaTeX (pdflatex), (LaTeX+DVIPS will not work);
% 2. Use LuaLaTeX (XeLaTeX will work too).
% When using LuaLaTeX You will need TTF or OTF CMU fonts
% (Computer Modern Unicode). The fonts are installed with 'cm-unicode' package in
% a distribution of LaTeX % (https://www.ctan.org/tex-archive/fonts/cm-unicode),
% either by downloading and installing these fonts system wide, the address of their page is
% http://canopus.iacp.dvo.ru/%7Epanov/cm-unicode/
% The second option won't work in XeLaTeX.
%
% For MiKTeX (LaTeX distribution for Windows),
%  1. Package 'cm-unicode' is installed manually with the MiKTeX administration Console.
%  2. For the compilation of this example, namely, the stub figure, one will also need to
% download package 'pgf' manually. This package uses in the popular
% package tikz.
%  3. Tests showed that the rest of the required packages MiKTeX loads automatically (if
%     it is allowed). The 'auto download' option is
%     configured in 'Settings' section in MiKTeX Console.
%
%
% The easiest way to compile an article is to use pdfLaTeX, but
% the final layout of the book will be compiled with LuaLaTeX,
% as a result will be of better quality thanks to the package 'microtype' and
% use vector OTF instead of standard raster fonts of pdfLaTeX.
%
% In the case of questions and problems with the article compilation,
% write letters to e-mail: eugeneai@irnok.net, Cherkashin Evgeny.
%
% New version of the correcting style file will be available at the website:
%     https://github.com/eugeneai/nla-style
%     file - nla.sty
%
% Further instructions are in the text body of the template. The template itself
% is an article example.
%
% The LaTeX2e format is used!

% 12 points font size is used.
\documentclass[12pt]{llncs}

% The correcting style file is added.
\usepackage{todonotes}

\usepackage{nla} % This package is needed for compiling
                 % this template, it should be removed
                 % from your article.

% Many popular packages (amsXXX, graphicx, etc.) are already imported in the style file.
% If there is a conflict with your packages, try disabling them and compile
% the text.
%
% It would be convenient in the layout of the proceedings if the file names
% of the figures of different authors do not clash.
% To minimize the clash, the drawings can be placed in a separate subfolder
% named after the author or the title of the paper.
%
% \graphicspath{{ivanov-petrov-pics/}} % specifies the folder with images in png, pdf formats.
% or
% \graphicspath{{great-problem-solving-paper-pics/}}.

\begin{document}

% Text should be formatted in accordance with the 'article' class, using extensions like
% AMS.
%
\fi

\title{A Numerical Scheme for Solving of a Bilinear Optimal Impulsive Control Problem with Intermediate State Constraints\thanks{This work is financially supported by the project of ISDCT SB RAS ``Theory and methods for studying  of evolutionary equations and control systems with applications,'' no.~121041300060-4.}}
% First author
\author{Olga Samsonyuk}
\institute{ISDCT SB RAS, Irkutsk, Russia\\
  \email{olga.samsonyuk@icc.ru}
}
% etc

\maketitle

\begin{abstract}

\keywords{impulsive control, trajectory of bounded variation, Lyapunov type function, optimality condition, numerical solution method}
\end{abstract}


This talk is devoted to a numerical scheme for solving of a bilinear  optimal impulsive control problem with intermediate phase constraints. This scheme is based on optimality conditions using compound monotone functions of the Lyapunov type \cite{samnla1,samnla2}.


\begin{thebibliography}{5}
\bibitem{samnla1} Samsonyuk O.N. Optimality conditions for optimal impulsive control
problems with multipoint state constraints. Journal of Global Optimization. 2020. Vol.
76, no. 3. Pp. 625--644. DOI: 10.1007/s10898-019-00868-w 

\bibitem{samnla2} Samsonyuk O.N., Sorokin S.P. Optimality Conditions for Impulsive
Processes with Intermediate State Constraints.  The 15th International Conference on
Stability and Oscillations of Nonlinear Control Systems (Pyatnitskiy's Conference)
(STAB), Moscow, Russia, 3--5  June 2020.  IEEE, 2020.  DOI: 
10.1109/STAB49150.2020.9140658
\end{thebibliography}
%\end{document}

%%% Local Variables:
%%% mode: latex
%%% TeX-master: t
%%% End:
