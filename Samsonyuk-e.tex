

\iffalse

% !TeX spellcheck = en_US
%%%%%%%%%%%%%%%%%%%%%%%%%%%%%%%%%%%%%%%%%%%%%%%%%%%%%%%%%%%%%%%%%%%%%%%%
%
% This is the template file for the 6th International conference
% NONLINEAR ANALYSIS AND EXTREMAL PROBLEMS
% June 25-30, 2018
% Irkutsk, Russia
%
%%%%%%%%%%%%%%%%%%%%%%%%%%%%%%%%%%%%%%%%%%%%%%%%%%%%%%%%%%%%%%%%%%%%%%%%
% The preparation of the article is based on the standard llncs class
% (Lecture Notes in Computer Sciences), which is adjusted with style
% file of the conference.
%
% There are two ways of compilation of the file into PDF
% 1. Use pdfLaTeX (pdflatex), (LaTeX+DVIPS will not work);
% 2. Use LuaLaTeX (XeLaTeX will work too).
% When using LuaLaTeX You will need TTF or OTF CMU fonts
% (Computer Modern Unicode). The fonts are installed with 'cm-unicode' package in
% a distribution of LaTeX % (https://www.ctan.org/tex-archive/fonts/cm-unicode),
% either by downloading and installing these fonts system wide, the address of their page is
% http://canopus.iacp.dvo.ru/%7Epanov/cm-unicode/
% The second option won't work in XeLaTeX.
%
% For MiKTeX (LaTeX distribution for Windows),
%  1. Package 'cm-unicode' is installed manually with the MiKTeX administration Console.
%  2. For the compilation of this example, namely, the stub figure, one will also need to
% download package 'pgf' manually. This package uses in the popular
% package tikz.
%  3. Tests showed that the rest of the required packages MiKTeX loads automatically (if
%     it is allowed). The 'auto download' option is
%     configured in 'Settings' section in MiKTeX Console.
%
%
% The easiest way to compile an article is to use pdfLaTeX, but
% the final layout of the book will be compiled with LuaLaTeX,
% as a result will be of better quality thanks to the package 'microtype' and
% use vector OTF instead of standard raster fonts of pdfLaTeX.
%
% In the case of questions and problems with the article compilation,
% write letters to e-mail: eugeneai@irnok.net, Cherkashin Evgeny.
%
% New version of the correcting style file will be available at the website:
%     https://github.com/eugeneai/nla-style
%     file - nla.sty
%
% Further instructions are in the text body of the template. The template itself
% is an article example.
%
% The LaTeX2e format is used!

% 12 points font size is used.
\documentclass[12pt]{llncs}

% The correcting style file is added.
\usepackage{todonotes}

\usepackage{nla} % This package is needed for compiling
                 % this template, it should be removed
                 % from your article.

% Many popular packages (amsXXX, graphicx, etc.) are already imported in the style file.
% If there is a conflict with your packages, try disabling them and compile
% the text.
%
% It would be convenient in the layout of the proceedings if the file names
% of the figures of different authors do not clash.
% To minimize the clash, the drawings can be placed in a separate subfolder
% named after the author or the title of the paper.
%
% \graphicspath{{ivanov-petrov-pics/}} % specifies the folder with images in png, pdf formats.
% or
% \graphicspath{{great-problem-solving-paper-pics/}}.

\begin{document}

 

% Text should be formatted in accordance with the 'article' class, using extensions like
% AMS.
%
\fi

\title{A scheme for numerical  solving of a bilinear optimal impulsive control problem with intermediate state constraints\thanks{This work is financially supported by the Ministry of Education and Science of the Russian Federation (state registration No.~121041300060-4.}}
\author{Olga Samsonyuk}
\institute{ISDCT SB RAS, Irkutsk, Russia\\
  \email{olga.samsonyuk@icc.ru}
}
% etc

\maketitle

\begin{abstract}
This note addresses a  scheme for numerical solving of a bilinear  optimal impulsive control problem with intermediate state constraints.  The scheme is based on optimality conditions using compound monotone functions of the Lyapunov type.

\keywords{impulsive control,  optimality conditions, numerical scheme}
\end{abstract}

Let us consider the optimal impulsive control problem  $(P)$: $\ J=l(q)\rightarrow \min,$
\begin{align}
& d{x}_1(t)=f_1(t)dt+\big(ax_2(t)+b\big)\mu_1(t),  
 \ \  d{x}_2(t)=f_2(t)dt+\big(cx_1(t)+d\big)\mu_2(t),  
 \label{nla3}\\ 
 &  x_1(0)=x_{10}, \  x_2(0)=x_{20}, \ \   q_j\in Q_j,\  j=0,\ldots,N   .\label{nla5}
\end{align}
Here, $T=[0,t_1]$ is a given time interval,  $\mu=(\mu_1, \mu_2)$ is a nonnegative bounded Borel measure on $T$,  $x_1(\cdot)$,  $x_2(\cdot)$,  $V(\cdot)$ are right continuous on $(0,t_1]$ and have bounded total variation on $T$,    $(\theta_0,\theta_1,\ldots,\theta_N)$ are given points such that $0=\theta_0<\theta_1<\ldots<\theta_N=t_1$, $q=\big(q_0,q_1,\ldots,q_N\big)$, where $ q_j=\big(x_1(\theta_j),x_2(\theta_j),V(\theta_j)\big)$, $j=0,\ldots,N$. The constraint sets $Q_j$, $j=0,\ldots,N$, are compact subsets from ${\mathbb R}^3$. The solution's concept to (\ref{nla3}) is given in \cite{samnla2}.

We propose a compound Lyapunov type function described by a continuous  piecewise smooth function $\varphi(t,x_1,x_2,V;t_0,y_{10},y_{20})$ such that $\varphi = \big\{ \varphi_i\big\}_{i=1,\ldots,7} ,$ where functions $\varphi_i$ are defined on   domains with disjoint interiors. The function $\varphi$ is such that: (i)    $\varphi_i$, $i=1,\ldots,7,$ are strongly monotone with respect to  (\ref{nla3})  on every interval $(\theta_{j-1},\theta_j]$, $j=\overline{1,N}$  ($t_0=\theta_{j-1}$); (ii)    $\varphi$ is weakly premonotone with respect to    (\ref{nla3}) on   $(\theta_{j-1},\theta_j]$, $j=\overline{1,N}$, and satisfies the condition
$
\big\{(x_1,x_2)\ |\  \varphi(t_0,x_1,x_2,0;t_0,y_{10},y_{20})\leq 0\big\}=\{y_{10},y_{20}\}.
$ 
The function $\varphi$ gives an approximation of the reachable set and allows us to use the optimality conditions with compound monotone functions of the Lyapunov type   from   \cite{samnla2,samnla1}. Based on these conditions, we propose a scheme for numerical solving of the problem $(P)$.

\begin{thebibliography}{5}
%\bibitem{MilRub2013nla3} Miller  B.M.,   Rubinovich E.Ya.  Discontinuous solutions in the optimal control problems and their representation by singular space-time transformations. Autom. Remote Control. 2013. Vol. 74. Pp. 1969--2006.
\bibitem{samnla2} Samsonyuk O.N., Sorokin S.P. Optimality conditions for impulsive
processes with intermediate state constraints.  The 15th International Conference on
Stability and Oscillations of Nonlinear Control Systems (Pyatnitskiy's Conference)
(STAB), Moscow, Russia, 3--5  June 2020.  IEEE, 2020.  %DOI: 10.1109/STAB49150.2020.9140658
\bibitem{samnla1} Samsonyuk O.N. Optimality conditions for optimal impulsive control
problems with multipoint state constraints. Journal of Global Optimization. 2020. Vol.
76, no. 3. Pp. 625--644. %DOI: 10.1007/s10898-019-00868-w 
\end{thebibliography}


%\end{document}

%%% Local Variables:
%%% mode: latex
%%% TeX-master: t
%%% End:
