\begin{englishtitle} % Настраивает LaTeX на использование английского языка
% Этот титульный лист верстается аналогично.
\title{Stationary Points of d.c. Lagrangians in Solving Inverse Problems of
the Control Theory}
% First author
\author{Nina Subbotina 
  \and
  Evgenii Krupennikov 
}
\institute{IMM UrB RAS, UrFU, Ekaterinburg, Russia\\
  \email{subb@uran.ru, krupennikov@imm.uran.ru}
}
% etc

\maketitle

\begin{abstract}
The talk is devoted to the problem of dynamic control reconstruction (the DCRP) for dynamical systems. In this problem, using known discrete inaccurate measurements of the observed trajectory of a dynamic system, it is necessary to approximate the observed trajectory and the control that generated it. A new approach to solving the DCRP, proposed by the authors of the report, is discussed. The proposed solution is based on finding stationary points of regularized integral residual functionals in auxiliary variational problems. The key feature of the approach is the use of the so-called d.c. Lagrangians in auxiliary problems. This approach provides a stable oscillatory character of the approximations, as well as the stability of the constructed structures with respect to measurement errors. The report discusses the properties of constructions obtained in auxiliary variational problems.

\keywords{dynamic reconstruction, d.c. functions, calculus of variations}
\end{abstract}
\end{englishtitle}

\iffalse

%%%%%%%%%%%%%%%%%%%%%%%%%%%%%%%%%%%%%%%%%%%%%%%%%%%%%%%%%%%%%%%%%%%%%%%%
%
%  This is the template file for the 6th International conference
%  NONLINEAR ANALYSIS AND EXTREMAL PROBLEMS
%  June 25-30, 2018
%  Irkutsk, Russia
%
%%%%%%%%%%%%%%%%%%%%%%%%%%%%%%%%%%%%%%%%%%%%%%%%%%%%%%%%%%%%%%%%%%%%%%%%

%  Верстка статьи осуществляется на основе стандартного класса llncs
%  (Lecture Notes in Computer Sciences), который корректируется стилевым
%  файлом конференции.
%
%  Скомпилировать файл в PDF можно двумя способами:
%  1. Использовать pdfLaTeX (pdflatex), (LaTeX+DVIPS не работает);
%  2. Использовать LuaLaTeX (XeLaTeX будет работать тоже).
%  При использовании LuaLaTeX потребуются TTF- или OTF-шрифты CMU
%  (Computer Modern Unicode). Шрифты устанавливаются либо пакетом
%  дистрибутива LaTeX cm-unicode
%              (https://www.ctan.org/tex-archive/fonts/cm-unicode),
%  либо загрузкой и установкой в операционной системе, адрес страницы:
%              http://canopus.iacp.dvo.ru/%7Epanov/cm-unicode/
%  Второй вариант не будет работать в XeLaTeX.
%
%  В MiKTeX (дистрибутив LaTeX для ОС Windows):
%  1. Пакет cm-unicode устанавливается вручную в программе MiKTeX Console.
%  2. Для верстки данного примера, а именно, картинки-заглушки необходимо,
%     также вручную, загрузить пакет pgf. Этот пакет используется популярным
%     пакетом tikz.
%  3. Тест показал, что остальные пакеты MiKTeX грузит автоматически (если
%     ему разрешено автоматически грузить пакеты). Режим автозагрузки
%     настраивается в разделе Settings в MiKTeX Console.
%
%
%  Самый простой способ сверстать статью - использовать pdfLaTeX, но
%  окончательный вариант верстки сборника будет собран в LuaLaTeX,
%  так как результат получится лучшего качества, благодаря пакету microtype и
%  использованию векторных шрифтов OTF вместо растровых pdfLaTeX.
%
%  В случае возникновения вопросов и проблем с версткой статьи,
%  пишите письма на электронную почту: eugeneai@irnok.net, Черкашин Евгений.
%
%  Новые варианты корректирующего стиля будут доступны на сайте:
%        https://github.com/eugeneai/nla-style
%        файл - nla.sty
%
%  Дальнейшие инструкции - в тексте данного шаблона. Он одновременно
%  является примером статьи.
%
%  Формат LaTeX2e!

\documentclass[12pt]{llncs}  % Необходимо использовать шрифт 12 пунктов.

% При использовании pdfLaTeX добавляется стандартный набор русификации babel.
% Если верстка производится в LuaLaTeX, то следующие три строки надо
% закомментировать, русификация будет произведена в корректирующем стиле автоматом.
\usepackage{iftex}

\ifPDFTeX
\usepackage[T2A]{fontenc}
\usepackage[utf8]{inputenc} % Кодировка utf-8, cp1251 и т.д.
\usepackage[english,russian]{babel}
\fi

% Для верстки в LuaLaTeX текст готовится строго в utf-8!

% В операционной системе Windows для редактирования в кодировке utf-8
% можно использовать программы notepad++ https://notepad-plus-plus.org/,
% techniccenter http://www.texniccenter.org/,
% SciTE (самая маленькая по объему программа) http://www.scintilla.org/SciTEDownload.html
% Подойдет также и встроенный в свежий дистрибутив MiKTeX редактор
% TeXworks.

% Добавляется корректирующий стилевой файл строго после babel, если он был включен.
% В параметре необходимо указать russian, что установит не совсем стандартные названия
% разделов текста, настроит переносы для русского языка как основного и т.п.

\usepackage{todonotes} % Этот пакет нужен для верстки данного шаблона, его
                       % надо убрать из вашей статьи.

\usepackage[russian]{nla}

% Многие популярные пакеты (amsXXX, graphicx и т.д.) уже импортированы в корректирующий стиль.
% Если возникнут конфликты с вашими пакетами, попробуйте их отключить и сверстать
% текст как есть.
%
%


% Было б удобно при верстке сборника, чтобы названия рисунков разных авторов не пересекались.
% Чтоб минимизировать такое пересечение, рисунки можно поместить в отдельную подпапку с
% названием - фамилией автора или названием статьи.
%
% \graphicspath{{ivanov-petrov-pics/}} % Указание папки с изображениями в форматах png, pdf.
% или
% \graphicspath{{great-problem-solving-paper-pics/}}.


\begin{document}

% Текст оформляется в соответствии с классом article, используя дополнения
% AMS.
%


\fi


\title{Стационарные точки d.c. Лагранжианов в решении обратных задач теории управления\thanks{Работа выполнена при поддержке РФФИ (проект 20-01-00362)}}
% Первый автор
\author{Н.~Н.~Субботина   % \inst ставит циферку над автором.
  \and  % разделяет авторов, в тексте выглядит как запятая.
% Второй автор
  Е.~А.~Крупенников
  \and
} % обязательное поле

% Аффилиации пишутся в следующей форме, соединяя каждый институт при помощи \and.
\institute{ИММ УрО РАН, УрФУ, Екатеринбург, Россия \\
  \email{subb@uran.ru}, \email{krupennikov@imm.uran.ru}
% \and Другие авторы...
}

\maketitle

\begin{abstract}
Доклад посвящен задаче динамической реконструкции управлений (ЗДРУ) для динамических систем. В этой задаче по известным дискретным неточным замерам наблюдаемой траектории динамической системы нужно аппроксимировать наблюдаемую траекторию и управление, её породившее. Обсуждается новый подход к решению ЗДРУ, предложенный авторами доклада. Предлагаемое решение опирается на нахождение стационарных точек регуляризованных интегральных функционалов невязки во вспомогательных вариационных задачах. Ключевая особенность подхода --- использование так называемых d.c. Лагранжианов во вспомогательных задачах. Такой подход обеспечивает устойчивый колебательный характер аппроксимаций, а также устойчивость построенных конструкций к погрешностям измерений. В докладе обсуждаются свойства конструкций, получаемых во вспомогательных вариационных задачах.

\keywords{динамическая реконструкция, d.c. функции, вариационное исчисление} % в конце списка точка не ставится
\end{abstract}

\section{Основные результаты} % не обязательное поле

В докладе обсуждается новый метод решения задачи динамической реконструкции управлений (ЗДРУ) для динамических управляемых систем, предложенный~\cite{Subbotina:uran,Subbotina:mian} авторами доклада. В этой задаче по известным дискретным неточным замерам наблюдаемой траектории динамической системы нужно аппроксимировать эту траекторию и управление, её породившее.

В докладе рассматриваются детерминированные управляемые системы, аффинные по управлениям. Допустимые управления --- измеримые функции, значения которых ограничены известным выпуклым компактом.

Заметим, что задача реконструкции неизвестного управления некорректна, так как одна и та же траектория может порождаться не единственным допустимым управлением. Поэтому вводится понятие нормального управления --- управления, порождающего наблюдаемую траекторию и имеющего минимальную норму в пространстве $L^2$. Показано~\cite{Subbotina:mian}, что при типичных предположениях о входных данных для любой траектории, порожденной допустимым управлением, существует единственное нормальное управление. Под ЗДРУ подразумевается задача реконструкции именно нормального управления.

Авторами доклада предложен и обоснован поход~\cite{Subbotina:uran,Subbotina:mian} к решению ЗДРУ, опирающийся на нахождение стационарных точек регуляризованного~\cite{Subbotina:tikh} интегрального функционала невязки. Отличительная особенность подхода --- использование интегральных функционалов, в которых интегрант является d.c. функцией (т. е. разностью двух выпуклых функций. См., например, ~\cite{Subbotina:strek}).

\textbf{Постановка ЗДРУ.}
Наблюдается некоторая траектория $x^*(\cdot):[0,T]\to R^n$ динамической управляемой системы вида
\begin{equation}
\begin{gathered}
\dot x(t) = G(t,x(t))u(t) + f(t,x(t)),\quad x\in R^n,\quad u\in U\subset  R^m,\quad m\geq n,\quad t\in[0,T],
\end{gathered}
\end{equation}
где $x(\cdot)$ --- вектор фазовых переменных, $u(\cdot)$ --- вектор управлений, $T<\infty$, $U$ --- выпуклый компакт. Наблюдаемая траектория порождается неизвестным допустимым управлением.

Роль входных данных в ЗДРУ играют неточные дискретные замеры наблюдаемой траектории $\{y_i^\delta:\ i=0,\ldots,N\}$, имеющие погрешность $\delta>0$ и поступающие с шагом $h^\delta>0$ в моменты времени $t_i=i h^\delta,\ i=0,\ldots,N$.

ЗДРУ ставится следующим образом: для известных параметров $\delta$ и $h^\delta$ и наборов замеров $\{y_i^\delta\}$ построить кусочно-непрерывные управления $u^\delta(\cdot): [0,T]\to R^m$, равномерно ограниченные по $\delta$ и $h^\delta$ такие, что при стремлении к нулю параметров $\delta$ и $h^\delta$ эти управления сходятся слабо со звездой к нормальному управлению, а траектории системы~(1), порожденные этими управлениями, сходятся равномерно к наблюдаемой траектории $x^*(\cdot)$.

В работах~\cite{Subbotina:uran,Subbotina:mian} разработан и обоснован пошаговый алгоритм решения ЗДРУ. На каждом шаге алгоритма (т. е. на отрезке времени $[t_{i-1},t_i]$) используются вспомогательные конструкции из задачи на поиск стационарных точек функционалов вида
\begin{equation}
I(x(\cdot),u(\cdot)) = \int \limits_{t_{i-1}}^{t_i}
 -\frac{\|x(t) - y^\delta(t)\|^2}{2}+\frac{\alpha^2\|u(t)\|^2}{2}dt,\quad i=1,\ldots,N,
\end{equation}
где $i$ --- номер шага алгоритма, $\alpha$ --- малый регуляризирующий (по Тихонову~\cite{Subbotina:tikh}) параметр. Функция $y^\delta(t)$ является гладкой интерполяцией дискретных замеров.

Отличительной особенностью алгоритма является использование стационарных точек функционала (2). Иными словами, не требуется находить экстремум функционала (2).

В докладе обсуждаются свойства стационарных точек функционалов вида (2) и использование этих свойств для построения решения ЗДРУ. Приводятся модельные примеры, наглядно демонстрирующие эти свойства.

%

\begin{thebibliography}{9}

\bibitem{Subbotina:uran} Субботина Н.Н, Крупенников Е.А. Слабые со звездой аппроксимации решения задачи динамической реконструкции // Тр. Ин-та математики и механики УрО РАН. 2021. Т. 27, № 2. С. 208-220.

\bibitem{Subbotina:mian} Субботина Н.Н, Крупенников Е.А.  Слабое со звездой решение задачи динамической реконструкции. // Труды Математического института им. В.А.Стеклова РАН, 315. 2021. С. 247–260.

\bibitem{Subbotina:tikh} Тихонов~А.Н. Об устойчивости обратных задач. // Докл. АН СССР. 1943. Т.~39, №~5. C.~195--198.

\bibitem{Subbotina:strek} Стрекаловский~А.С. Элементы невыуклой оптимизации. Новосибирск.:~Наука,~2003.

\end{thebibliography}

% После библиографического списка в русскоязычных статьях необходимо оформить
% англоязычный заголовок.




%\end{document}

%%% Local Variables:
%%% mode: latex
%%% TeX-master: t
%%% End:
