\iffalse

%%%%%%%%%%%%%%%%%%%%%%%%%%%%%%%%%%%%%%%%%%%%%%%%%%%%%%%%%%%%%%%%%%%%%%%%
%
% This is the template file for the 6th International conference
% NONLINEAR ANALYSIS AND EXTREMAL PROBLEMS
% June 25-30, 2018
% Irkutsk, Russia
%
%%%%%%%%%%%%%%%%%%%%%%%%%%%%%%%%%%%%%%%%%%%%%%%%%%%%%%%%%%%%%%%%%%%%%%%%
% The preparation of the article is based on the standard llncs class
% (Lecture Notes in Computer Sciences), which is adjusted with style
% file of the conference.
%
% There are two ways of compilation of the file into PDF
% 1. Use pdfLaTeX (pdflatex), (LaTeX+DVIPS will not work);
% 2. Use LuaLaTeX (XeLaTeX will work too).
% When using LuaLaTeX You will need TTF or OTF CMU fonts
% (Computer Modern Unicode). The fonts are installed with 'cm-unicode' package in
% a distribution of LaTeX % (https://www.ctan.org/tex-archive/fonts/cm-unicode),
% either by downloading and installing these fonts system wide, the address of their page is
% http://canopus.iacp.dvo.ru/%7Epanov/cm-unicode/
% The second option won't work in XeLaTeX.
%
% For MiKTeX (LaTeX distribution for Windows),
%  1. Package 'cm-unicode' is installed manually with the MiKTeX administration Console.
%  2. For the compilation of this example, namely, the stub figure, one will also need to
% download package 'pgf' manually. This package uses in the popular
% package tikz.
%  3. Tests showed that the rest of the required packages MiKTeX loads automatically (if
%     it is allowed). The 'auto download' option is
%     configured in 'Settings' section in MiKTeX Console.
%
%
% The easiest way to compile an article is to use pdfLaTeX, but
% the final layout of the book will be compiled with LuaLaTeX,
% as a result will be of better quality thanks to the package 'microtype' and
% use vector OTF instead of standard raster fonts of pdfLaTeX.
%
% In the case of questions and problems with the article compilation,
% write letters to e-mail: eugeneai@irnok.net, Cherkashin Evgeny.
%
% New version of the correcting style file will be available at the website:
%     https://github.com/eugeneai/nla-style
%     file - nla.sty
%
% Further instructions are in the text body of the template. The template itself
% is an article example.
%
% The LaTeX2e format is used!

% 12 points font size is used.
\documentclass[12pt]{llncs}

% The correcting style file is added.
\usepackage{todonotes}

\usepackage{nla} % This package is needed for compiling
                 % this template, it should be removed
                 % from your article.

% Many popular packages (amsXXX, graphicx, etc.) are already imported in the style file.
% If there is a conflict with your packages, try disabling them and compile
% the text.
%
% It would be convenient in the layout of the proceedings if the file names
% of the figures of different authors do not clash.
% To minimize the clash, the drawings can be placed in a separate subfolder
% named after the author or the title of the paper.
%
% \graphicspath{{ivanov-petrov-pics/}} % specifies the folder with images in png, pdf formats.
% or
% \graphicspath{{great-problem-solving-paper-pics/}}.

\begin{document}

% Text should be formatted in accordance with the 'article' class, using extensions like
% AMS.
%
\fi

\title{A Sequential Approach to a Minimum Norm Partial Pole Assignment Problem}
% First author
\author{Bazaragchaa Barsbold\inst{1}
  \and
  Balkhuu Batbayasgalan\inst{1}
  \and
  Dovdon Batsuuri\inst{2}
  \and
  Dorjkhuu Enkhtaivan\inst{3}
}
\institute{School of Engineering and Applied Sciences\\ National University of Mongolia\\ 
	Ulaanbaatar, Mongolia\\
  \email{barsbold@seas.num.edu.mn}
  \and
Bussiness School\\ University of the Humanities\\
\email{batsuuri@humanities.mn}
\and
Alfa Agula LLC\\ 
\email{enkhtaivan@agula.mn}
}
% etc

\maketitle

\begin{abstract}
In this research we propose a numerical procedure for solving the partial eigenvalue
assignment problem with minimum norm feedback force. Our approach is based on 
sequential combination of techniques related to matrix optimization, projection and 
deflation.
\keywords{pole assignment, matrix optimization, projection, deflation}
\end{abstract}

% at the end of the list, there should be no final dot
%\section{The main results}


%The text of the report.

% The figures and tables are drawn according to the standard class 'article'.

\begin{thebibliography}{9}  
\bibitem{BND98} Datta B.N., Numerical Linear Algebra and Applications. Brook/Cole Publishing Co., Pacific Grove, California, 1998.
\bibitem{BND03} Datta B.N. Numerical Methods for Linear Control Systems Design and Analysis. Elsevier 
Academic Press, 2003.
\bibitem{chu96} Chu E.K. and Datta B.N. Numerically Robust Pole Assignment for the Second-Order 
Systems. Int. J. Control 1996. Vol. 4. Pp. 1113–-1127.
\bibitem{keel85}  Keel L.H., Fleming J.A. and Bhattacharyya S.P. Minimum norm pole assignment via 
Sylvester equation, Contemporary Mathematics  1985. Vol. 47. Pp. 265--272.
\end{thebibliography}
%\end{document}

%%% Local Variables:
%%% mode: latex
%%% TeX-master: t
%%% End:
