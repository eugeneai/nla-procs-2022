\begin{englishtitle}
\title{On the Reduction of a Singular Linear-quadratic 
	Control Problem to the Problem of Calculus of Variations\thanks{Результаты получены в рамках госзадания Минобрнауки России по проекту "Теория и методы исследования эволюционных уравнений и управляемых систем с их приложениями", \textnumero гос регистрации: ~121041300060-4.}}
% First author
\author{V.F. Chistyakov}
\institute{ISDCT SB RAS, Irkutsk, Russia\\
  \email{chist@icc.ru}
 }
% etc

\maketitle

\begin{abstract}

\keywords{optimal control, calculus of variations, differential algebraic equations}
\end{abstract}
\end{englishtitle}

\iffalse


\documentclass[12pt]{llncs}
\usepackage[T2A]{fontenc}
\usepackage[utf8]{inputenc}
\usepackage[english,russian]{babel}
\usepackage[russian]{nla}

%\usepackage[english,russian]{nla}

% \graphicspath{{pics/}} %Set the subfolder with figures (png, pdf).

%\usepackage{showframe}
\begin{document}
%\selectlanguage{russian}
\fi

\title{О сведении вырожденной линейно-квадратичной задачи управления к задачам вариационного исчисления}
% Первый автор
\author{В.~Ф.~Чистяков
 % Второй автор
%  И.~О.~Фамилия\inst{2}
%  \and
% Третий автор
%  И.~О.~Фамилия\inst{2}
} % обязательное поле
\institute{ИДСТУ СО РАН, Иркутск, Россия\\
  \email{chist@icc.ru}
}
% Другие авторы...

\maketitle

%\begin{abstract}
%В докладе рассматривается сведение вырожденной линейно-квадратичной задачи
%управления к задачам вариационного исчисления

%\keywords{оптимальное управление, вариационное исчисление, диф\-фе\-рен\-ци\-ально-алебраические уравнения}
%\end{abstract}

%\section{Основные результаты} % не обязательное поле

В последние два десятилетия появилось большое число публикаций, в которых исследуются
свойства так называемых ``анормальных''\ задач оптимального управления (ОУ) и вариационного исчисления (ВИ).
Под ``анормальными''\ задачами часто понимаются задачи ОУ или ВИ, у которых 
система обыкновенных дифференциальных уравнений (ОДУ), вытекающая из принципов максимума Понтрягина или Лагранжа имеет особенности.

Наиболее исследованным классом ``анормальных''\ задач  ОУ  являются линейно- квадратичные задачи, когда функционал качества
\begin{equation}\label{post1}
\Phi (u)=\int\limits_{\alpha}^{\beta}[\left\langle A(t)u,u \right\rangle +2
\left\langle B(t)u,x \right\rangle+\left\langle  C(t)x,x \right\rangle ]dt,
\end{equation}
а связи имеют вид линейной системы ОДУ
\begin{equation}\label{post2}
\Lambda_1x=A_1(t)\dot x+A_0(t)x= H(t)u,\ t\in T=[\alpha,\ \beta],
\end{equation}
\begin{equation}\label{post3}
x(\alpha)=a,
\end{equation}
где $A(t)-(m\times m)$-матрица, $B(t)-(n\times m)$-матрица,
$C(t)-(n\times n)$-матрица,
$\left\langle  .\ ,. \right\rangle -$скалярное
произведение в пространствах ${\bf R}^q,\ q=1,2,\ ...\ $,
$A_1(t),\  A_0(t)-(n\times n)$-матрицы,
$H(t)-(n\times m)$-матрицa, $x\equiv x(t)-n$-мерная вектор-функция состояния системы, гладкая на  $T$,
$\dot {x}\equiv dx/dt$, $u\equiv u(t)-m$-мерная вектор-функция управления, $a -$известный вектор из ${\bf R}^n$.
Обычно предполагают, что нарушается усиленное условие Лежандра на  дискретном множестве отрезка $T$ \cite{aru} или множестве $T_1\subseteq T$ с ненулевой мерой \cite{dmit2021}, \cite{kurmerz}.

В докладе предполагается,  что
\begin{equation}\label{post4}
1. \ \det A_1(t)=0 \ \forall t\in T,\ 2. \ \det A(t)=0 \ \forall t\in T,
\end{equation}
и входные данные в формулах (\ref{post1}), (\ref{post2}) обладают гладкостью,
необходимой при дальнейших рассуждениях.
Системы ОДУ вида (\ref{post2}), удовлетворяющие условию 1 из формулы (\ref{post4}), называют в настоящее время диф\-фе\-ренциально- ал\-геб\-раическими уравнениями (ДАУ).

При выполнении условий (\ref{post4}) система ОДУ, вытекающая из принципов максимума Понтрягина или Лагранжа является ДАУ.
Множество допустимых управлений в докладе имеет вид
\begin{equation}\label{post31}
u\equiv u(t)\in {\bf U}=\{u: {\bf C}^{\infty }(T), \ u^{(i)}(\alpha )=u^{(i)}(\beta )=0,\ i=\overline{0,q}\},
\end{equation}
где  $q-$некоторое целое число, $u^{(i)}(t)=(d/dt)^iu(t)$, $u^{(0)}(t)=u(t)$.



В ряде работ (см., например, \cite{kurmerz}) предполагается выпуклость
подынтегрального выражения в (\ref{post1}), что эквивалентно выполнению
соотношения
$
\begin{pmatrix}{A}(t) & {B(t)}\cr  {B}^\top(t) &
	{C}(t)\end{pmatrix}\geq 0\ \forall t\in T.
$
В докладе не предполагается выполнение этого условия и исследуется задача общего вида.

В докладе на основе теории ДАУ излагаются результаты, относительно свойств задачи (\ref{post1})-(\ref{post3}), удовлетворяющей условиям (\ref{post4}), начатые в статьях \cite{chistpesh}, \cite{chist2Fu2020}.

Основными задачами исследования являются:
\par \noindent
1) {\it  поиск условий, при выполнении которых  $\Phi (u)\geq 0\ \forall u\in {\bf U}$};
\par \noindent
2)
{\it поиск условий, при  которых из неравенства $\Phi (u)-\Phi (u^\ast)\leq \varepsilon ,$
	вытекают соотношения:\\ \centerline {$\Vert u^\ast(t)-u(t)\Vert _{{\bf L}_{2}(T)}^2\leq \kappa_0 \varepsilon $ или $\Vert x^\ast(t)-x(t)\Vert _{{\bf L}_{2}(T)}^2\leq \kappa_1\varepsilon $},
	\vskip 0.15cm
	\par \noindent
	где $\varepsilon\in [0,\varepsilon_0],\ \varepsilon_0 -$некоторое положительное число,
	$\kappa_0= const>0,\  \kappa_1= const>0$.}

Иначе говоря ищутся условия, при выполнении которых малым изменениям
функционала $\Phi (u)$ соответствуют малые  изменения управления или траектории.


% Список литературы.
\begin{thebibliography}{99}

\bibitem{aru} Арутюнов А.В. Условия экстремума. Анормальные и вырожденные
задачи. М.: Факториал, 1997.

\bibitem{dmit2021}
Дмитрук А.В.,  Мануйлович Н.А. {\it О минимизации вырожденных интегральных квадратичных функционалов}. Оптимальное управление и дифференциальные игры, Сборник статей., Труды МИАН. 2012. \textnumero 315.  C.~108–127.

\bibitem{kurmerz} Kurina G.A., M{\"a}rz R. {\it On linear-quadratic optimal control
problems for time varying descriptor systems}. SIAM J. Control Optim. 2006. Vol.~42. No ~6. Pp. 2062-2077.

\bibitem{chistpr1992}
Чистяков В.Ф. {\it О связи свойств вырожденной задачи вариационного исчисления и уравнения Якоби}.  Методы оптимизации. Новосибирск: Наука, 1992. С.~189-197.

\bibitem{chistpesh}
Чистяков В.Ф., Пешич М. {\it К вопросу о свойствах тождественно вырожденной задачи Лагранжа}. 
Автоматика и телемеханика. 2009. \textnumero 1. С.~85–103.

\bibitem{chist2Fu2020}
Chistyakov V.F., Chistyakova E.V., Fuong T.Z. {\it On the Relation between the Properties of a Degenerate Linear-Quadratic Control Problem and the Euler–Poisson Equation}. Computational Mathematics and Mathematical Physics. 2020. Vol. 60. No 3. Pp. 390-403. 
	


\end{thebibliography}






%\end{document}

%%% Local Variables:
%%% mode: latex
%%% TeX-master: t
%%% End:
