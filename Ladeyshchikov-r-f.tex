\begin{englishtitle} % Настраивает LaTeX на использование английского языка
% Этот титульный лист верстается аналогично.
\title{Time-optimal Problem on a Three-dimensional Heisenberg Group}
% First author
\author{E. Ladeyshchikov\inst{1}
  \and
  L. Lokutsievskiy\inst{2}
}
\institute{Lomonosov Moscow State University, Moscow, Russia\\
  \email{evgen310864@gmail.com}
  \and
Steklov Mathematical Institute RAS, Moscow, Russia\\
\email{lion.lokut@gmail.com}}
% etc

\maketitle

\begin{abstract}
In the present paper, we obtain explicit formulae for geodesics in a left-invariant Finsler problem on the Heisenberg group. As a unit ball, there will be an arbitrary convex set with 0 in its interior. The main assumption is that it allows parametrization with generalized spherical coordinates based on a convex trigonometry functions. However, the class under study is quite broad that includes, for example, all \(L_p\) norms. The paper fully describes all solutions of the Pontryagin's maximum principle, including both the solutions of the vertical subsystem and the projections of the extremals themselves onto the original group.
\keywords{Heisenberg group, Finsler geometry, Pontryagin's maximum principle} % в конце списка точка не ставится
\end{abstract}
\end{englishtitle}

\iffalse


%%%%%%%%%%%%%%%%%%%%%%%%%%%%%%%%%%%%%%%%%%%%%%%%%%%%%%%%%%%%%%%%%%%%%%%%
%
%  This is the template file for the 6th International conference
%  NONLINEAR ANALYSIS AND EXTREMAL PROBLEMS
%  June 25-30, 2018
%  Irkutsk, Russia
%
%%%%%%%%%%%%%%%%%%%%%%%%%%%%%%%%%%%%%%%%%%%%%%%%%%%%%%%%%%%%%%%%%%%%%%%%

%  Верстка статьи осуществляется на основе стандартного класса llncs
%  (Lecture Notes in Computer Sciences), который корректируется стилевым
%  файлом конференции.
%
%  Скомпилировать файл в PDF можно двумя способами:
%  1. Использовать pdfLaTeX (pdflatex), (LaTeX+DVIPS не работает);
%  2. Использовать LuaLaTeX (XeLaTeX будет работать тоже).
%  При использовании LuaLaTeX потребуются TTF- или OTF-шрифты CMU
%  (Computer Modern Unicode). Шрифты устанавливаются либо пакетом
%  дистрибутива LaTeX cm-unicode
%              (https://www.ctan.org/tex-archive/fonts/cm-unicode),
%  либо загрузкой и установкой в операционной системе, адрес страницы:
%              http://canopus.iacp.dvo.ru/%7Epanov/cm-unicode/
%  Второй вариант не будет работать в XeLaTeX.
%
%  В MiKTeX (дистрибутив LaTeX для ОС Windows):
%  1. Пакет cm-unicode устанавливается вручную в программе MiKTeX Console.
%  2. Для верстки данного примера, а именно, картинки-заглушки необходимо,
%     также вручную, загрузить пакет pgf. Этот пакет используется популярным
%     пакетом tikz.
%  3. Тест показал, что остальные пакеты MiKTeX грузит автоматически (если
%     ему разрешено автоматически грузить пакеты). Режим автозагрузки
%     настраивается в разделе Settings в MiKTeX Console.
%
%
%  Самый простой способ сверстать статью - использовать pdfLaTeX, но
%  окончательный вариант верстки сборника будет собран в LuaLaTeX,
%  так как результат получится лучшего качества, благодаря пакету microtype и
%  использованию векторных шрифтов OTF вместо растровых pdfLaTeX.
%
%  В случае возникновения вопросов и проблем с версткой статьи,
%  пишите письма на электронную почту: eugeneai@irnok.net, Черкашин Евгений.
%
%  Новые варианты корректирующего стиля будут доступны на сайте:
%        https://github.com/eugeneai/nla-style
%        файл - nla.sty
%
%  Дальнейшие инструкции - в тексте данного шаблона. Он одновременно
%  является примером статьи.
%
%  Формат LaTeX2e!

\documentclass[12pt]{llncs}  % Необходимо использовать шрифт 12 пунктов.

% При использовании pdfLaTeX добавляется стандартный набор русификации babel.
% Если верстка производится в LuaLaTeX, то следующие три строки надо
% закомментировать, русификация будет произведена в корректирующем стиле автоматом.
\usepackage{iftex}

\ifPDFTeX
\usepackage[T2A]{fontenc}
\usepackage[utf8]{inputenc} % Кодировка utf-8, cp1251 и т.д.
\usepackage[english,russian]{babel}

\usepackage{amsmath}
\newcommand{\Si}[1][\theta]{\sin_{\Omega}{#1}}

\newcommand{\So}[1][\theta^o]{\sin_{\Omega^o}{#1}}

\newcommand{\Ci}[1][\theta]{\cos_{\Omega}{#1}}

\newcommand{\Co}[1][\theta^o]{\cos_{\Omega^o}{#1}}

\newcommand{\Ti}[1][\theta]{\tan_{\Omega}{#1}}

\newcommand{\To}[1][\theta^o]{\tan_{\Omega^o}{#1}}

\newcommand{\ATi}{\arctan_{\Omega}\theta}

\newcommand{\ATo}[1][\theta^o]{\arctan_{\Omega^o}{#1}}
\fi

% Для верстки в LuaLaTeX текст готовится строго в utf-8!

% В операционной системе Windows для редактирования в кодировке utf-8
% можно использовать программы notepad++ https://notepad-plus-plus.org/,
% techniccenter http://www.texniccenter.org/,
% SciTE (самая маленькая по объему программа) http://www.scintilla.org/SciTEDownload.html
% Подойдет также и встроенный в свежий дистрибутив MiKTeX редактор
% TeXworks.

% Добавляется корректирующий стилевой файл строго после babel, если он был включен.
% В параметре необходимо указать russian, что установит не совсем стандартные названия
% разделов текста, настроит переносы для русского языка как основного и т.п.


\usepackage[russian]{nla}

% Многие популярные пакеты (amsXXX, graphicx и т.д.) уже импортированы в корректирующий стиль.
% Если возникнут конфликты с вашими пакетами, попробуйте их отключить и сверстать
% текст как есть.
%
%


% Было б удобно при верстке сборника, чтобы названия рисунков разных авторов не пересекались.
% Чтоб минимизировать такое пересечение, рисунки можно поместить в отдельную подпапку с
% названием - фамилией автора или названием статьи.
%
% \graphicspath{{ivanov-petrov-pics/}} % Указание папки с изображениями в форматах png, pdf.
% или
% \graphicspath{{great-problem-solving-paper-pics/}}.


\begin{document}

% Текст оформляется в соответствии с классом article, используя дополнения
% AMS.
%
\fi

\title{Задача быстродействия на трёхмерной группе Гейзенберга с управлением из выпуклого множества}
% Первый автор
\author{Е.~А.~Ладейщиков\inst{1}  % \inst ставит циферку над автором.
  \and  % разделяет авторов, в тексте выглядит как запятая.
% Второй автор
  Л.~В.~Локуциевский\inst{2}
  \and
} % обязательное поле

% Аффилиации пишутся в следующей форме, соединяя каждый институт при помощи \and.
\institute{МГУ им. Ломоносова, Москва, Россия \\
  \email{evgen310864@gmail.com}
  \and   % Разделяет институты и присваивает им номера по порядку.
МИАН им.Стеклова, Москва, Россия\\
  \email{lion.lokut@gmail.com}
% \and Другие авторы...
}

\maketitle

\begin{abstract}
В этой работе исследуется левоинвариантная финслерова задача на группе Гейзенберга. В качестве единичного шара случит произвольное выпуклое компактное трехмерное множество с 0 во внутренности, которое допускает введение обобщенных сферических координат на основе функций выпуклой тригонометрии. Вообще говоря, единичный шар некоторой нормы в трехмерном пространстве может не допускать введения таких координат. Однако, исследуемый класс является достаточно широким и включает в себя, например, все Lp нормы. В работе полностью описаны все решения принципа максимума Понтрягина, включая как решения вертикальной подсистемы, так и проекции самих экстремалей на исходную группу. 

\keywords{Группа Гейзенберга, финслерова геометрия, принцип максимума Понтрягина} % в конце списка точка не ставится
\end{abstract}

\section{Основные результаты} % не обязательное поле

Пусть \(\mathbb{H}_3\) - трёхмерная группа Гейзенберга - множество матриц вида 
\[\begin{pmatrix}
	1& q_1& q_2\\
	0& 1& q_3\\
	0& 0& 1\\
\end{pmatrix} \] со стандартной операцией матричного умножения и \(e\) - единичная матрица. Пусть \(\Omega \subset \mathbb{R}^2\) - компактное выпуклое множество, содержащее 0 во внутренности, \(m > 0, M > 0\), а \(f: [-m, M] \rightarrow \mathbb{R}\) такова, что \((-f)\) - непрерывна, выпукла, \(f(M) = f(-m) = 0\).  Положим \[U_e \subset T_eM,\ U_e = \{\begin{pmatrix}
	0& u& w\\
	0& 0& v\\
	0& 0& 0\\
\end{pmatrix}| (u, v) = f(w)\cdot\omega,\ \omega\in\Omega, w\in [-m, M] \}\]
Тогда получим, что  \(U_e\) выпукло, компактно (топология наследуется из \(\mathbb{R}^3\)) и содержит во внутренности 0(например, под определение подходит эллипсоид с центром в нуле, октаэдр, шар нормы \(L_p\)), поэтому существует конечная \(\mu_{U_e}(x)\ \forall x \in T_e\mathbb{H}_3\), где \(\mu_A\) - функция Минковского для множества \(A\). С помощью дифференциалов левых сдвигов \(L_q(q_0) = q\cdot q_0\) определим: \(U_q := dL|_q(U_e)\). Тогда для любого почти всюду дифференцируемого пути, у которого ограничен модуль скорости в евклидовой метрике \(q: [0, t] \rightarrow M\) получим существование длины \(\int_0^{t} \|\dot{q}\|dt < \infty\), где подразумевается \(\|\dot{q}(t)\| = \mu_{U_q(t)}(\dot{q}(t))\). Потребуем, чтобы в почти каждый момент времени \(\dot{q}(t) \in U_{q(t)}\). В таком случае, кривая \(q(t)\) почти всюду удовлетворяет ОДУ:
\[\dot{q} = q\cdot U(t),\ U(t) \in U_e,\]
где \("\cdot"\) - умножение матриц. Ставится задача быстродействия \[q(0) = e,\ q(t_0) = q_0,\ t_0\mapsto\min\]

% Рисунки и таблицы оформляются по стандарту класса article. Например,
После параметризации \(\omega\in\partial\Omega,\ \omega = (\Ci, \Si)\) выпуклыми тригонометрическими функциями, построенными по множеству \(\Omega\), удалось записать принцип максимума Понтрягина в более удобной форме. А после замены координат \(T^*\mathbb{H}_3 \leftrightarrow T^*_e\mathbb{H}_3\times \mathbb{H}_3\) - найти явные решения:

\[q_1(t) = c_3(\So[\theta^o(t)] - \So[\alpha_0])\]
\[q_2(t) = c_3( -\Co[\theta^o(t)] + \Co[\alpha_0])\]
\[q_3(t) = wt + c_1\So + \frac{1}{2}\theta^o(t) - \frac{1}{2}\So\Co + c_2,\]
где \(w, c_1, c_2, с_3, \alpha_0 -\) - некоторые константы, зависящие от начальных условий(определённым образом), а \(\theta^o(t)\) - аффинная функция от \(t\).

Таким образом, получается, что управление на экстремалях - это движение по поляре к слою множества \(U_e\): \(\{(u, v) = f(w_0)\cdot\omega,\ \omega\in\Omega, w_0\in [-m, M]\}^o\). Проекции этих траекторий на плоскость \((q_1, q_2)\) - растянутая в \(c_3\) раз \(\Omega^o\), а координата \(q_3\) - площадь, заметаемая этими проекциями на плоскость и двумя фиксированными лучами, в сумме с аффинным по \(t\) слагаемым.

% Современные издательства требуют использовать кавычки-елочки << >>.

% В конце текста можно выразить благодарности, если этого не было
% сделано в ссылке с заголовка статьи, например,
%

% Список литературы оформляется подобно ГОСТ-2008.
% Примеры оформления находятся по этому адресу -
%     https://narfu.ru/agtu/www.agtu.ru/fad08f5ab5ca9486942a52596ba6582elit.html
%

\begin{thebibliography}{9} % или {99}, если ссылок больше десяти.
\bibitem{Agrachev} Аграчев~A.A., Сачков~Ю.Л. Геометрическая теория управления. М.:~ФИЗМАТЛИТ,~2005. 
\bibitem{Lokutsievskiy} Локуциевский~Л.В. Выпуклая тригонометрия с приложениями к субфинслеровой геометрии. Матем. сб. 2019. Vol. 210, no. 8. Pp. 120--148.
\end{thebibliography}

% После библиографического списка в русскоязычных статьях необходимо оформить
% англоязычный заголовок.



%\end{document}


