\begin{englishtitle} % Настраивает LaTeX на использование английского языка
% Этот титульный лист верстается аналогично.
\title{On Exact Solutions of Equations Used in Modeling the Motion of Distributed Formations\thanks{Работа выполнена при поддержке РНФ, проект \textnumero~22-29-00819.}}
% First author
\author{Alexander Kosov 
  \and
  Edward Semenov 
}
\institute{	
ISDCT SB RAS, Irkutsk, Russia\\
  \email{kosov\_idstu@mail.ru, edwseiz@gmail.com}}
% etc

\maketitle

\begin{abstract}
Partial differential equations are considered, which are used in modeling the motion of formations with distributed characteristics. A method of reduction to ordinary differential equations is developed in order to construct exact solutions.

\keywords{equations of parabolic type, exact solutions, formation modeling} % в конце списка точка не ставится
\end{abstract}
\end{englishtitle}


\iffalse
\documentclass[12pt]{llncs}  % Необходимо использовать шрифт 12 пунктов.

% При использовании pdfLaTeX добавляется стандартный набор русификации babel.
% Если верстка производится в LuaLaTeX, то следующие три строки надо
% закомментировать, русификация будет произведена в корректирующем стиле автоматом.
\usepackage[T2A]{fontenc}
\usepackage[cp1251]{inputenc} % Кодировка utf-8, win1251 (cp1251) не тестировалась.
\usepackage[english,russian]{babel}

% Для верстки в LuaLaTeX текст готовится строго в utf-8!

% В операционной системе Windows для редактирования в кодировки utf-8
% можно использовать программы notepad++ https://notepad-plus-plus.org/,
% techniccenter http://www.texniccenter.org/,
% SciTE (самая маленькая по объему программа) http://www.scintilla.org/SciTEDownload.html
% Подойдет также и встроенный в свежий дистрибутив MiKTeX редактор
% TeXworks.

% Добавляется корректирующий стилевой файл строго после babel, если он был включен.
% В параметре необходимо указать russian, что установит не совсем стандартные названия
% разделов текста, настроит переносы для русского языка как основного и т.п.

\usepackage{todonotes} % Удрать из вашей статьи, нужен для верстки данного шаблона.

\usepackage[russian]{nla}

\begin{document}

% Текст оформляется в соответствии с классом article, используя дополнения
% AMS.
%
\fi

\title{О точных решениях уравнений, используемых при моделировании движения распределенных формаций}
% Первый автор
\author{А.~А.~Косов   % \inst ставит циферку над автором.
  \and  % разделяет авторов, в тексте выглядит как запятая.
% Второй автор
  Э.~И.~Семенов 
} % обязательное поле

% Аффилиации пишутся в следующей форме, соединяя каждый институт при помощи \and.
\institute{ИДСТУ СО РАН, Иркутск, Россия \\
  \email{kosov\_idstu@mail.ru, edwseiz@gmail.com}
% \and Другие авторы...
}

\maketitle

\begin{abstract}
Рассматриваются дифференциальные уравнения в частных производных, используемые при моделировании движения формаций с распределенными характеристиками. Развивается метод редукции к обыкновенным дифференциальным уравнениям с целью построения точных решений.\\
{{\bf Ключевые слова:} уравнения параболического типа, точные решения, моделирование формаций} % в конце списка точка не ставится
\end{abstract}

Дифференциальные уравнения в частных производных используются при моделировании движения формаций подвижных объектов с распределенными характеристиками. Так, в~\cite{wfj} для моделирования формации с управлением по принципу отклонения от движения лидера применялось уравнение параболического типа следующего вида
\begin{equation}\label{Eq1}
u_t=F\left(u,u_{x},u_{xx}\right),\quad t\geq 0,\quad x\in\mathbb{R}. 
\end{equation}
В докладе будут представлены результаты по построению новых точных решений уравнения типа (\ref{Eq1}). Основное внимание уделим случаю, когда 
$$
F\left(u,u_{x},u_{xx}\right)=u^{\sigma}u_{xx}+\sigma u^{\sigma-1}u_{x}^2,
$$
где $u=u(x,t)$, $\sigma\ne 0$~--- вещественный параметр нелинейности.

При учете времени на обмен информацией с лидером и формирование сигнала обратной связи возникает необходимость учитывать запаздывание в уравнениях~\cite{wfj}, поэтому наряду с уравнением (\ref{Eq1}) рассматривают также уравнение
\begin{equation}\label{Eq2}
u_t(t,x)=F\left(u(t-\tau,x),u(t,x),u_{x}(t,x),u_{xx}(t,x)\right),\quad t\geq 0,\quad\tau>0,\quad x\in\mathbb{R}. 
\end{equation}
В докладе будут представлены также результаты по построению точных решений для уравнений типа (\ref{Eq2}) с запаздыванием, а именно многомерного уравнения следующего вида.
$$
u_{t}=\nabla\cdot\left(u^{\lambda}\nabla u\right)+\alpha\bar{u},
$$
где $u=u({\bf x},t)$, $\bar{u}=u({\bf x},\bar{t})$, $\bar{t}=t-\tau$, ${\bf x}\in\mathbb{R}^n$; $\tau>0$; $\nabla$~--- градиент; $\lambda\ne 0$~--- вещественный параметр нелинейности среды; $\alpha\ne 0$~--- произвольная постоянная, знак которой характеризует либо источник (процесс выделения тепла), либо сток (процесс поглощения тепла). Отличительной особенностью этого уравнения от классического уравнения нелинейной теплопроводности является зависимость линейного источника (стока) $\bar{u}=u({\bf x},\bar{t})$ от запаздывающего аргумента $\bar{t}=t-\tau$. При этом процесс тепловыделения или поглощения тепла проходит с некоторым запаздыванием от текущего момента времени $t$ на заданную положительную величину $\tau$. Примером такого процесса является теплообмен в ядерном реакторе~\cite{gor1988,kb2008}.

\begin{thebibliography}{9} % или {99}, если ссылок больше десяти.

\bibitem{wfj}  Wei~J., Fridman~E., Johansson~K.H. A PDE approach to deployment of mobile agents under leader relative position measurements. Automatica. 2019. Vol.~106. Pp.~47--53.

\bibitem{zp2001} Зайцев В.Ф., Полянин А.Д.~Справочник по обыкновенным дифференциальным
уравнениям.~М.: ФИЗМАТЛИТ,~2001.~576~с.

\bibitem{gor1988} Горяченко В. Д.~Качественные методы в динамике ядерных реакторов.~М.: Энергоатомиздат.~1983.

\bibitem{kb2008} Кириллов П.Л., Богословская Г.П.~Тепломассообмен в ядерных энергетических установках. 2-е изд.~М.: ИздАТ.~2008.~256~c.

\end{thebibliography}




%\end{document}

%%% Local Variables:
%%% mode: latex
%%% TeX-master: t
%%% End:



