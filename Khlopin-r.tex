\begin{englishtitle}
		\title{On Necessary Conditions  if  Limits are Minimized
		%for Minimizing the Limits
		}
		% First author
		\author{Dmitry Khlopin}
		\institute{IMM UrB RAS,  Yekaterinburg, Russia\\
			\email{khlopin@imm.uran.ru}
		}
		% etc

		\maketitle

		\begin{abstract}

	An infinite-horizon optimal control problem	with asymptotic terminal constraints at infinity is considered. Under overtaking criteria the necessary boundary condition for co-state arcs at infinity is shown by	the stability of normal cones with respect to lower/upper limits.

\keywords{Infinite-horizon control problem, necessary conditions, Pontryagin maximum principle, overtaking optimal control} % в конце списка точка не ставится
\end{abstract}
	\end{englishtitle}




\title{О необходимых условиях при минимизации пределов %\thanks{Работа выполнена при поддержке РФФИ (РНФ, другие фонды), проект \textnumero~00-00-00000.}
}
% Первый автор
\author{Д.~В.~Хлопин
%  \and
%% Второй автор
%  И.~О.~Фамилия\inst{2}
%  \and
%% Третий автор
%  И.~О.~Фамилия\inst{2}
} % обязательное поле
\institute{ИММ УрО РАН, Екатеринбург, Россия\\
  \email{khlopin@imm.uran.ru}
%  \and
%Институт (название в краткой форме), Город, Страна\\
%\email{email}
}
% Другие авторы...

\maketitle

\begin{abstract}
	   Для задачи оптимального управления на бесконечном промежутке  показано необходимое краевое условие на сопряженную переменную, не требующее каких-либо дополнительных предположений. Это условие основано на устойчивости  нормальных конусов при переходе к верхнему/нижнему пределу.

\keywords{управление на бесконечном промежутке, необходимые условия, принцип максимума Понтрягина, обгоняющее управление}
\end{abstract}



	Рассмотрим задачу управления на бесконечном промежутке:
$$
\begin{array}{l}
	\min \int_{0}^\infty f_0(\tau,y(\tau),u(\tau))\, d\tau,\\[1em]
\dot{y}(t)=f(t,y(t),u(t)),  \quad y(0)=x_0\in\mathbb{R}^m
,\quad
\ \mathop{\hbox{ \rm{Limsup}}}_{\theta\uparrow\infty} \{l(y(\theta))\}\subset \mathcal{C}_\infty.
\end{array}
$$
%при помощи выбора борелевского  для
%с целью

Будем полагать, что $\mathcal{C}_\infty$ и $U$  замкнуты в $\mathbb{R}^m$ и некотором конечномерном евклидовом пространстве, функция $l:\mathbb{R}^m\to\mathbb{R}$ непрерывна,
отображения  $f:\mathbb{R}_+\times\mathbb{R}^m\times U\to\mathbb{R}^m$ и
$f_0:\mathbb{R}_+\times\mathbb{R}^m\times U\to\mathbb{R}$
непрерывны по $(x,u)$ и борелевские по~$t$.
Для простоты формулировки пусть также для каждого борелевского управления $u\in B(\mathbb{R}_+;U)$ найдется такая локально суммируемая функция $C_u\in B(\mathbb{R}_+;\mathbb{R}_+)$, что  для любых $(t,x)\in\mathbb{R}_+\times\mathbb{R}^m$
\[ \|f(t,x,u(t))\|^2+|f_0(t,x,u(t))|^2+\Big\|\frac{\partial f}{\partial x}(t,x,u(t))\Big\|^2+\Big\|\frac{\partial f_0}{\partial x}(t,x,u(t))\Big\|^2\leq C_u(t)(1+||x||).\]
Тогда каждое~$u\in B(\mathbb{R}_+;U)$  при любом начальном условии $y(0)=\bar{x}$ задает на $\mathbb{R}_+$ единственное  решение ${ \mathrm{y}}(\bar{x},u;\cdot)$, а значит и для всех $\theta>0$ накопленный к моменту $\theta$ платеж
\[      	J(\bar{x},u;\theta)=\int_{0}^{\theta} f_0\big(t,{ \mathrm{y}}(\bar{x},u;t),u(t)\big)\,dt\qquad\forall
\bar{x}\in\mathbb{R}^m,u\in B(\mathbb{R}_+;U),\theta\in\mathbb{R}_+. \]

Введем также
 %функцию Гамильтона---Понтрягина
%$H:{\mathbb{R}^m}\times{\mathbb{R}^m}^*\times {U}\times{\mathbb{R}_+}\times{\mathbb{R}_+}\to{\mathbb{R}}$
%правилом:
%для всех $(x,\psi,u,\lambda,t)\in{\mathbb{R}^{m}}\times{\mathbb{R}^{m,*}}\times {U}\times{\mathbb{R}_+}\times{\mathbb{R}_+}$
гамильтониан
$
H(x,\psi,u,\lambda,t)=\psi f\big(t,x,u\big)-\lambda
f_0\big(t,x,u\big).
$

{\bf Теорема 1}.
{\it 	 Пусть допустимый процесс $(\hat{y},\hat{u})$ слабо обгоняюще оптимален, то есть
	\[  \limsup_{\theta\uparrow\infty}\
	\big[J(x_0, u; \theta)-J(x_0, \hat{u}; \theta)\big]
	\geq 0\qquad \forall u\in B(\mathbb{R}_+,U).\]
%\en

	Тогда найдется ненулевая пара $({\psi},{\lambda})\in C(\mathbb{R}_+,\mathbb{R}^{m,*})\times\{0,1\}$,
	удовлетворяющая
	\begin{align*}
	\sup_{\upsilon\in
		U}H\big(\hat{y}(t),{\psi}(t),\upsilon,\lambda,t\big)
	=
	H\big(\hat{y}(t),{\psi}(t),\hat{u}(t),\lambda,t\big) \textrm{\ п.в.,}\qquad \textrm{(условие максимума)}\\
	\dot{\psi}(t)=-\frac{\partial
		{H}}{\partial x}\big(\hat{y}(t),{\psi}(t),\hat{u}(t),\lambda,t\big)  \textrm{\ п.в.},\qquad\textrm{(сопряженная система)}\\
	-{\psi}(0)\in \mathop{\hbox{  \rm{co}}} N(x;\mathcal{C}_{ \mathrm{as}})+\mathop{\hbox{  \rm{co}}}
	\mathop{\hbox{  \rm{Limsup}}}_{\substack
		{x_n\to x_0,\ \lambda_n\downarrow {\lambda},\ \theta_n\uparrow\infty\\ {J}(x_n,\hat{u};\theta_n)-{J}(x_0,\hat{u};\theta_n)\to 0}
	}
	\Big\{\lambda_n\frac{\partial {J}}{\partial x}(x_n,\hat{u};\theta_n)
	\Big\},\quad\textrm{(краевое условие)}
	\end{align*}
%	для почти всех неотрицательных $t>0$
%	удовлетворяющая условию максимума
%	\begin{equation*}%\label{H}
%	\sup_{\upsilon\in		U}H\big(\hat{y}(t),{\psi}(t),\upsilon,\lambda,t\big)
%	=H\big(\hat{y}(t),{\psi}(t),\hat{u}(t),\lambda,t\big),
%	\end{equation*}
%	и сопряженной системе
%	\begin{equation*}
%	\dot{\psi}(t)=-\frac{\partial
%		{H}}{\partial x}\big(\hat{y}(t),{\psi}(t),\hat{u}(t),\lambda,t\big),
%	%	\label{maxH}
%	\end{equation*}
%	 а также удовлетворяющая краевому условию
%	\begin{equation*}
%	-{\psi}(0)\in \mathop{\hbox{  \rm{co}}} N(x;\mathcal{C}_{ \mathrm{as}})+\mathop{\hbox{  \rm{co}}}
%	\mathop{\hbox{  \rm{Limsup}}}_{\substack
%		{x_n\to x_0,\ \lambda_n\downarrow {\lambda},\ \theta_n\uparrow\infty\\ {J}(x_n,\hat{u};\theta_n)-{J}(x_0,\hat{u};\theta_n)\to 0}
%	}
%	\Big\{\lambda_n\frac{\partial {J}}{\partial x}(x_n,\hat{u};\theta_n)
%	\Big\},
%	\end{equation*}
	где
	$\mathcal{C}_{ \mathrm{as}}=\big\{x\in\mathbb{R}^m\,\big|\,
	\mathop{\hbox{  \rm{Limsup}}}_{\theta\to \infty} \{l({ \mathrm{y}}(x,\hat{u};\theta))\}\subset \mathcal{C}_{\infty}\big\}$ ---
  множество тех  позиций, начинаясь с которых порождаемая $\hat{u}$ траектория, соблюдает терминальное условие.

  Более того, если процесс $(\hat{y},\hat{u})$ также обгоняюще оптимален, то есть
  \[ \liminf_{\theta\uparrow\infty}\
  \big[J(x_0, u; \theta)-J(\hat{y}(0), \hat{u}; \theta)\big]
  \geq 0\qquad \forall u\in B(\mathbb{R}_+,U),\]
  то для $(\hat{y},\hat{u})$ можно считать выполненным и  более сильное краевое условие
	\begin{equation*}
-{\psi}(0)\in N(x_0;\mathcal{C}_{ \mathrm{as}})+\bigcap_{\{(\theta_n)_{n\in\mathbb{N}}\in\mathbb{R}_{+}^{\mathbb{N}}\,|\,\theta_n\uparrow\infty\}}
\mathop{\hbox{  \rm{Limsup}}}_{\substack
	{x_n\to x_0,\ \lambda_n\downarrow {\lambda}\\ {J}(x_n,\hat{u};\theta_n)-{J}(x_0,\hat{u};\theta_n)\to 0}
}
\Big\{\lambda_n\frac{\partial {J}}{\partial x}(x_n,\hat{u};\theta_n)
\Big\}.
\label{khlopin_ALL2}
\end{equation*}
}






 Без условия трансверсальности необходимость принципа максимума Понтрягина хорошо известна; см. \cite{Halkin}. Основная особенность теоремы выше ---  краевые условия, необходимость которых не требует каких-либо асимптотических предположений.

 %Отметим, что в ряде случаев условие $(1)$ можно и упростить,
 %например, при $\mathcal{C}_{\infty}=\mathbb{R}^m$   нормальный %конус в условии~$(1)$ становится нулевым и его можно опустить.
 %Если  градиенты по $x$ от  ${J}(x,\hat{u};\theta)$ равномерно %ограничены по $(x,\theta)$, то можно присвоить $\lambda=1$ и не %брать  в $(1)$ выпуклую оболочку, дополнительно в верхнем %пределе потребовав  $x_n\in\mathcal{C}_{ \mathrm{as}}$.



  	Само доказательство теоремы~1 проводится также, как и \cite[Theorem~4.1,Theorem~4.6]{arxiv}, требуется лишь  для оценок сверху  предельных нормальных конусов верхних и нижних пределов множеств вместо \cite[Lemma~6.1,6.7]{arxiv},
  	%\cite[Theorem~3.1]{Mnghia},\cite[Theorem~4.8]{Peres},
  	\cite[Theorem~6.2]{Ledyaev} применить

	{\bf Теорема 2.} {\it       Пусть даны подмножества  $\Omega_\theta\subset\mathbb{R}^m$  $(\theta> 0)$ и
		точка $z\in\mathbb{R}^m$. Тогда,
				\begin{align*}
		N\Big(z;\mathop{  \rm{Liminf}}_{\theta\uparrow\infty}\Omega_\theta\Big)\subset
			\mathop{\hbox{  \rm{co}}}
		\mathop{  \rm{Limsup}}_{{z}_n\to z,\theta_n\uparrow\infty}  {N}({z}_n;\Omega_{\theta_n}).
		\end{align*}
		Более того, если при всех  $\theta> 0$
		точка $z$ принадлежит замыканию $\Omega_\theta$, то
				\begin{align*}
		N\Big(z; \mathop{  \rm{Limsup}}_{\theta\uparrow\infty} \Omega_\theta\Big)
		\subset
		\bigcap_{\{(\theta_n)_{n\in\mathbb{N}}\in\mathbb{R}_{+}^{\mathbb{N}}\,|\,\theta_n\uparrow\infty\}}
		        \mathop{  \rm{Limsup}}_{{z}_n\to {z}}{N}({z}_n;\Omega_{\theta_n}).
		        %\qquad\forall z\in\cup_{\theta>0} \mathop{  \rm{cl}}\Omega_\theta.
		\end{align*}
	}

	\begin{thebibliography}{99}
\bibitem{Halkin}
Halkin H. {    Necessary conditions for optimal control problems with infinite horizons}. Econometrica. 1974. Vol.~42. Pp.~267--272.

\bibitem{arxiv}
Khlopin~D.V. {  Necessary conditions in infinite-horizon control problem
	that need no asymptotic assumptions}. arXiv preprint arXiv:1910.12092

\bibitem{Ledyaev}
 Ledyaev Y.S., Treiman J.S. {  Sub-and supergradients of envelopes, semicontinuous clo-
sures, and limits of sequences of functions.} Russ. Math. Surv. 2012. Vol.~67. Pp.~345--373.
	\end{thebibliography}






%\end{document}
