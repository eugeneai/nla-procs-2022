\begin{englishtitle} % Настраивает LaTeX на использование английского языка
% Этот титульный лист верстается аналогично.
\title{On the Numerical Solution of Linear Multidimensional Differential-algebraic Systems\thanks{Работа выполнена в рамках базового проекта ``Теория и методы исследования эволюционных уравнений и управляемых систем с их приложениями'', \textnumero~121041300060-4.}}
% First author
\author{Svetlana Svinina}
\institute{ISDCT 	SB RAS, Irkutsk, Russia\\
  \email{svinina@icc.ru}}
% etc

\maketitle

\begin{abstract}
An initial-boundary value problem for a linear multidimensional differential-algebraic system of hyperbolic and parabolic types is considered. For its numerical solution, an additive spline-collocation difference scheme is used. The results of numerical experiments are presented.

\keywords{differential-algebraic system, additive difference schemes, index, spline-collocation method, hyperbolic and parabolic systems} 
\end{abstract}
\end{englishtitle}


\iffalse
\documentclass[12pt]{llncs}  
\usepackage[T2A]{fontenc}
\usepackage[cp1251]{inputenc} 
\usepackage[english,russian]{babel}
\usepackage{todonotes}
\usepackage[russian]{nla}

\begin{document}
\fi

\title{О численном решении линейных многомерных дифференциально-алгебраических систем}
\author{С.~В.~Свинина 
 }

% Аффилиации пишутся в следующей форме, соединяя каждый институт при помощи \and.
\institute{ ИДСТУ СО РАН, Иркутск, Россия\\
  \email{svinina@icc.ru}}

\maketitle

\begin{abstract}
Рассматривается начально-краевая задача для линейной многомерной диф\-фе\-рен\-ци\-аль\-но-алгебраической системы гиперболического и параболического типов. Для её численного решения применяются аддитивная сплайн-кол\-ло\-ка\-ци\-онная разностная схема. Представлены результаты численных экспериментов.


\keywords{дифференциально-алгебраическая система, аддитивные разностные схемы, индекс, сплайн-коллокационный метод, гиперболические и параболические системы} 
\end{abstract}

Среди уравнений газовой динамики хорошо известны системы с вырожденной матрицей при старшей производной. Такие системы называют не разрешёнными относительно старшей производной и относят к системам более общего вида, которые в современной литературе получили название дифференциально-алгебраических систем. 
В докладе рассматривается линейная многомерная дифференциально-алгебраическая система: 
\begin{equation}\label{e_1}
A(x,t) \frac{\partial u}{\partial t}+Lu=f(x,t), \ \ \mbox{где}\ \ Lu=\sum_{\alpha=1}^{p}B_{\alpha}(x,t) \frac{\partial u}{\partial x_{\alpha}} +C(x,t)u,
\end{equation}
заданная в цилиндре $\mathcal{U}_{T}=G\times\{t:\ t_{0}\leq t\leq T\}$, основанием которого является $p$-мерный прямоугольный параллелепипед
$G=\{x=(x_{1},x_{2},\dots,x_{p}):\ x_{\alpha,0}\leq x_{\alpha}\leq I_{\alpha},\ \ \alpha=\overline{1,p}\}$, $(x,t)\in \mathcal{U}_{T}$. 
В системе (\ref{e_1}) $f(x,t)$ и $u(x,t)$ -- заданная и искомая, соответственно, $n$-мерные  вектор-функ\-ции: 
$f(x,t)=(f^{1}(x,t),f^{2}(x,t),\dots,f^{n}(x,t))^{\top}$ и
$u(x,t)=(u^{1}(x,t),u^{2}(x,t),\dots,u^{n}(x,t))^{\top}$;
$A(x,t)$, $B_{\alpha}(x,t)$ и $C(x,t)$ -- заданные квадратные матрицы-функции порядка $n$:
$$
\det A(x,t)=0,\ \ \ \det B_{\alpha}(x,t)=0\ \ \forall \alpha=\overline{1,p},\ (x,t)\in \mathcal{U}_{T}.
$$
Начально-краевые условия имеют следующий вид:
$$
\begin{array}{c}
u(x,t_{0})=\phi(x), \\
u(x_{1},\dots,x_{\alpha,0},\dots,x_{p},t)=\psi_{\alpha}(x_{1},\dots,x_{\alpha-1},x_{\alpha+1},\dots,x_{p},t),
\ \ \alpha=\overline{1,p}.
\end{array}
$$
При численном решении многомерных систем наиболее популярны методы расщепления~\cite{Sam}-\cite{Vab}. К ним относят методы переменных направлений и локально-одномерные методы. 
В этом случае система (\ref{e_1}) записывается в расщеплённом виде
$$
\sum_{\alpha=1}^{p}\mathfrak{L}_{\alpha}u=0,\ \ \ \mathfrak{L}_{\alpha}u=\frac{1}{p}A(x,t)\frac{\partial u}{\partial t}+L_{\alpha}u-f_{\alpha}(x,t), 
$$
где
$L_{\alpha}u=B_{\alpha}(x,t) \frac{\partial u}{\partial x_{\alpha}} +C_{\alpha}(x,t)u$,
$\sum_{\alpha=1}^{p}C_{\alpha}(x,t)=C(x,t),\ \ \ \sum_{\alpha=1}^{p}f_{\alpha}(x,t)=f(x,t)$.
На каждом отрезке $\Delta_{\alpha}$, где $\alpha=\overline{1,p}$ последовательно решаются одномерные диф\-фе\-рен\-ци\-аль\-но-алгебраические системы
\begin{equation}\label{e_2} 
\sum_{\alpha=1}^{p}\mathfrak{L}_{\alpha}v_{(\alpha)}=\frac{1}{p}A(x,t)\frac{\partial v_{(\alpha)}}{\partial t}+L_{\alpha}v_{(\alpha)}-f_{\alpha}(x,t)=0,\ \ x\in G,\ \ t\in \Delta_{\alpha},\ \ \alpha=\overline{1,p}
\end{equation}
с начально-краевыми условиями: 
$$
v_{(\alpha)}(x_{1},\dots,x_{\alpha, 0},\dots,x_{p},t)=\psi_{\alpha}(x_{1},\dots,x_{\alpha-1},x_{\alpha+1},\dots,x_{p},t),
$$
$$
v_{(1)}(x,t_{0})=\phi(x),\ \ \ \ v_{(\alpha)}(x,t_{i_{0}+(\alpha-1)/p})=v_{(\alpha-1)}(x,t_{i_{0}+(\alpha-1)/p}),\ \ \ \alpha=\overline{2,p},
$$
$$
v_{(1)}(x,t_{i_{0}})=v_{(p)}(x,t_{i_{0}}).
$$
Каждая система из (\ref{e_2}) аппроксимируется сплайн-коллокационной разносной схемой~\cite{Sv}. В итоге получаются разностные схемы, которые названы аддитивными сплайн-кол\-ло\-ка\-ци\-он\-ными разностными схемами. Такие разностные схемы имеют высокий порядок точности по временной и пространственным переменным. 

\begin{thebibliography}{9} 
\bibitem{Sam} Самарский~А.А. Об одном экономичном разностном методе решения многомерного параболического уравнения в произвольной области~// Ж. вычисл. матем. и матем. физ. 1962. Т.~2. \textnumero~5. C.~787--811.
\bibitem{Kov} Ковеня~В.М. Разностные методы решения многомерных задач: Курс лекций~/ Новосиб. гос. ун-т. Новосибирск,~2004.
\bibitem{Vab} Вабищевич~П.Н. Аддитивные операторно-разностные схемы (схемы расщепления). М.:~КРАСАНД,~2013.
\bibitem{Sv}  Гайдомак С.В. Об устойчивости неявной сплайн-коллокационной разностной схемы для линейных дифференциально-алгебраических уравнений с частными производными~// Ж. вычисл. матем. и матем. физ. 2013.  Т.~53. \textnumero~9. C.~1460--1479.   
\end{thebibliography}

% После библиографического списка в русскоязычных статьях необходимо оформить
% англоязычный заголовок.




%\end{document}

%%% Local Variables:
%%% mode: latex
%%% TeX-master: t
%%% End:
