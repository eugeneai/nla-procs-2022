\title{Impulse Response Matrix for Time-Varying System of Differential-Algebraic Equations}
\author{Alla A. Shcheglova}
\institute{ISDCT SB RAS, Irkutsk, Russia\\
  \email{shchegl@icc.ru}
}


\maketitle

\begin{abstract}
\keywords{differential-algebraic equations, impulse response  matrix, minimal realization}
\end{abstract}


A range of questions related to the impulse response matrix [1] for a system of linear differential-algebraic equations (DAE) [2] is considered. For systems with infinitely differentiable coefficients, it is shown that this matrix is represented as a sum of the impulse response matrices of the differential and algebraic subsystems.  The form of  non-degenerate change of variables has been found, which does not affect the view of impulse response matrix.  Realizations of this matrix are proposed to construct  in the class of index 1 DAE, which are separated into differential and algebraic parts. The necessary and sufficient conditions for the realisability  of an impulse response matrix in the class of algebraic systems are obtained. The questions of the methods of construction and  the dimension of the minimal realization of such a matrix are considered under various assumptions.

\begin{thebibliography}{9}
\bibitem{reff1}
{ D'Angelo H.} Linear systems with variable parameters.  Moscow:  Mashinostroenie, 1974.  
\bibitem{reff2}
{ Shcheglova A.A.} The solvability of the initial problem for a degenerate linear hybrid system with variable coefficients. Russian Mathematics.  2010. No. 9.  Pp. 49--61.  
\end{thebibliography}
