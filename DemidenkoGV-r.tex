\begin{englishtitle} % Настраивает LaTeX на использование английского языка
% Этот титульный лист верстается аналогично.
\title{Existence of Periodic Solutions for One Class of~Systems of Differential Equations\thanks{Работа
выполнена в рамках государственного задания
Института математики им.~С.\,Л.~Соболева СО РАН 
(проект \textnumero~FWNF-2022-0008).
}}
% First author
\author{Gennadii Demidenko%\inst{1,2}
}
\institute{Sobolev Institute of Mathematics SB RAS, Novosibirsk, Russia \\
%  \and
\email{demidenk@math.nsc.ru}}
% etc

\maketitle

\begin{abstract}
We consider systems of nonlinear ordinary differential equations, the linear part of which has periodic coefficients and is exponentially dichotomous. Estimates of the dichotomy parameters are obtained, an analog of the perturbation theorem for exponential dichotomy is proved, conditions for the existence of periodic solutions to nonlinear systems, as well as their stability, are established.

\keywords{periodic solutions, exponential dichotomy,
projectors, Lyapunov differential equation} % в конце списка точка не ставится
\end{abstract}
\end{englishtitle}

\iffalse
%%%%%%%%%%%%%%%%%%%%%%%%%%%%%%%%%%%%%%%%%%%%%%%%%%%%%%%%%%%%%%%%%%%%%%%%
%
%  This is the template file for the 6th International conference
%  NONLINEAR ANALYSIS AND EXTREMAL PROBLEMS
%  June 25-30, 2018
%  Irkutsk, Russia
%
%%%%%%%%%%%%%%%%%%%%%%%%%%%%%%%%%%%%%%%%%%%%%%%%%%%%%%%%%%%%%%%%%%%%%%%%

%  Верстка статьи осуществляется на основе стандартного класса llncs
%  (Lecture Notes in Computer Sciences), который корректируется стилевым
%  файлом конференции.
%
%  Скомпилировать файл в PDF можно двумя способами:
%  1. Использовать pdfLaTeX (pdflatex), (LaTeX+DVIPS не работает);
%  2. Использовать LuaLaTeX (XeLaTeX будет работать тоже).
%  При использовании LuaLaTeX потребуются TTF- или OTF-шрифты CMU
%  (Computer Modern Unicode). Шрифты устанавливаются либо пакетом
%  дистрибутива LaTeX cm-unicode
%              (https://www.ctan.org/tex-archive/fonts/cm-unicode),
%  либо загрузкой и установкой в операционной системе, адрес страницы:
%              http://canopus.iacp.dvo.ru/%7Epanov/cm-unicode/
%  Второй вариант не будет работать в XeLaTeX.
%
%  В MiKTeX (дистрибутив LaTeX для ОС Windows):
%  1. Пакет cm-unicode устанавливается вручную в программе MiKTeX Console.
%  2. Для верстки данного примера, а именно, картинки-заглушки необходимо,
%     также вручную, загрузить пакет pgf. Этот пакет используется популярным
%     пакетом tikz.
%  3. Тест показал, что остальные пакеты MiKTeX грузит автоматически (если
%     ему разрешено автоматически грузить пакеты). Режим автозагрузки
%     настраивается в разделе Settings в MiKTeX Console.
%
%
%  Самый простой способ сверстать статью - использовать pdfLaTeX, но
%  окончательный вариант верстки сборника будет собран в LuaLaTeX,
%  так как результат получится лучшего качества, благодаря пакету microtype и
%  использованию векторных шрифтов OTF вместо растровых pdfLaTeX.
%
%  В случае возникновения вопросов и проблем с версткой статьи,
%  пишите письма на электронную почту: eugeneai@irnok.net, Черкашин Евгений.
%
%  Новые варианты корректирующего стиля будут доступны на сайте:
%        https://github.com/eugeneai/nla-style
%        файл - nla.sty
%
%  Дальнейшие инструкции - в тексте данного шаблона. Он одновременно
%  является примером статьи.
%
%  Формат LaTeX2e!

\documentclass[12pt]{llncs}  % Необходимо использовать шрифт 12 пунктов.

% При использовании pdfLaTeX добавляется стандартный набор русификации babel.
% Если верстка производится в LuaLaTeX, то следующие три строки надо
% закомментировать, русификация будет произведена в корректирующем стиле автоматом.
%\usepackage{iftex}

%\ifPDFTeX
\usepackage[T2A]{fontenc}
\usepackage[utf8]{inputenc} % Кодировка utf-8, cp1251 и т.д.
\usepackage[english,russian]{babel}
%\fi

% Для верстки в LuaLaTeX текст готовится строго в utf-8!

% В операционной системе Windows для редактирования в кодировке utf-8
% можно использовать программы notepad++ https://notepad-plus-plus.org/,
% techniccenter http://www.texniccenter.org/,
% SciTE (самая маленькая по объему программа) http://www.scintilla.org/SciTEDownload.html
% Подойдет также и встроенный в свежий дистрибутив MiKTeX редактор
% TeXworks.

% Добавляется корректирующий стилевой файл строго после babel, если он был включен.
% В параметре необходимо указать russian, что установит не совсем стандартные названия
% разделов текста, настроит переносы для русского языка как основного и т.п.

%\usepackage{todonotes} % Этот пакет нужен для верстки данного шаблона, его
                       % надо убрать из вашей статьи.

\usepackage[russian]{nla}

% Многие популярные пакеты (amsXXX, graphicx и т.д.) уже импортированы в корректирующий стиль.
% Если возникнут конфликты с вашими пакетами, попробуйте их отключить и сверстать
% текст как есть.
%
%


% Было б удобно при верстке сборника, чтобы названия рисунков разных авторов не пересекались.
% Чтоб минимизировать такое пересечение, рисунки можно поместить в отдельную подпапку с
% названием - фамилией автора или названием статьи.
%
% \graphicspath{{ivanov-petrov-pics/}} % Указание папки с изображениями в форматах png, pdf.
% или
% \graphicspath{{great-problem-solving-paper-pics/}}.


\begin{document}

% Текст оформляется в соответствии с классом article, используя дополнения
% AMS.
%
\fi

\title{Существование периодических решений
для одного класса систем дифференциальных уравнений}
% Первый автор
\author{Г.~В.~Демиденко%\inst{1,2}  % \inst ставит циферку над автором.
} % обязательное поле

% Аффилиации пишутся в следующей форме, соединяя каждый институт при помощи \and.
\institute{Институт математики им. С.\,Л. Соболева СО РАН, Новосибирск, Россия \\
%  \and   % Разделяет институты и присваивает им номера по порядку.
  \email{demidenk@math.nsc.ru}
% \and Другие авторы...
}

\maketitle

\begin{abstract}
В работе рассматриваются системы нелинейных обыкновенных дифференциальных уравнений, линейная часть которых имеет периодические коэффициенты и экспоненциально дихотомична. Получены оценки параметров дихотомии, доказан аналог теорем о возмущении для экспоненциальной дихотомии, установлены условия существования периодических решений нелинейных систем, а также их устойчивость.

\keywords{периодические решения, экспоненциальная дихотомия,
проекторы, дифференциальное уравнение Ляпунова} % в конце списка точка не ставится
\end{abstract}

%\section{Основные результаты} % не обязательное поле

В работе рассматриваются системы дифференциальных уравнений вида 
$$                                                               
\frac{dx}{dt} = A(t)x + f(t,x), \quad  -\infty < t< \infty,
\eqno(1)
$$
где элементы матриц 
$A(t)$ 
размера 
$n\times n$ 
являются непрерывными
$T$-периодическими
функциями, а непрерывная вектор-функция 
$f(t,x)$ 
локально удовлетворяет условию Липшица по 
$x$ 
$$
\|f(t,x^1) - f(t,x^2)\| \le L\|x^1 - x^2\|
$$
и следующим условиям:
$$
f(t + T, x) \equiv f(t,x), \quad \|f(t,x)\| \le q(1 + \|x\|)^{\omega}, 
$$
где 
$q > 0$,
$\omega \ge 0$~---  
константы. Будем предполагать, что однородная линейная система 
экспоненциально дихотомична (см., например, [1]). 

В работе [2] был установлен критерий экспоненциальной дихотомии системы 
линейных дифференциальных уравнений с периодическими коэффициентами.
А именно, если 
$Q(t) \in C[0,T]$~---
эрмитова матрица, удовлетворяющая условию 
$$
\int\limits_0^T Q(s) ds > 0, 
$$
и 
$X(T)$~--- матрица монодромии (1),  
то экспоненциальная дихотомия системы (1) эквивалентна существованию 
эрмитовой матрицы 
$H(t)$ 
и матрицы  
$P$,   
являющихся решением задачи 
$$
\left\{ \begin{array}{l}
	\displaystyle
	\frac{d}{dt}H + HA(t) + A^*(t)H = - \left(X^{-1}(t)\right)^*P^*Q(t)PX^{-1}(t)\\
        \\
        + (X^{-1}(t))^*(I - P)^*Q(t)(I - P)X^{-1}(t), \quad 0 < t < T,\\
        \\
        H(0) = H(T) > 0,\\
        \\
        H(0) = P^*H(0)P + (I - P)^*H(0)(I - P),\\
        \\
        P^2 = P, \quad PX(T) = X(T)P.
        \end{array}
        \right.
$$
Отметим, что этот критерий является аналогом соответствующих утверждений 
о задаче дихотомии для систем дифференциальных уравнений с постоянными 
коэффициентами.

Используя этот критерий экспоненциальной дихотомии, мы получаем оценки параметров 
дихотомии, доказываем аналог теорем о возмущении [3] для экспоненциальной дихотомии, 
устанавливаем условия существования периодических решений нелинейных систем (1), 
а также их устойчивость [4].
                                                                          	
\begin{thebibliography}{9}

\bibitem{dem1}
Далецкий Ю.Л., Крейн М.Г.
Устойчивость решений дифференциальных уравнений в банаховом пространстве.
М.: Наука, 1970.

\bibitem{dem2}
Demidenko G.V.
On conditions for exponential dichotomy of systems of 
linear differential equations with periodic coefficients.
Int. J. Dyn. Syst. Differ. Equ. 2016. Vol.~6, No.~1. Pp.~63--74.

\bibitem{dem3}
Демиденко Г.В.
Cистемы дифференциальных уравнений с периодическими коэффициентами~//
Сиб. журн. индустр. матем. 2013. Т.~16, \textnumero~4. С.~38--46.

\bibitem{dem4}
Демиденко Г.В.
Об одном классе систем дифференциальных уравнений
с периодическими коэффициентами в линейных членах~//
Сиб. мат. журн. 2021. Т.~62, \textnumero~5. С.~995--1012.

\end{thebibliography}

% После библиографического списка в русскоязычных статьях необходимо оформить
% англоязычный заголовок.



%\end{document}

%%% Local Variables:
%%% mode: latex
%%% TeX-master: t
%%% End:
