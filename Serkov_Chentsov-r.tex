
\begin{englishtitle} % Настраивает LaTeX на использование английского языка
% Этот титульный лист верстается аналогично.
\title{On a property of continuous dependence of sets\\ in the space of measures}
% First author
\author{Dmitrii Serkov\inst{}
  \and
  Alexander Chentsov\inst{}
}
\institute{Krasovskii institute of mathematics and mechanics, Yekaterinburg, RF\\
  \email{serkov@imm.uran.ru}, \email{chentsov@imm.uran.ru}}
% etc

\maketitle

\begin{abstract}
With the aim of applying to program constructions in the theory of differential games and other guarantee optimization problems, we study the dependence of the set of Borel measures on the product $X\times Y$ of metric compact sets on their marginals from some $*$-weak compact set of measures on $X$.
The continuity of such a dependence in the Hausdorff metric generated by the $*$-weak metric on the union of the specified sets of Borel measures is shown to be continuous under varying marginals in the $*$-weak compact set.
\keywords{Borel measure, Hausdorff metric} % в конце списка точка не ставится
\end{abstract}
\end{englishtitle}

\iffalse

%%%%%%%%%%%%%%%%%%%%%%%%%%%%%%%%%%%%%%%%%%%%%%%%%%%%%%%%%%%%%%%%%%%%%%%%
%
%  This is the template file for the 6th International conference
%  NONLINEAR ANALYSIS AND EXTREMAL PROBLEMS
%  June 25-30, 2018
%  Irkutsk, Russia
%
%%%%%%%%%%%%%%%%%%%%%%%%%%%%%%%%%%%%%%%%%%%%%%%%%%%%%%%%%%%%%%%%%%%%%%%%

%  Верстка статьи осуществляется на основе стандартного класса llncs
%  (Lecture Notes in Computer Sciences), который корректируется стилевым
%  файлом конференции.
%
%  Скомпилировать файл в PDF можно двумя способами:
%  1. Использовать pdfLaTeX (pdflatex), (LaTeX+DVIPS не работает);
%  2. Использовать LuaLaTeX (XeLaTeX будет работать тоже).
%  При использовании LuaLaTeX потребуются TTF- или OTF-шрифты CMU
%  (Computer Modern Unicode). Шрифты устанавливаются либо пакетом
%  дистрибутива LaTeX cm-unicode
%              (https://www.ctan.org/tex-archive/fonts/cm-unicode),
%  либо загрузкой и установкой в операционной системе, адрес страницы:
%              http://canopus.iacp.dvo.ru/%7Epanov/cm-unicode/
%  Второй вариант не будет работать в XeLaTeX.
%
%  В MiKTeX (дистрибутив LaTeX для ОС Windows):
%  1. Пакет cm-unicode устанавливается вручную в программе MiKTeX Console.
%  2. Для верстки данного примера, а именно, картинки-заглушки необходимо,
%     также вручную, загрузить пакет pgf. Этот пакет используется популярным
%     пакетом tikz.
%  3. Тест показал, что остальные пакеты MiKTeX грузит автоматически (если
%     ему разрешено автоматически грузить пакеты). Режим автозагрузки
%     настраивается в разделе Settings в MiKTeX Console.
%
%
%  Самый простой способ сверстать статью - использовать pdfLaTeX, но
%  окончательный вариант верстки сборника будет собран в LuaLaTeX,
%  так как результат получится лучшего качества, благодаря пакету microtype и
%  использованию векторных шрифтов OTF вместо растровых pdfLaTeX.
%
%  В случае возникновения вопросов и проблем с версткой статьи,
%  пишите письма на электронную почту: eugeneai@irnok.net, Черкашин Евгений.
%
%  Новые варианты корректирующего стиля будут доступны на сайте:
%        https://github.com/eugeneai/nla-style
%        файл - nla.sty
%
%  Дальнейшие инструкции - в тексте данного шаблона. Он одновременно
%  является примером статьи.
%
%  Формат LaTeX2e!

\documentclass[12pt]{llncs}  % Необходимо использовать шрифт 12 пунктов.

% При использовании pdfLaTeX добавляется стандартный набор русификации babel.
% Если верстка производится в LuaLaTeX, то следующие три строки надо
% закомментировать, русификация будет произведена в корректирующем стиле автоматом.
\usepackage{iftex}

\ifPDFTeX
\usepackage[T2A]{fontenc}
\usepackage[utf8]{inputenc} % Кодировка utf-8, cp1251 и т.д.
\usepackage[english,russian]{babel}
\fi

% Для верстки в LuaLaTeX текст готовится строго в utf-8!

% В операционной системе Windows для редактирования в кодировке utf-8
% можно использовать программы notepad++ https://notepad-plus-plus.org/,
% techniccenter http://www.texniccenter.org/,
% SciTE (самая маленькая по объему программа) http://www.scintilla.org/SciTEDownload.html
% Подойдет также и встроенный в свежий дистрибутив MiKTeX редактор
% TeXworks.

% Добавляется корректирующий стилевой файл строго после babel, если он был включен.
% В параметре необходимо указать russian, что установит не совсем стандартные названия
% разделов текста, настроит переносы для русского языка как основного и т.п.

\usepackage{todonotes} % Этот пакет нужен для верстки данного шаблона, его
                       % надо убрать из вашей статьи.

\usepackage[russian]{nla}

% Многие популярные пакеты (amsXXX, graphicx и т.д.) уже импортированы в корректирующий стиль.
% Если возникнут конфликты с вашими пакетами, попробуйте их отключить и сверстать
% текст как есть.
%
%


% Было б удобно при верстке сборника, чтобы названия рисунков разных авторов не пересекались.
% Чтоб минимизировать такое пересечение, рисунки можно поместить в отдельную подпапку с
% названием - фамилией автора или названием статьи.
%
% \graphicspath{{ivanov-petrov-pics/}} % Указание папки с изображениями в форматах png, pdf.
% или
% \graphicspath{{great-problem-solving-paper-pics/}}.


\begin{document}

% Текст оформляется в соответствии с классом article, используя дополнения
% AMS.
%
\fi

\title{Об одном свойстве непрерывной зависимости множеств\\ в пространстве мер}%\thanks{Работа выполнена при поддержке РФФИ (РНФ, другие фонды), проект \textnumero~00-00-00000.}
% Первый автор
\author{Д.~А.~Серков\inst{}  % \inst ставит циферку над автором.
  \and  % разделяет авторов, в тексте выглядит как запятая.
% Второй автор
А.~Г.~Ченцов\inst{}
  \and
} % обязательное поле

% Аффилиации пишутся в следующей форме, соединяя каждый институт при помощи \and.
\institute{ИММ УрО РАН, Екатеринбург, РФ\\
  \email{serkov@imm.uran.ru},  \email{chentsov@imm.uran.ru}
}

\maketitle

\begin{abstract}
С целью применения к программным конструкциям в теории дифференциальных игр и других задачах оптимизации гарантии, изучается зависимость множества борелевских мер на произведении $X\times Y$ метрических компактов от их маргиналов из некоторого $*$-слабого компакта мер на $X$.
Показана непрерывности такой зависимости в метрике Хаусдорфа, порождённой $*$-слабой метрикой на объединении указанных множеств борелевских мер, при изменении маргиналов в $*$-слабом компакте.

\keywords{борелевская мера, метрика Хаусдорфа} % в конце списка точка не ставится
\end{abstract}

%\section{Основные результаты} % не обязательное поле

Рассматривается декартово произведение $X\times Y$ двух непустых метрических компактов $X$, $Y$ с $\sigma$-алгебрами своих борелевских подмножеств.
На компакте $X$ задан непустой компакт $M$ борелевских (неотрицательных) мер в относительной *-слабой топологии (в этом случае каждая борелевская мера регулярна (см., например, \cite[гл. 1. \S1]{Bil})).
Каждой мере $\mu$ из этого компакта сопоставляется множество $\mathfrak{N}_\mu$ всех борелевских мер на произведении компактов $X\times Y$ с общим маргинальным распределением в виде указанной меры $\mu$;
данное множество назовем программой, отвечающей данной мере $\mu$ на метрическом компакте $X$.
Пространство знакопеременных борелевских мер на произведении $X\times Y$ оснащаем *-слабой топологией.
Объединение всех программ оказывается *-слабо замкнутым и сильно ограниченным в силу (*-слабой) компактности множества мер $M$.
Поскольку пространство непрерывных функций на произведении метрических компактов сепарабельно, то (*-слабо компактное) объединение программ, рассматриваемое как подпространство пространства (знакопеременных) борелевских мер на $X\times Y$, метризуемо \cite[Теорема 1.3.12]{Warga}.
Фиксируя некоторую метрику $\rho$, порождающую указанную относительную $*$-слабую топологию на объединении программ, получаем метрический компакт, на непустых замкнутых подмножествах которого вводим метрику Хаусдорфа.
В частности, оценивая в этой метрике различие программ $\mathfrak{N}_\mu$, $\mu\in M$, показываем, что зависимость упомянутых программ $\mathfrak{N}_\mu$ от порождающих эти программы маргинальных мер $\mu\in M$ непрерывна.

Данное утверждение существенно обобщает положение \cite[Лемма П.2]{Chentsov}, использовавшееся в конструкциях решения регулярных дифференциальных игр \cite[глава VII]{KraSub};
добавим к этому свойство непрерывной зависимости пучков обобщённых траекторий в дифференциальных играх при изменении порождающих эти пучки управлений мер одного из игроков \cite{Chentsov-VINITI}.
Свойства такого рода существенны в конструкциях \cite{KraSub} решения нелинейных дифференциальных игр, использующих понятие программного максимина.
Данные конструкции, в свою очередь, являются глубоким развитием понятия регулярности для линейных дифференциальных игр, предложенного в  \cite{Kra}.



% Рисунки и таблицы оформляются по стандарту класса article. Например,

% Современные издательства требуют использовать кавычки-елочки << >>.

% В конце текста можно выразить благодарности, если этого не было
% сделано в ссылке с заголовка статьи, например,
%Работа выполнена при поддержке РФФИ (РНФ, другие фонды), проект \textnumero~00-00-00000.
%

% Список литературы оформляется подобно ГОСТ-2008.
% Примеры оформления находятся по этому адресу -
%     https://narfu.ru/agtu/www.agtu.ru/fad08f5ab5ca9486942a52596ba6582elit.html
%

\begin{thebibliography}{9} % или {99}, если ссылок больше десяти.

\bibitem{Bil}Биллингсли П. Сходимость вероятностных мер. М.: Наука, 1977.

\bibitem{Warga} Варга Д. Оптимальное управление дифференциальными и функциональными уравнениями. М.: Наука, 1977.

\bibitem{Chentsov} Ченцов А. Г. Об одной игровой задаче управления на минимакс // Изв. АН СССР. Сер. Техн. кибернетика. 1975. № 1. С. 39--46.

\bibitem{KraSub} Красовский Н. Н., Субботин A. И. Позиционные дифференциальные игры. М.: Наука. 1974.

\bibitem{Chentsov-VINITI} Ченцов A. Г. Метод программных итераций для дифференциальной игры сближения–уклонения // Уральский политехнический институт им. С.М.Кирова. Свердловск. Рукопись депонирована в ВИНИТИ 1933–79. 1979.

\bibitem{Kra} Красовский Н. Н. Игровые задачи о встрече движений. М.: Наука, 1970.

\end{thebibliography}

% После библиографического списка в русскоязычных статьях необходимо оформить
% англоязычный заголовок.




%\end{document}

%%% Local Variables:
%%% mode: latex
%%% TeX-master: t
%%% End:
