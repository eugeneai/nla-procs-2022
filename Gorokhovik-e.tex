\iffalse

%%%%%%%%%%%%%%%%%%%%%%%%%%%%%%%%%%%%%%%%%%%%%%%%%%%%%%%%%%%%%%%%%%%%%%%%
%
% This is the template file for the 6th International conference
% NONLINEAR ANALYSIS AND EXTREMAL PROBLEMS
% June 25-30, 2018
% Irkutsk, Russia
%
%%%%%%%%%%%%%%%%%%%%%%%%%%%%%%%%%%%%%%%%%%%%%%%%%%%%%%%%%%%%%%%%%%%%%%%%
% The preparation of the article is based on the standard llncs class
% (Lecture Notes in Computer Sciences), which is adjusted with style
% file of the conference.
%
% There are two ways of compilation of the file into PDF
% 1. Use pdfLaTeX (pdflatex), (LaTeX+DVIPS will not work);
% 2. Use LuaLaTeX (XeLaTeX will work too).
% When using LuaLaTeX You will need TTF or OTF CMU fonts
% (Computer Modern Unicode). The fonts are installed with 'cm-unicode' package in
% a distribution of LaTeX % (https://www.ctan.org/tex-archive/fonts/cm-unicode),
% either by downloading and installing these fonts system wide, the address of their page is
% http://canopus.iacp.dvo.ru/%7Epanov/cm-unicode/
% The second option won't work in XeLaTeX.
%
% For MiKTeX (LaTeX distribution for Windows),
%  1. Package 'cm-unicode' is installed manually with the MiKTeX administration Console.
%  2. For the compilation of this example, namely, the stub figure, one will also need to
% download package 'pgf' manually. This package uses in the popular
% package tikz.
%  3. Tests showed that the rest of the required packages MiKTeX loads automatically (if
%     it is allowed). The 'auto download' option is
%     configured in 'Settings' section in MiKTeX Console.
%
%
% The easiest way to compile an article is to use pdfLaTeX, but
% the final layout of the book will be compiled with LuaLaTeX,
% as a result will be of better quality thanks to the package 'microtype' and
% use vector OTF instead of standard raster fonts of pdfLaTeX.
%
% In the case of questions and problems with the article compilation,
% write letters to e-mail: eugeneai@irnok.net, Cherkashin Evgeny.
%
% New version of the correcting style file will be available at the website:
%     https://github.com/eugeneai/nla-style
%     file - nla.sty
%
% Further instructions are in the text body of the template. The template itself
% is an article example.
%
% The LaTeX2e format is used!

% 12 points font size is used.



\documentclass[12pt]{llncs}

% The correcting style file is added.
\usepackage{todonotes}

\usepackage{nla} % This package is needed for compiling
                 % this template, it should be removed
                 % from your article.

% Many popular packages (amsXXX, graphicx, etc.) are already imported in the style file.
% If there is a conflict with your packages, try disabling them and compile
% the text.
%
% It would be convenient in the layout of the proceedings if the file names
% of the figures of different authors do not clash.
% To minimize the clash, the drawings can be placed in a separate subfolder
% named after the author or the title of the paper.
%
% \graphicspath{{ivanov-petrov-pics/}} % specifies the folder with images in png, pdf formats.
% or
% \graphicspath{{great-problem-solving-paper-pics/}}.

\begin{document}

\fi

\title{Separation of Convex Sets by Halfspaces with Applications to Convex Optimization Problems\thanks{The work is supported by the National Program for Scientific Research of
the Republic of Belarus for 2021--2025 ``Convergence--2025'', project No.~1.3.01.}}
% First author
\author{Valentin Gorokhovik }
  

\institute{Institute of Mathematics of the National Academy of Sciences of Belarus, Minsk, Belarus\\
  \email{gorokh@im.bas-net.by}}
  
\maketitle

\begin{abstract}
We introduce the class of step-affine functions defined on a real vector space and establish the duality
between step-affine functions and half-spaces, i.e., convex sets whose complements are convex as well. Using
this duality, we prove that two convex sets are disjoint if and only if they are separated by some step-affine
function. This criterion is actually the analytic version of the Kakutani-Tukey criterion of the separation of
disjoint convex sets by halfspaces. As applications of these results, we derive a minimality criterion for solutions
of convex vector optimization problems considered in real vector spaces without topology and an optimality
criterion for admissible points in classical convex programming problems not satisfying the Slater regularity
condition.

\keywords{convex sets, separation, halfspaces, vector optimization, convex programming problem}
\end{abstract}

\section{The main results}

The main tools that usually are used to derive optimality
conditions for admissible solutions of convex optimization
problems are the theorems on separation of convex subsets by
hyper\-pla\-nes. In most cases there is a gap between necessary
optimality conditions and sufficient ones obtained in such a way.
To overcome this gap some regularity conditions (like the Slater
regularity condition in the convex programming problem) are
additionally assumed. The main purpose of this talk is to present
a characterization of optimal solutions of convex optimization
problems which holds without any additional assumptions of
regularity. To this end we use the theorem on separation of convex
sets by halfspaces and its analytical versions.

A subset $H$ of a real vector space $X$ is called a halfspace if
both $H$ and its complement $X \setminus H$ are convex. In the
thirties of the last century S. Kakutani \cite{Kakutani} and J.W. Tukey \cite{Tukey} proved
independently the following general separation theorem: any two
convex subsets of $X,$ say $A$ and $B,$  are disjoint if and only
if there exists a halfspace $H$ in $X$ such that $A \subset H$ and
$B \subset {X \setminus H}$. For the case when $X$ is a
finite-dimensional  vector space J.-E. Martinez-Legaz and
I.~Singer \cite{M-L_S88} found an analytical version of this theorem. However
their approach can not be extended directly to the
infinite-dimensional case. In the talk we present such analytical
version of the Kakutani---Tukey theorem that is suitable both for
finite-dimensional and infinite-dimensional cases and then we
apply it to the study of convex optimization problems. We begin
with a study of geometric structure and classification of
infinite-dimensional halfspaces. Then we introduce and study a new
class of real-valued functions called step-affine and show that
functions of this class are dual counterparts of halfspaces. Using
this duality we reformulate the Kakutani---Tukey theorem in an
analytical form as a theorem on separation of convex sets by
step-affine functions. As applications of this theorem we derive
an analytical characterization of optimality for a convex vector
optimization problem. Besides we prove a criterion of optimality
for a convex programming problem which extends the well-known
Kuhn--Tucker criterion to the problems not holding the Slater
regularity condition.

More detailed presentation of above results can be found in \cite{G1,G2}.

%

\begin{thebibliography}{9} % or {99}, if there is more than ten references.
\bibitem{Kakutani}
Kakutani S. Ein Beweis des Satzen von M.~Eidelheit \"{u}ber konvexe Mengen. Proc. Imp. Acad. Tokio. 1938. Vol.~14. Pp.~93--94.

\bibitem{Tukey}
Tukey J.W. Some notes on the separation of convex sets. Portugaliae Math. 1942 Vol.~3. Pp.~95--102.

\bibitem{M-L_S88}
Martinez-Legaz J.-E., Singer I. The structure of hemispaces in ${\mathbb R}^n$. Linear Algebra Appl. 1988. Vol.~110. Pp.~117--179.

\bibitem{G1} Gorokhovik V.V., Shinkevich E.A. Geometric structure and classification of infinite-dimensional halfspaces.  Algebraic Analysis and Related Topics. Banach Center Publications. Vol.~53. Warsaw: Institute of Mathematics PAN, 2000. Pp.~121--138.

\bibitem{G2}  Gorokhovik V.V. Step-affine functions, halfspaces, and separation of convex sets with applications to convex optimization problems. Proc. Steklov Inst. Math. 2021. Vol.~313, Suppl.~1. Pp.~S83--S99.
    

\end{thebibliography}
%\end{document}

%%% Local Variables:
%%% mode: latex
%%% TeX-master: t
%%% End:
