
\begin{englishtitle} % Настраивает LaTeX на использование английского языка
% Этот титульный лист верстается аналогично.
\title{On Numerical Solution of the Second Order Differential-algebraic Equations}
% First author
\author{Liubov Solovarova\inst{1}
  \and
  Ta Duy Phuong\inst{2}
}
\institute{ISDCT SB RAS, Irkutsk, Russia\\
  \email{soleilu@mail.ru}
  \and
Institute of Mathematics of the Vietnamese Academy of Science and Technology, Hanoi, Vietnam\\
\email{tdphuong@math.ac.vn}}
% etc

\maketitle

\begin{abstract}
A multistep method and its version based on a reformulated notation of the original problem are investigated for numerical solution of the second order differential-algebraic equations.

\keywords{differential-algebraic equations, high order} % в конце списка точка не ставится
\end{abstract}
\end{englishtitle}

\iffalse

%%%%%%%%%%%%%%%%%%%%%%%%%%%%%%%%%%%%%%%%%%%%%%%%%%%%%%%%%%%%%%%%%%%%%%%%
%
%  This is the template file for the 6th International conference
%  NONLINEAR ANALYSIS AND EXTREMAL PROBLEMS
%  June 25-30, 2018
%  Irkutsk, Russia
%
%%%%%%%%%%%%%%%%%%%%%%%%%%%%%%%%%%%%%%%%%%%%%%%%%%%%%%%%%%%%%%%%%%%%%%%%

%  Верстка статьи осуществляется на основе стандартного класса llncs
%  (Lecture Notes in Computer Sciences), который корректируется стилевым
%  файлом конференции.
%
%  Скомпилировать файл в PDF можно двумя способами:
%  1. Использовать pdfLaTeX (pdflatex), (LaTeX+DVIPS не работает);
%  2. Использовать LuaLaTeX (XeLaTeX будет работать тоже).
%  При использовании LuaLaTeX потребуются TTF- или OTF-шрифты CMU
%  (Computer Modern Unicode). Шрифты устанавливаются либо пакетом
%  дистрибутива LaTeX cm-unicode
%              (https://www.ctan.org/tex-archive/fonts/cm-unicode),
%  либо загрузкой и установкой в операционной системе, адрес страницы:
%              http://canopus.iacp.dvo.ru/%7Epanov/cm-unicode/
%  Второй вариант не будет работать в XeLaTeX.
%
%  В MiKTeX (дистрибутив LaTeX для ОС Windows):
%  1. Пакет cm-unicode устанавливается вручную в программе MiKTeX Console.
%  2. Для верстки данного примера, а именно, картинки-заглушки необходимо,
%     также вручную, загрузить пакет pgf. Этот пакет используется популярным
%     пакетом tikz.
%  3. Тест показал, что остальные пакеты MiKTeX грузит автоматически (если
%     ему разрешено автоматически грузить пакеты). Режим автозагрузки
%     настраивается в разделе Settings в MiKTeX Console.
%
%
%  Самый простой способ сверстать статью - использовать pdfLaTeX, но
%  окончательный вариант верстки сборника будет собран в LuaLaTeX,
%  так как результат получится лучшего качества, благодаря пакету microtype и
%  использованию векторных шрифтов OTF вместо растровых pdfLaTeX.
%
%  В случае возникновения вопросов и проблем с версткой статьи,
%  пишите письма на электронную почту: eugeneai@irnok.net, Черкашин Евгений.
%
%  Новые варианты корректирующего стиля будут доступны на сайте:
%        https://github.com/eugeneai/nla-style
%        файл - nla.sty
%
%  Дальнейшие инструкции - в тексте данного шаблона. Он одновременно
%  является примером статьи.
%
%  Формат LaTeX2e!

\documentclass[12pt]{llncs}  % Необходимо использовать шрифт 12 пунктов.

% При использовании pdfLaTeX добавляется стандартный набор русификации babel.
% Если верстка производится в LuaLaTeX, то следующие три строки надо
% закомментировать, русификация будет произведена в корректирующем стиле автоматом.
\usepackage{iftex}

\ifPDFTeX
\usepackage[T2A]{fontenc}
\usepackage[utf8]{inputenc} % Кодировка utf-8, cp1251 и т.д.
\usepackage[english,russian]{babel}
\fi

% Для верстки в LuaLaTeX текст готовится строго в utf-8!

% В операционной системе Windows для редактирования в кодировке utf-8
% можно использовать программы notepad++ https://notepad-plus-plus.org/,
% techniccenter http://www.texniccenter.org/,
% SciTE (самая маленькая по объему программа) http://www.scintilla.org/SciTEDownload.html
% Подойдет также и встроенный в свежий дистрибутив MiKTeX редактор
% TeXworks.

% Добавляется корректирующий стилевой файл строго после babel, если он был включен.
% В параметре необходимо указать russian, что установит не совсем стандартные названия
% разделов текста, настроит переносы для русского языка как основного и т.п.

%\usepackage{todonotes} % Этот пакет нужен для верстки данного шаблона, его
                       % надо убрать из вашей статьи.

\usepackage[russian]{nla}

% Многие популярные пакеты (amsXXX, graphicx и т.д.) уже импортированы в корректирующий стиль.
% Если возникнут конфликты с вашими пакетами, попробуйте их отключить и сверстать
% текст как есть.
%
%


% Было б удобно при верстке сборника, чтобы названия рисунков разных авторов не пересекались.
% Чтоб минимизировать такое пересечение, рисунки можно поместить в отдельную подпапку с
% названием - фамилией автора или названием статьи.
%
% \graphicspath{{ivanov-petrov-pics/}} % Указание папки с изображениями в форматах png, pdf.
% или
% \graphicspath{{great-problem-solving-paper-pics/}}.


\begin{document}

% Текст оформляется в соответствии с классом article, используя дополнения
% AMS.
%
\fi
\title{О численном решении дифференциально-алгебраических уравнений второго порядка%\thanks{}
}
% Первый автор
\author{Л.~С.~Соловарова\inst{1}  % \inst ставит циферку над автором.
  \and  % разделяет авторов, в тексте выглядит как запятая.
% Второй автор
  Т.~З.~Фыонг\inst{2}
  } % обязательное поле

% Аффилиации пишутся в следующей форме, соединяя каждый институт при помощи \and.
\institute{ИДСТУ СО РАН, Иркутск, Россия\\
  \email{soleilu@mail.ru}
  \and   % Разделяет институты и присваивает им номера по порядку.
Институт математики Вьетнамской Академии Наук и Технологий, Ханой, Вьетнам\\
\email{tdphuong@math.ac.vn}
% \and Другие авторы...
}

\maketitle

\begin{abstract}
Для численного решения дифференциально-алгебраических уравнений второго порядка исследуются многошаговый метод и его вариант, основанный на переформулированной записи исходной задачи. 

\keywords{дифференциально-алгебраические уравнения, высокий порядок} % в конце списка точка не ставится
\end{abstract}

Обыкновенные дифференциальные уравнения (ОДУ) различных порядков один из основных инструментов для моделирования важных прикладных задач. Если все уравнения одинакового порядка, то они образуют систему ОДУ. Если процесс или явление описываются взаимосвязанными ОДУ различных порядков и трансцендентными  (конечномерными) уравнениями, то, объединяя их, получим систему ОДУ с тождественно вырожденной матрицей перед старшей производной. Такие системы принято называть дифференциально-алгебраическими уравнениями (ДАУ). Если порядок такой системы выше первого, то их называют ДАУ высокого порядка. 

К настоящему времени бурно развиваются качественная теория и численные методы решения ДАУ первого порядка (как для начальной, так и для краевой задачи). Для ДАУ высокого порядка обычно применяют следующий стандартный подход \cite{MehDAE2}. Путем введения новой вектор-функции размерности $nk$ ($n$ -- размерность исходной задачи, $k$ -- порядок ДАУ) записывают эту задачу в виде ДАУ первого порядка. Такое преобразование имеет ряд недостатков: увеличивает размерность в $k$ раз и значительно ухудшает свойства полученной задачи.

В докладе для ДАУ второго порядка вида
\begin{equation}									\label{dae2nd}
A(t)x^{''}(t)+B(t)x^{'}(t)+C(t)x(t)=f(t), \;t\in[0,1],
\end{equation}
\begin{equation}									\label{ic2nd}
x(0)=x_{0},x^{'}(t)|_{t=0}=x_{0}^{'},
\end{equation}
где $A(t),B(t),C(t)$ -- $(n\times n)$-матрицы, $f(t)$ и $x(t)$ -- заданная и искомая $ n $-мерные вектор-функции, соответственно, $x_{0}, x_{0}^{'}\in  R^{n}$, $detA(t)\equiv 0$, 
предлагаются многошаговые методы, построение которых основано на идеях из \cite{BulLee}.

\begin{thebibliography}{9} % или {99}, если ссылок больше десяти.

\bibitem{MehDAE2}
Mehrmann~V., Shi~C. Transformation of high order linear differential-algebraic systems to first order. Numerical Algorithms. 2006. \textnumero~42. Pp.~281--307.

\bibitem{BulLee}
Булатов М.В., Ли Минг Гонг, Соловарова Л.С. О разностных схемах первого и второго порядков для дифференциально-алгебраических уравнений индекса не выше двух~//Журнал вычисл. математики и матем. физики. 2010. Т. 50, \textnumero~11. С.~1909--1918. 

\end{thebibliography}

% После библиографического списка в русскоязычных статьях необходимо оформить
% англоязычный заголовок.



%\end{document}

%%% Local Variables:
%%% mode: latex
%%% TeX-master: t
%%% End:
