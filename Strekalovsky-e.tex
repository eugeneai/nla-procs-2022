\iffalse
%%%%%%%%%%%%%%%%%%%%%%%%%%%%%%%%%%%%%%%%%%%%%%%%%%%%%%%%%%%%%%%%%%%%%%%%
%
% This is the template file for the 6th International conference
% NONLINEAR ANALYSIS AND EXTREMAL PROBLEMS
% June 25-30, 2018
% Irkutsk, Russia
%
%%%%%%%%%%%%%%%%%%%%%%%%%%%%%%%%%%%%%%%%%%%%%%%%%%%%%%%%%%%%%%%%%%%%%%%%
% The preparation of the article is based on the standard llncs class
% (Lecture Notes in Computer Sciences), which is adjusted with style
% file of the conference.
%
% There are two ways of compilation of the file into PDF
% 1. Use pdfLaTeX (pdflatex), (LaTeX+DVIPS will not work);
% 2. Use LuaLaTeX (XeLaTeX will work too).
% When using LuaLaTeX You will need TTF or OTF CMU fonts
% (Computer Modern Unicode). The fonts are installed with 'cm-unicode' package in
% a distribution of LaTeX % (https://www.ctan.org/tex-archive/fonts/cm-unicode),
% either by downloading and installing these fonts system wide, the address of their page is
% http://canopus.iacp.dvo.ru/%7Epanov/cm-unicode/
% The second option won't work in XeLaTeX.
%
% For MiKTeX (LaTeX distribution for Windows),
%  1. Package 'cm-unicode' is installed manually with the MiKTeX administration Console.
%  2. For the compilation of this example, namely, the stub figure, one will also need to
% download package 'pgf' manually. This package uses in the popular
% package tikz.
%  3. Tests showed that the rest of the required packages MiKTeX loads automatically (if
%     it is allowed). The 'auto download' option is
%     configured in 'Settings' section in MiKTeX Console.
%
%
% The easiest way to compile an article is to use pdfLaTeX, but
% the final layout of the book will be compiled with LuaLaTeX,
% as a result will be of better quality thanks to the package 'microtype' and
% use vector OTF instead of standard raster fonts of pdfLaTeX.
%
% In the case of questions and problems with the article compilation,
% write letters to e-mail: eugeneai@irnok.net, Cherkashin Evgeny.
%
% New version of the correcting style file will be available at the website:
%     https://github.com/eugeneai/nla-style
%     file - nla.sty
%
% Further instructions are in the text body of the template. The template itself
% is an article example.
%
% The LaTeX2e format is used!

% 12 points font size is used.
\documentclass[12pt]{llncs}

% The correcting style file is added.
\usepackage{todonotes}

\usepackage{nla} % This package is needed for compiling
                 % this template, it should be removed
                 % from your article.

% Many popular packages (amsXXX, graphicx, etc.) are already imported in the style file.
% If there is a conflict with your packages, try disabling them and compile
% the text.
%
% It would be convenient in the layout of the proceedings if the file names
% of the figures of different authors do not clash.
% To minimize the clash, the drawings can be placed in a separate subfolder
% named after the author or the title of the paper.
%
% \graphicspath{{ivanov-petrov-pics/}} % specifies the folder with images in png, pdf formats.
% or
% \graphicspath{{great-problem-solving-paper-pics/}}.

\begin{document}

% Text should be formatted in accordance with the 'article' class, using extensions like
% AMS.
%

\fi

\title{On Nonconvex Optimal Control Problems\thanks{The research is supported by the Ministry of Education and Science of the Russian Federation (state registration No. 121041300065-9).}}
% First author
\author{Alexander Strekalovsky}
\institute{Matrosov Institute for System Dynamics and Control Theory of SB of RAS, Irkutsk, Russia\\ 
\email{strekal@icc.ru}}
% etc


\maketitle

\begin{abstract}
We address an optimal control (OC) problem along a state-linear control system  with inequality constraints given by the Bolza functionals with state-DC data. Therefore, the problem turns out to be nonconvex, i.e. possesses a large number of locally optimal controls and Pontryaguin's extremals. 
Using the Exact Penalization Theory, the original Problem is reduced to an auxiliary penalized Problem, which is also state-DC. 
For the latter OC problem were developed new Global Optimality Conditions which possess the classical constructive property the effectiveness of that is demonstrated by examples. 

\keywords{state-convex and state-DC functionals, nonconvex optimal control problem,  exact penalty, Global Optimality Conditions, constructive property}
\end{abstract}
% at the end of the list, there should be no final dot


%\section{Statement of the problem}
Here we address the nonconvex Optimal Control (OC) Problem as follows
\begin{equation}\label{eq:1}
   \lefteqn{\hspace{-3.20cm}({\cal P})}
%\left.
%\begin{array}{c}
 J_0 (x(\cdot),u(\cdot)) \downarrow \min\limits_u, \;\;\; 
 J_i (x(\cdot),u(\cdot)) \leq 0, \;\; i\in I:=\{1,...m\}.
%\end{array}
%\right\}
\end{equation}
along the next state-linear Control System (StLCS)
\begin{equation}\label{eq:2}
\left.
  \begin{array}{c}
    \dot{x}(t) = A(t)x(t) + B(u(t),t) \;\;\; \mathop{\forall}\limits^{\circ} t \in T:= ]t_0,t_1[, \\
    x(t_0) = x_0,\;\; -\infty <t_0<t_1< +\infty;
  \end{array}
\right \}
\end{equation}
\begin{equation}\label{eq:3}
  u(\cdot) \in {\cal U} := \{u(\cdot) \in L^r_{\infty}(T) \mid u(t) \in U \;\; \mathop{\forall}\limits^{\circ} t \in T \};
\end{equation}
 under standard assumptions \cite{3ChernAnanResh2008,5KurzhVar2014}.
Here the sign $\mathop{\forall}\limits^{\circ}$ denotes "almost everywhere"$\;$in the sense of the Lebesque measure, and we consider the absolutely continuous solution 
$
x(\cdot)=x(\cdot,u) \in AC_n (T)
$
to StLCS~(\ref{eq:2}). Besides, the functionals $J_i (\cdot),\;\;i \in \{0\} \cup I$, have the Bolza form
\begin{equation}\label{eq:4}
J_i(u): = J_i (x(\cdot),u(\cdot)):=\varphi_{1i}(x(t_1)) + \int\limits_T \varphi_i (x(t),u(t),t) dt,
\end{equation}
where the functions  $\varphi_{1i}: I\!\!R^n \rightarrow I\!\!R$ are DC, i.e.
%\begin{equation}\label{eq:5}
$ \varphi_{1i}(x):=g_{1i}(x)-h_{1i}(x), \; x \in \Omega_1 \subset I\!\!R^n,$
%\end{equation}
where the reachable set ${\cal R}(t_1)$ belongs to $\Omega_1$, and the functions $g_{1i}(\cdot)$, $h_{1i}(\cdot)$ are convex on $\Omega_1$.


In addition, the functions $\varphi_i(x,u,t)$ are also state-DC, i.e. have the next DC-decomposition\linebreak
%\begin{equation}\label{eq:6}
$ \varphi_{i}(x,u,t):=g_{i}(x,u,t)-h_{i}(x,t),\;\; i \in \{0\} \cup I,$
%\end{equation}
$\forall x \in \Omega(t), \; \forall(u,t) \in U \times T$, ${\cal R}(t) \subset \Omega(t), t \in T$, 
where  the functions $g_i(\cdot)$ and  $h_i(\cdot)$ are state-convex.

Finally, we assume the data from above is state-differentiable. Clearly, the feasible region of Problem $({\cal P})$--(\ref{eq:1})--(\ref{eq:3})
and Problem $({\cal P})$ itself turn out to be nonconvex, i.e. it may possess a large number of locally optimal, stationary and critical (say, in the sense of the Pontryaguins Principle (PMP)) controls, which may be rather far from the set of globally optimal controls.

To overcome these drawbacks, we apply the popular tools of the Exact Penalization Theory  %\cite{16Eremin1966,17Zangwill1967} 
introducing the penalty functional
%\begin{equation}\label{eq:7}
$\mbox{\Large $\pi$}(u):=\mbox{\Large $\pi$}(x(\cdot),u(\cdot)):=\max\{0,J_1(u),...,J_m(u)\},$
%\end{equation}
 the penalized cost functional
%\begin{equation}\label{eq:8}
$J_\sigma (u) := J_0 (u)+\sigma\mbox{\Large $\pi$}(u),$
%\end{equation}
and addressing the auxiliary (penalized) problem
\begin{equation}\label{eq:9}
   \lefteqn{\hspace{-2.85cm}({\cal P}_\sigma)}
%\left.
%\begin{array}{c}
J_\sigma (u)  \downarrow \min\limits_u, \;\;\; u(\cdot) \in {\cal U},
%\end{array}
%\right\}
\end{equation}
along the control system (\ref{eq:1})--(\ref{eq:2}). As well-known, if there exists the so-called threshold value $\sigma_*$ of the penalty parameter, when %\cite{12JBHULemar1993,17Zangwill1967,18Zaslav2013,19Burke1991,20Dolg2016,21DolgFom2019}
\begin{equation}\label{eq:10}
{\cal V}({\cal P})={\cal V}({\cal P}_\sigma),\;\; {\rm Sol}({\cal P})={\rm Sol}({\cal P}_\sigma)\;\;\forall \sigma > \sigma_*,
\end{equation}
then  Problems $({\cal P})$ and $({\cal P}_\sigma)$ turns out to be equivalent. Furthermore, we carry out the DC-decomposition of the goal  functional $J_\sigma(u)$ and get
  \begin{equation}\label{eq:11}
%\begin{array}{c}
	J_\sigma (u):=G_\sigma (x(\cdot),u(\cdot))-F_\sigma (x(\cdot)),
%\end{array}
\end{equation}
that the functionals $G_\sigma(x,u)$ and $F_\sigma(x)$ are state-convex.

Employing the
DC structure of the penalized Problem $({\cal P}_\sigma)$--(\ref{eq:9}), we develop the new Global Optimality Conditions (GOCs) for Problems $({\cal P}_\sigma)$ and $({\cal P})$, which have the following properties.

First,they reduce the solution of the nonconvex OC Problem $({\cal P})$--(\ref{eq:1})--(\ref{eq:3}) to a solving  of Problem $({\cal P}_\sigma)$, that, in turn, is also reduced to the investigation of the family of the (partially)
linearized and state-convex problems of the form \cite{24Strek2013,25Strek2021,26Strek2019}
\begin{equation}%\label{eq:12}
\lefteqn{\hspace{-1.2cm}
	({\cal P}_\sigma L (y))\colon}
	\left.
	\begin{array}{c}
		\quad\;\Phi_{\sigma y} (x(\cdot,u),u(\cdot))\!:=\!G_\sigma(x(\cdot,u),u(\cdot))\!-\!\langle \langle \nabla F_\sigma (y(\cdot)), x(\cdot,u) \rangle \rangle\!\downarrow \min\limits_u(\cdot),  u(\cdot) \in {\cal U},
	\end{array}\right.%\}
\end{equation}
depending on the pair $(y(\cdot),\beta)\in AC_n(T)\times I\!\! R$, which fulfills the following equation
$$% \begin{equation*}%\label{eq:13}
	F_\sigma (y(\cdot))=\beta-\zeta,\;\;\zeta:=J_0(w),
$$%\end{equation}
where the control $w(\cdot)\in {\cal U}$ (feasible in Problem $({\cal P})$) and the corresponding state\linebreak  $z(t)=x(t,w),\;t\in T$ are under scrutiny.


Furthermore, the GOCs possess the classical constructive (algorithmic) property: once the GOC are violated, then one can find a feasible control $\overline{u}(\cdot)\in {\cal U}$, which is better (in the original  Problem $({\cal P})$--(\ref{eq:1})--(\ref{eq:3})) than this one in question \cite{24Strek2013,25Strek2021,26Strek2019}.

Examples demonstrate  the effectiveness of the constructive property of the GOCs.






% At the end of the text, acknowledgments are expressed, if you haven't
% made a footnote from the title. For example, we can write
%The research is carried on with support of RFBR (RNF, other funds), project No.~00-00-00000.

\begin{thebibliography}{9} % or {99}, if there is more than ten references.

\bibitem{3ChernAnanResh2008}
	Chernousko F.L., Ananievski I.M., Reshmin S.A. Control of nonlinear dynamical systems: methods and applications. Springer, Berlin, 2008.
	
\bibitem{5KurzhVar2014}
	Kurzhanski A.B., Varaiya, P. Dynamics and control of trajectory tubes: theory and computation. Birkhauser, Boston, 2014.
	
\bibitem{24Strek2013}
    Strekalovsky A.S. Global optimality conditions for optimal control problems with functions of A.D. Alexandrov. J. Optim. Theory Appl. 2013. Vol. 159. Pp. 297--321.
    
\bibitem{25Strek2021} Strekalovsky A.S. On Global Optimality Conditions for D.c. Minimization Problems With D.c. Constraints. J. Appl. Numer. Optim. 2021. Vol.~3, No.~1. Pp. 175--196

\bibitem{26Strek2019} Strekalovsky A.S. Global optimality conditions and exact penalization. Optim.	Lett.  2019.~Vol.~13. Pp. 597--615.
    
%\bibitem{DLions1976} Duvaut D., Lions J.L. Inequalities in Mechanics and Phisics. Springer, Berlin, 1976.
%
%\bibitem{Gur1997}  Gurman V.I. The Extension Principle in Optimal Control Problems. 2nd~ed. Fizmatlit, Moscow, 1997.~[In Russian]
%
%\bibitem{Moreau1977} Moreau J.-J. Evolution problem associated with a moving convex set in a Hilbert space. J. Differential Eq.~1977. Vol.~26. Pp.~347--374.
%
%\bibitem{BrKr2013}  Brokate M., Krej\u{c}\'{\i} P. Optimal control of ODE systems involving a rate independent variational inequality. Disc. Cont. Dyn. Syst. Ser.~B. 2013. Vol.~18, no~2. Pp.~331--348.
%
%\bibitem{Karpinski2014} Kapinski J., Deshmukh J, Sankaranarayanan S., Arechiga N. Simulation-guided Lyapunov analysis for hybrid dynamical systems. In Proceedings of the 17th International Conference on Hybrid Systems: Computation and Control (HSCC 2014), Berlin, Germany, 2014. Pp.~133--142.
%
%\bibitem{Forsman1991} Forsman K. Construction of Lyapunov functions using Grobner bases. In Proceedings of the 30th IEEE Conference on Decision and Control, Brighton, UK, 1991. Vol.~1. Pp.~798--799.

\end{thebibliography}
%\end{document}

%%% Local Variables:
%%% mode: latex
%%% TeX-master: t
%%% End:
