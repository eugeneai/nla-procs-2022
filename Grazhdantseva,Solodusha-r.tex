\begin{englishtitle}
\title{On an Analytical Solution of a Nonlinear Partial Differential Equation\thanks{Работа выполнена в рамках государственного задания (проект FWEU-2021-0006, тема №~AAAA-A21-121012090034-3) программы фундаментальных исследований РФ на 2021-2030 гг.}}
\author{E.Yu. Grazhdantseva\inst{1},
	S.V. Solodusha\inst{2}}
\institute{Irkutsk State  University, Institute of Mathematics and Information Technologies, Irkutsk, Russia; Melentiev Energy Systems Institute Siberian Branch of the Russian Academy of Sciences, Irkutsk, Russia\\
	\email{grelyur@mail.ru}
	\and
	Melentiev Energy Systems Institute Siberian Branch of the Russian Academy of Sciences, Irkutsk, 664033 Russia\\
	\email{solodusha@isem.irk.ru}}
% etc

\maketitle

\begin{abstract}
The self-similar solution written  for a nonlinear differential equation in partial derivatives of the first order.

\keywords{partial derivative, self-similar solution}
\end{abstract}
\end{englishtitle}

\iffalse
\documentclass[12pt]{llncs}
\usepackage[T2A]{fontenc}
\usepackage[utf8]{inputenc}
\usepackage[english,russian]{babel}
\usepackage[russian]{nla}

%\usepackage[english,russian]{nla}

% \graphicspath{{pics/}} %Set the subfolder with figures (png, pdf).

%\usepackage{showframe}
\begin{document}
%\selectlanguage{russian}
\fi

\title{Об одном аналитическом решении нелинейного дифференциального уравнения в частных производных}
% Первый автор
\author{Е.~Ю.~Гражданцева\inst{1}
	\and
	% Второй автор
	С.~В.~Солодуша\inst{2}
} % обязательное поле
\institute{ИГУ, Институт математики и информационных технологий, Иркутск, Россия;\\
	Институт систем энергетики им. Л.\,А.\,Мелентьева СО РАН, Иркутск, Россия\\
	\email{grelyur@mail.ru}
	\and
	Институт систем энергетики им. Л.\,А.\,Мелентьева, Иркутск, Россия;\\
	ИГУ, Институт математики и информационных технологий, Иркутск, Россия\\
	\email{solodusha@isem.irk.ru}}
% Другие авторы...

\maketitle

\begin{abstract}
Для нелинейного дифференциального уравнения в частных производных первого порядка выписано автомодельное решение.

\keywords{частная производная, автомодельное решение}
\end{abstract}

\section{Основные результаты} % не обязательное поле

Рассматривается нелинейное дифференциальное уравнение в частных производных
$$\alpha \frac{\partial u}{\partial t} + \beta \frac{\partial u}{\partial x} +\gamma u^2 =0,  \eqno(1)$$
где 
$u=u(x,t): D \to R$
- функция независимых переменных
$(x,t),$
$D=\{(x,t): x \in R, t \in R, t>0 \},$
$\alpha, \beta, \gamma $
- действительные числа.

Уравнение (1) имеет автомодельное решение [1] типа
$\displaystyle u(x,t)=\frac{1}{t} W(y),$
( здесь 
$\displaystyle y=\frac{x}{t},$ 
$ W(y)$ 
- решение уравнения 
$(\beta - \alpha y) W' +\gamma W^2 -\alpha W =0,$)
которое имеет вид
$$u(x,t)= \frac{\alpha}{\gamma t - C (\beta t - \alpha x)},\eqno(2) $$
где
$C$ - произвольная постоянная.

Однако, при 
$\gamma =0$
уравнение (1) представляет собой простейшее дифференциальное уравнение в частных производных, общее решение которого определяется формулой [2]
$$u(x,t)=\varphi (\beta t -\alpha x), \eqno(3)$$
где
$\varphi (z)$
является произвольной дифференцируемой функцией.
А при 
$\gamma \ne 0$
уравнение (1) классифицируется как квазилинейное уравнение, которому сопоставляется система обыкновенных дифференциальных уравнений (характеристических уравнений)
$$\displaystyle \frac{dt}{\alpha}=\frac{dx}{\beta}=\frac{du}{\gamma u^2}, \eqno(4)$$
решением которой являются первые интегралы 
$\beta t -\alpha x=C_1$ 
и
$\displaystyle \gamma t +\frac{\alpha}{u} = C_2,$ 
что позволяет выписать решение уравнения (1) в неявной форме (как общий интеграл системы обыкновенных дифференциальных уравнений (4))
$$\Phi\left(\beta t -\alpha x, \gamma t +\frac{\alpha}{u}\right) =0, \eqno(5) $$
где
$\Phi (\xi, \eta)$
- произвольная дифференцируемая функция.
Очевидно, что (3) является следствием (5) при
$\gamma =0.$

Таким образом, полученное автомодельное решение (2)  описывает подкласс точных решений типа  (5) для уравнения (1) .


% Список литературы.
\begin{thebibliography}{99}
\bibitem{1}
% Format for books
Полянин~А.Д., Зайцев~В.Ф., Журов~А.И. Методы решения нелинейных уравнений математической физики и механики. М.:~ФИЗМАТЛИТ,~2005.

\bibitem{2}
Филиппов~А.Ф. Сборник задач по дифференциальным уравнениям: учебное пособие для вузов. М.:~Наука,~1992.
\end{thebibliography}






%\end{document}

%%% Local Variables:
%%% mode: latex
%%% TeX-master: t
%%% End:
