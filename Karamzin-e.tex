\iffalse

%%%%%%%%%%%%%%%%%%%%%%%%%%%%%%%%%%%%%%%%%%%%%%%%%%%%%%%%%%%%%%%%%%%%%%%%
%
% This is the template file for the 6th International conference
% NONLINEAR ANALYSIS AND EXTREMAL PROBLEMS
% June 25-30, 2018
% Irkutsk, Russia
%
%%%%%%%%%%%%%%%%%%%%%%%%%%%%%%%%%%%%%%%%%%%%%%%%%%%%%%%%%%%%%%%%%%%%%%%%
% The preparation of the article is based on the standard llncs class
% (Lecture Notes in Computer Sciences), which is adjusted with style
% file of the conference.
%
% There are two ways of compilation of the file into PDF
% 1. Use pdfLaTeX (pdflatex), (LaTeX+DVIPS will not work);
% 2. Use LuaLaTeX (XeLaTeX will work too).
% When using LuaLaTeX You will need TTF or OTF CMU fonts
% (Computer Modern Unicode). The fonts are installed with 'cm-unicode' package in
% a distribution of LaTeX % (https://www.ctan.org/tex-archive/fonts/cm-unicode),
% either by downloading and installing these fonts system wide, the address of their page is
% http://canopus.iacp.dvo.ru/%7Epanov/cm-unicode/
% The second option won't work in XeLaTeX.
%
% For MiKTeX (LaTeX distribution for Windows),
%  1. Package 'cm-unicode' is installed manually with the MiKTeX administration Console.
%  2. For the compilation of this example, namely, the stub figure, one will also need to
% download package 'pgf' manually. This package uses in the popular
% package tikz.
%  3. Tests showed that the rest of the required packages MiKTeX loads automatically (if
%     it is allowed). The 'auto download' option is
%     configured in 'Settings' section in MiKTeX Console.
%
%
% The easiest way to compile an article is to use pdfLaTeX, but
% the final layout of the book will be compiled with LuaLaTeX,
% as a result will be of better quality thanks to the package 'microtype' and
% use vector OTF instead of standard raster fonts of pdfLaTeX.
%
% In the case of questions and problems with the article compilation,
% write letters to e-mail: eugeneai@irnok.net, Cherkashin Evgeny.
%
% New version of the correcting style file will be available at the website:
%     https://github.com/eugeneai/nla-style
%     file - nla.sty
%
% Further instructions are in the text body of the template. The template itself
% is an article example.
%
% The LaTeX2e format is used!

% 12 points font size is used.
\documentclass[12pt]{llncs}

% The correcting style file is added.
\usepackage{todonotes}

\usepackage{nla} % This package is needed for compiling
                 % this template, it should be removed
                 % from your article.

% Many popular packages (amsXXX, graphicx, etc.) are already imported in the style file.
% If there is a conflict with your packages, try disabling them and compile
% the text.
%
% It would be convenient in the layout of the proceedings if the file names
% of the figures of different authors do not clash.
% To minimize the clash, the drawings can be placed in a separate subfolder
% named after the author or the title of the paper.
%
% \graphicspath{{ivanov-petrov-pics/}} % specifies the folder with images in png, pdf formats.
% or
% \graphicspath{{great-problem-solving-paper-pics/}}.

\begin{document}

% Text should be formatted in accordance with the 'article' class, using extensions like
% AMS.
%
\fi

\title{On the Issue of Normality in State-constrained Optimal Control Problems\thanks{The research is supported by FCT, Portugal.}}
% First author
\author{Dmitry Karamzin\inst{1}
  \and
  Fernando Lobo Pereira\inst{2}
}
\institute{Federal Research Center Computer Science and Control RAS, Moscow, Russia\\
  \email{d.yu.karamzin@gmail.com}
  \and
Engineering Faculty of University of Porto, Portugal\\
\email{flp@fe.up.pt}}
% etc

\maketitle

\begin{abstract}
Normality condition in the maximum principle for state-constrained optimal control problems is investigated. The approach based on the controllability matrix is applied.

\keywords{Maximum principle, normality condition, controllability Gramian}
\end{abstract}

% at the end of the list, there should be no final dot
\section{The main results}

In this presentation, the normality condition in state-constrained optimal control problems is investigated, that is, such a condition which ensures
that $\lambda^0>0$ in the Pontryagin maximum principle, where $\lambda^0$ is the Lagrange multiplier for the minimising functional. For this purpose, a conventional approach is applied which is based on the notion of controllability matrix. The strict positiveness of such a matrix is the key condition for normality of the control process in question.

There exists a vast array of work and literature investigating normality in state-constrained control problems, see, for example, the sources \cite{Rampazzo_Vinter_1999,Frankowska_2009,Bettiol_e_al_2016}. In these sources, the issue of normality has been studied based on the Inward/Outward Pointing Conditions (IPC/OPC). The advantage of this approach is the simplicity and verifiability of normality for a given data. At the same time, a restriction on the endpoints is put forward, as one of them is generally supposed to be free. In the current work, we do not impose any a priori restriction on the endpoint constraints, and in particular, the control problem with fixed endpoints can be examined for normality.

{\small {\bf Acknowledgements.} The authors acknowledge the support of  ARISE AL, Ref. LA/P/0112/2020, SYSTEC-Base, Ref. UIDB/00147/2020, and Programmatic, Ref. UIDP/00147/2020, and projects SNAP, Ref. NORTE-01-0145-FEDER-000085, and MAGIC, Ref. PTDC/EEI-AUT/32485/2017, COMPETE2020-POCI and FCT/MCTES (PIDDAC)}.



\begin{thebibliography}{9}

\bibitem{Rampazzo_Vinter_1999}
Rampazzo  F., Vinter  R.B.  A theorem on existence of neighbouring trajectories satisfying a state constraint, with applications to optimal control. IMA Journal of Mathematical Control and Information. 1999. Vol. 16, no. 4, 335--351.

\bibitem{Frankowska_2009}
Frankowska  H.  Normality of the maximum principle for absolutely continuous solutions to Bolza problems under state constraints. Control and Cybernetics. 2009. Vol. 38, no. 4, 1327--1340.

\bibitem{Bettiol_e_al_2016}
Bettiol  P., Khalil  N., Vinter  R.B.  Normality of Generalized Euler-Lagrange Conditions for State Constrained Optimal Control Problems. Journal of Convex Analysis. 2016.  Vol. 23. Pp. 291--311.

\end{thebibliography}

%\end{document}

%%% Local Variables:
%%% mode: latex
%%% TeX-master: t
%%% End:
