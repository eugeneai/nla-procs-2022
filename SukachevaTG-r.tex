\begin{englishtitle} % Настраивает LaTeX на использование английского языка
% Этот титульный лист верстается аналогично.
\title{Oskolkov Models and Sobolev-type Equations in Magnetohydrodynamics}
% First author
\author{T.G. Sukacheva}
\institute{The Yaroslav-the-Wise Novgorod State University (NovSU), 	Veliky Novgorod, Russia\\
South Ural State University (SUSU NRU), Chelyabinsk, Russia\\
  \email{tamara.sukacheva@novsu.ru}
  }
% etc

\maketitle

\begin{abstract}
Various Oskolkov models are considered that describe the motion of an incompressible viscoelastic Kelvin-Voigt fluid in the Earth's magnetic field. The study is carried out within the framework of the theory of semilinear Sobolev-type  equations based on the concept of a degenerate resolving semigroup of operators. Existence theorems for a unique solution, which is a quasi-stationary semitrajectory, are obtained. Descriptions of the corresponding phase spaces are given.

\keywords{Oskolkov models, Sobolev-type equations, phase space, quasi-stationary semi-trajectories, incompressible viscoelastic Kelvin-Voigt fluid} % в конце списка точка не ставится
\end{abstract}
\end{englishtitle}

\iffalse

%  ооооооооооооооооооооооооооооооооооооооооооооооооооооооооооооооооооооорротоооооооооооооооооооооооооооооооооооооооооооооооооооооооооооооооооооооооооооооооооооооооооооооооооооооооооооооооооооооооооооооооооооооооооооооооооооооооооооооооооооооооооооооооооооооооооооортоооо %%%%%%%%%%%%%%%%%%%%%%%%%%%%%%%%%%%%%%%%%%%%%%%%%%%%%%%%%%%%%%%%%%%%%%%%
%
%  This is the template file for the 6th International conference
%  NONLINEAR ANALYSIS AND EXTREMAL PROBLEMS
%  June 25-30, 2018
%  Irkutsk, Russia
%
%%%%%%%%%%%%%%%%%%%%%%%%%%%%%%%%%%%%%%%%%%%%%%%%%%%%%%%%%%%%%%%%%%%%%%%%

%  Верстка статьи осуществляется на основе стандартного класса llncs
%  (Lecture Notes in Computer Sciences), который корректируется стилевым
%  файлом конференции.
%
%  Скомпилировать файл в PDF можно двумя способами:
%  1. Использовать pdfLaTeX (pdflatex), (LaTeX+DVIPS не работает);
%  2. Использовать LuaLaTeX (XeTeX не будет работать).
%  При использовании LuaLaTeX потребуются TTF- или OTF-шрифты CMU
%  (Computer Modern Unicode). Шрифты устанавливаются либо пакетом
%  дистрибутива LaTeX cm-unicode
%              (https://www.ctan.org/tex-archive/fonts/cm-unicode),
%  либо загрузкой и установкой в операционной системе, адрес страницы:
%  http://canopus.iacp.dvo.ru/%7Epanov/cm-unicode/
%
%  В MiKTeX (дистрибутив LaTeX для ОС Windows):
%  1. Пакет cm-unicode устанавливается вручную в программе MiKTeX Console.
%  2. Для верстки данного примера, а именно, картинки-заглушки необходимо,
%     также вручную, загрузить пакет pgf. Этот пакет используется популярным
%     пакетом tikz.
%  3. Тест показал, что остальные пакеты MiKTeX грузит автоматически (если
%     ему разрешено автоматически грузить пакеты). Режим автозагрузки
%     настраивается в разделе Settings в MiKTeX Console.
%
%
%  Самый простой способ сверстать статью - использовать pdfLaTeX, но
%  окончательный вариант верстки сборника будет собран в LuaLaTeX,
%  так как результат получится лучшего качества.
%
%  В случае возникновения вопросов и проблем с версткой статьи,
%  пишите письма на электронную почту: eugeneai@irnko.net, Черкашин Евгений
%
%  Новые варианты корректирующего стиля будут доступны на сайте:
%        https://github.com/eugeneai/nla-style
%        файл - nla.sty
%
%  Дальнейшие инструкции - в тексте данного шаблона. Он одновременно
%  является примером статьи.
%
%  Формат LaTeX2e!

\documentclass[12pt]{llncs}  % Необходимо использовать шрифт 12 пунктов.

% При использовании pdfLaTeX добавляется стандартный набор русификации babel.
% Если верстка производится в LuaLaTeX, то следующие три строки надо
% закомментировать, русификация будет произведена в корректирующем стиле автоматом.
\usepackage[T2A]{fontenc}
\usepackage[cp1251]{inputenc} % Кодировка utf-8, win1251 (cp1251) не тестировалась.
\usepackage[english,russian]{babel}

% Для верстки в LuaLaTeX текст готовится строго в utf-8!

% В операционной системе Windows для редактирования в кодировки utf-8
% можно использовать программы notepad++ https://notepad-plus-plus.org/,
% techniccenter http://www.texniccenter.org/,
% SciTE (самая маленькая по объему программа) http://www.scintilla.org/SciTEDownload.html
% Подойдет также и встроенный в свежий дистрибутив MiKTeX редактор
% TeXworks.

% Добавляется корректирующий стилевой файл строго после babel, если он был включен.
% В параметре необходимо указать russian, что установит не совсем стандартные названия
% разделов текста, настроит переносы для русского языка как основного и т.п.

%\usepackage{todonotes} % Удрать из вашей статьи, нужен для верстки данного шаблона.

\usepackage[russian]{nla}

% Многие популярные пакеты (amsXXX, graphicx и т.д.) уже импортированы в корректирующий стиль.
% Если возникнут конфликты с вашими пакетами, попробуйте их отключить и сверстать
% текст как есть.
%
%


% Было б удобно при верстке сборника, чтобы названия рисунков разных авторов не пересекались.
% Чтоб минимизировать такое пересечение, рисунки можно поместить в отдельную подпапку с
% названием - фамилией автора или названием статьи.
%
% \graphicspath{{ivanov-petrov-pics/}} % Указание папки с изображениями в форматах png, pdf.
% или
% \graphicspath{{great-problem-solving-paper-pics/}} % Указание папки с изображениями в форматах png, pdf.


\begin{document}

% Текст оформляется в соответствии с классом article, используя дополнения
% AMS.
%
\fi

\title{Модели Осколкова и уравнения соболевского типа в магнитогидродинамике}
% Первый автор
\author{Т.~Г.~Сукачева}  % \inst ставит циферку над автором.



% Аффилиации пишутся в следующей форме, соединяя каждый институт при помощи \and.
\institute{Новгородский государственный университет имени Ярослава Мудрого (НовГУ), Великий Новгород, Российская Федерация\\
Южно-Уральский государственный университет (ЮУрГУ НИУ),
Челябинск, Российская федерация\\
  \email{tamara.sukacheva@novsu.ru}}

\maketitle

\begin{abstract}
Рассматриваются различные модели Осколкова, описывающие движение несжимаемой вязкоупругой жидкости Кельвина-Фойгта в магнитном поле Земли. Исследование проводится в рамках теории полулинейных уравнений соболевского типа на основе понятия вырожденной разрешающей полугруппы операторов. Получены теоремы существования единственного решения, являющегося квазистационарной полутраекторией. Приводятся описания соответствующих фазовых пространств.

\keywords{модели Осколкова, уравнения соболевского типа, фазовое пространство, квазистационарные полутраектории, несжимаемая вязкоупругая жидкость Кельвина--Фойгта} % в конце списка точка не ставится
\end{abstract}


Система уравнений Осколкова
\begin{equation}
\begin{array}{l}
(1-\kappa\nabla^2)v_t=\nu\nabla^2v-(v\cdot\nabla)v+\displaystyle\sum_{l=1}^K \beta_{l}\nabla^{2} w_{l}-\frac{1}{\rho}\nabla p-2\Omega\times v+
\\  \frac{1}{\rho\mu}(\nabla\times b)\times b+f^{1},  \\
\mathop{\mathop\nabla\cdot v=0, \quad \nabla\cdot b=0, \quad b_t=\delta\nabla^2b+\nabla\times(v\times b)+f^{2}.
\quad}\limits^{\ }\\
\mathop{\mathop{\dfrac{\partial w_{l}}{\partial t}=v+\alpha_{l}w_{l}, \quad \alpha_{l}\in {\mathbb{R_-}}, \quad \beta_{l}\in {\mathbb{R_+}}, \quad l=\overline{1,~K},\quad}\limits^{\ }}
\limits^{\ }
\end{array}
\label{u1}
\end{equation}%(1)
моделирует поток несжимаемой вязкоупругой жидкости Кельвина--Фойгта  ненулевого порядка $K$ \cite{OAPN} в магнитном поле Земли.  Здесь вектор-функции $v=(v_1(x,t),
\ldots,v_n(x,t))$ и $b=(b_1(x,t),
\ldots,b_n(x,t))$
характеризуют скорость жидкости и магнитную индукцию соответственно, $p=p(x,t)$~-- давление, $\kappa$~-- коэффициент упругости, $\nu$~-- коэффициент вязкости, $\Omega$~-- угловая
скорость, $\delta$~-- магнитная вязкость, $\mu$~-- магнитная проницаемость, $\rho$~-- плотность, параметры $\beta_{l}, \quad l=\overline{1,~K}$~-- определяют время ретардации (запаздывания) давления. Свободные члены
$f^{1}=(f^{1}_{1},\ldots , f^{1}_{n}),~ f^{1}_{i}=f^{1}_{i}(x,t), f^{2}=f^{2}(x,t)$
отвечают внешнему воздействию на жидкость.

Рассмотрим первую начально-краевую задачу для системы (1)
\begin{equation}
\begin{array}{l}
v(x,0)=v_0(x), \quad b(x,0)=b_0(x),  \quad  w_{l}(x,0)=w_{l0}(x) \quad x\in D,\\
\mathop{\mathop{
v(x,t)=0, \quad b(x,t)=0, \quad  w_{l}(x,t)=0 \quad  (x,t)\in\partial D\times\mathbb{R}_+.
%, \quad l=\overline{1,~K}
\quad}\limits^{\ }}
\end{array}
\label{u2}
\end{equation}%(2)
Здесь $l=\overline{1,~K};$ \quad $D\subset\mathbb{R}^n,$
$n=2,3,$~-- ограниченная область с границей $\partial D$ класса ~$C^\infty.$

Заметим, что задачи такого типа возникают, например, в геофизике \cite{HROP}.
Ранее вырожденные автономные модели магнитогидродинамики
%в автономном случае
изучались в работах \cite{STGF}, \cite{STGFN}. В неавтономном случае исследование проводилось, например, в
%было начато в
%\cite{KAOTGS}  и продолжено в
\cite{KS19}.

Задача (1), (2) исследуется в рамках теории полулинейных уравнений соболевского типа.
%Основным ннструментом исследования служит
Исследование проводится
на основе понятия относительно $p$-секториального оператора и порожденной им разрешающей вырожденной полугруппы операторов %\cite{SGAKOT},
\cite{SGALST}. Доказана теорема существования единственного решения указанной задачи, являющегося квазистационарной полутраекторией, и получено описание ее расширенного фазового пространства.
% Полученная теорема обобщает соответствующие результаты \cite{KAOTGS}.
 Обзор %этих и других
 результатов по исследованию моделей Осколкова в рамках теории уравнений соболевского типа содержится в   \cite{STG22}.
 % Список литературы оформляется подобно ГОСТ-2008.
% Примеры оформления находятся по этому адресу -
%     https://narfu.ru/agtu/www.agtu.ru/fad08f5ab5ca9486942a52596ba6582elit.html
%

\begin{thebibliography}{99} % или {99}, если ссылок больше десяти.
\bibitem{OAPN}
 Осколков~А. П. Начально-краевые задачи для уравнений движения жидкостей Кельвина-Фойгта и Олдройта// Тр. Мат. института им. В. А. Стеклова. 1988. Т. 179. С. 126-164.

\bibitem{HROP}
Hide~R.
On planetary atmospheres and interiors// Mathematical Problems in the Geophisical Sciences, 1, W.H.Raid, ed. Am. Math. Soc., Providence R.I., 1971.

\bibitem{STGF}
Sukacheva~T.G., Kondyukov~A.O.
Phase Space of a Model of Magnetohydrodynamics. 
%Differentsial’nye Uravneniya,
Differential Equations.
2015.  Vol.~51, №~4. Pp.~502-509. %DOI: 10.1134/S0012266115040072

%\bibitem{KSIKAO}
%{\it Kadchenko~S.~I.} Numerical study of a flow of viscoelastic fluid of Kelvin-Voigt having zero order in a magnetic field/  S. I. Kadchenko  A. O %Kondyukov // Journal of Computational and Engineering Mathematics.   2016.   Vol.~3, №~2.   P.~40--47.

\bibitem{STGFN}
Сукачева~Т.Г.,  Кондюков~А.О.
Фазовое пространство модели магнитогидродинамики ненулевого порядка // Дифф. уравнения.  2017.  Т.~53, №~8.  С.~1083--1090.

%\bibitem{KAOTGS}
%Kondyukov~A.O.,  Sukacheva~T. G., Kadchenko~S. I. ,  Ryazanova~L. S.
%Computational experiment for a class of mathematical models of magnetohydrodynamics // Вестник ЮУрГУ. Серия: Математическое моделирование и %программирование. 2017. Т.~10, №~1.   С.~149--155.

\bibitem{KS19}
Kondyukov~A.O., Sukacheva~T.G. A Non-stationary Model of the Incompressible Viscoelastic Kelvin-Voigt Fluid of Non-zero Order in the  Magnetic Field of the Earth // Вестник ЮУрГУ. Серия: Математическое моделирование и программирование. 2019.
  Т.~12,  №~3.  С.~42-51.

%\bibitem{SGAKOT}
%{\it Свиридюк Г.А.}
%К общей теории полугрупп операторов/Г.А. Свиридюк/ УМН.   1994.   Т.~49, № 4.   С.~47~--~74.

\bibitem{SGALST}
Sviridyuk~G.A., Fedorov V.E.
Linear Sobolev Type Equations and Degenerate Semigroups of Operators -- Utrecht. Boston. K\"oln. Tokyo. : VSP, 2003.
% -- 179~p.

\bibitem{STG22}
 Sukacheva~T.G. Oskolkov models and Sobolev-type equations. Bulletin of the South Ural State University. Ser. Mathematical Modelling, Programming
\& Computer Software (Bulletin SUSU MMCS). 2022. Vol. 15, № 1. Pp. 5--22.
%DOI: 10.14529/mmp220101



\end{thebibliography}

% После библиографического списка в русскоязычных статьях необходимо оформить
% англоязычный заголовок.




%\end{document}

%%% Local Variables:
%%% mode: latex
%%% TeX-master: t
%%% End:
