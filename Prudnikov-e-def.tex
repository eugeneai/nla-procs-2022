\iffalse
%%%%%%%%%%%%%%%%%%%%%%%%%%%%%%%%%%%%%%%%%%%%%%%%%%%%%%%%%%%%%%%%%%%%%%%%
%
% This is the template file for the 6th International conference
% NONLINEAR ANALYSIS AND EXTREMAL PROBLEMS
% June 25-30, 2022
% Irkutsk, Russia
%
%%%%%%%%%%%%%%%%%%%%%%%%%%%%%%%%%%%%%%%%%%%%%%%%%%%%%%%%%%%%%%%%%%%%%%%%
% The preparation of the article is based on the standard llncs class
% (Lecture Notes in Computer Sciences), which is adjusted with style
% file of the conference.
%
% There are two ways of compilation of the file into PDF
% 1. Use pdfLaTeX (pdflatex), (LaTeX+DVIPS will not work);
% 2. Use LuaLaTeX (XeLaTeX will work too).
% When using LuaLaTeX You will need TTF or OTF CMU fonts
% (Computer Modern Unicode). The fonts are installed with 'cm-unicode' package in
% a distribution of LaTeX % (https://www.ctan.org/tex-archive/fonts/cm-unicode),
% either by downloading and installing these fonts system wide, the address of their page is
% http://canopus.iacp.dvo.ru/%7Epanov/cm-unicode/
% The second option won't work in XeLaTeX.
%
% For MiKTeX (LaTeX distribution for Windows),
%  1. Package 'cm-unicode' is installed manually with the MiKTeX administration Console.
%  2. For the compilation of this example, namely, the stub figure, one will also need to
% download package 'pgf' manually. This package uses in the popular
% package tikz.
%  3. Tests showed that the rest of the required packages MiKTeX loads automatically (if
%     it is allowed). The 'auto download' option is
%     configured in 'Settings' section in MiKTeX Console.
%
%
% The easiest way to compile an article is to use pdfLaTeX, but
% the final layout of the book will be compiled with LuaLaTeX,
% as a result will be of better quality thanks to the package 'microtype' and
% use vector OTF instead of standard raster fonts of pdfLaTeX.
%
% In the case of questions and problems with the article compilation,
% write letters to e-mail: eugeneai@irnok.net, Cherkashin Evgeny.
%
% New version of the correcting style file will be available at the website:
%     https://github.com/eugeneai/nla-style
%     file - nla.sty
%
% Further instructions are in the text body of the template. The template itself
% is an article example.
%
% The LaTeX2e format is used!

% 12 points font size is used.
\documentclass[12pt]{llncs}

% The correcting style file is added.
\usepackage{todonotes}

\usepackage{nla} % This package is needed for compiling
                 % this template, it should be removed
                 % from your article.



\newcommand{\nid}{\noindent}
\newcommand{\bc}{\begin{center}}
\newcommand{\br}{\begin{right}}
\newcommand{\ec}{\end{center}}
\newcommand{\be}{\begin{equation}}
\newcommand{\ee}{\end{equation}}
\newcommand{\vhi}{\varphi}
\newcommand{\infi}{\infty}
\newcommand{\pr}{\parallel}
\newcommand{\vl}{\mid}
\newcommand{\llra}{\; \longleftrightarrow \;}
\newcommand{\lar}{\leftarrow}
\newcommand{\rar}{\rightarrow}
\newcommand{\grad}{\nabla}
\newcommand{\lapl}{\triangle}
\newcommand{\lrar}{\longrightarrow}
\newcommand{\uar}{\uparrow}
\newcommand{\dar}{\downarrow}
\newcommand{\overl}{\overline}
\newcommand{\underl}{\underline}
\newcommand{\p}{\partial}
\newcommand{\rrr}{R^n_+}
\newcommand{\ball}[2]{\B^n_#1 (#2)}
\newcommand{\sphere}[2]{S^{n-1}_#1 (#2)}
\newcommand{\subcl}{\p_{Cl} \, f(x_0)}
\newcommand{\bint}{\mbox{int} \,}
\newcommand{\bargmin}{\mbox{arg min} \,}
\newcommand{\bcone}{\mbox{cone} \,}
\newcommand{\bco}{\mbox{conv} \,}
\newcommand{\bepi}{\mbox{epi} \,}
\newcommand{\bdom}{\mbox{dom} \,}
\newcommand{\blim}{\mbox{lim} \,}
\newcommand{\bsup}{\mbox{sup} \,}
\newcommand{\bmax}{\mbox{max} \,}
\newcommand{\bmin}{\mbox{min} \,}
\newcommand{\binf}{\mbox{inf} \,}
\newcommand{\exs}{\exists}
\newcommand{\all}{\,\, \forall}
\newcommand{\meq}{\geq}
\newcommand{\g}{\gamma}

\newcommand{\la}{\lambda}
\newcommand{\La}{\Lambda}
\newcommand{\al}{\alpha}
\newcommand{\bet}{\beta}
\newcommand{\del}{\delta}
\newcommand{\Del}{\Delta}
\newcommand{\eps}{\varepsilon}
\newcommand{\bx}{\bar{x}}

\newcommand{\nid}{\noindent}
\newcommand{\bc}{\begin{center}}
\newcommand{\br}{\begin{right}}
\newcommand{\ec}{\end{center}}
\newcommand{\be}{\begin{equation}}
\newcommand{\ee}{\end{equation}}
\newcommand{\vhi}{\varphi}
\newcommand{\infi}{\infty}
\newcommand{\pr}{\parallel}
\newcommand{\vl}{\mid}
\newcommand{\llra}{\; \longleftrightarrow \;}
\newcommand{\lar}{\leftarrow}
\newcommand{\rar}{\rightarrow}
\newcommand{\grad}{\nabla}
\newcommand{\lapl}{\triangle}
\newcommand{\lrar}{\longrightarrow}
\newcommand{\uar}{\uparrow}
\newcommand{\dar}{\downarrow}
\newcommand{\overl}{\overline}
\newcommand{\underl}{\underline}
\newcommand{\p}{\partial}
\newcommand{\rrr}{R^n_+}
\newcommand{\ball}[2]{\B^n_#1 (#2)}
\newcommand{\sphere}[2]{S^{n-1}_#1 (#2)}
\newcommand{\subcl}{\p_{Cl} \, f(x_0)}
\newcommand{\bint}{\mbox{int} \,}
\newcommand{\bargmin}{\mbox{arg min} \,}
\newcommand{\bcone}{\mbox{cone} \,}
\newcommand{\bco}{\mbox{conv} \,}
\newcommand{\bepi}{\mbox{epi} \,}
\newcommand{\bdom}{\mbox{dom} \,}
\newcommand{\blim}{\mbox{lim} \,}
\newcommand{\bsup}{\mbox{sup} \,}
\newcommand{\bmax}{\mbox{max} \,}
\newcommand{\bmin}{\mbox{min} \,}
\newcommand{\binf}{\mbox{inf} \,}
\newcommand{\exs}{\exists}
\newcommand{\all}{\,\, \forall}
\newcommand{\meq}{\geq}
\newcommand{\g}{\gamma}

\newcommand{\la}{\lambda}
\newcommand{\La}{\Lambda}
\newcommand{\al}{\alpha}
\newcommand{\bet}{\beta}
\newcommand{\del}{\delta}
\newcommand{\Del}{\Delta}
\newcommand{\eps}{\varepsilon}
\newcommand{\bx}{\bar{x}}

\newtheorem{thm}{Theorem}[section]
\newtheorem{prop}{Proposition}[section]
\newtheorem{cor}{Corollary}[section]
\newtheorem{as}{Assumption}[section]
\newtheorem{lem}{Lemma}[section]

%\theoremstyle{definition}
\newtheorem{defi}{Definition}[section]
\newtheorem{ex}{Example}[section]
%\theoremstyle{remark}
\newtheorem{rem}{Remark}[section]
\newcommand{\avin}[6]{#3^{-1} \; \int^#2_#1 \, \grad #6(r(#4, \tau, #5))d\tau}
\newcommand{\curve}{r(x_0,\cdot,g)}
\newcommand{\incl}{\subset}


\fi




% Many popular packages (amsXXX, graphicx, etc.) are already imported in the style file.
% If there is a conflict with your packages, try disabling them and compile
% the text.
%
% It would be convenient in the layout of the proceedings if the file names
% of the figures of different authors do not clash.
% To minimize the clash, the drawings can be placed in a separate subfolder
% named after the author or the title of the paper.
%
% \graphicspath{{ivanov-petrov-pics/}} % specifies the folder with images in png, pdf formats.
% or
% \graphicspath{{great-problem-solving-paper-pics/}}.

%\begin{document}

% Text should be formatted in accordance with the 'article' class, using extensions like
% AMS.
%
\title{Constructions of the Subdifferentials and Codifferentials}
% First author
\author{Igor Prudnikov 
}
\institute{Scientific Center of Smolensk State Medical University, Smolensk, Russia\\
  \email{pim\underline{ }10@hotmail.com}
 }
% etc

\maketitle

\begin{abstract}
%Insert your english abstract here. Include 3-6 keywords below.

\keywords{positively homogeneous functions,
quasidifferentiable functions, Generalized Gradients,
codifferentiable functions, subdifferential of the first and
second order, Clarke subdifferential, second
codifferential, generalized matrices of second derivatives}
\end{abstract}

% at the end of the list, there should be no final dot
\section{The main results}


The author studies the  codifferentiable  and twice codifferentiable
functions introduced by Professor V.F. Demyanov, and how to calculate their
subdifferentials and codifferentials. The first and second subdifferentials,
introduced by the author,
are used to compute the first and second codifferential at a point. One
condition is given when a function is continuously codifferentiable.
The proved theorems give the rules for calculation the subdifferentials
and continuous codifferentials.

By definition, a function $ f (\cdot): \mathbb{R}^ n \rightarrow
\mathbb{R} $  is called quasidifferentiable at $x$ if the
decomposition \begin{equation} f (x + \Delta) = f (x) + \max_ {v \in
\underline{\partial} f (x)} (v, \Delta) + \min_ {w \in \overline{\partial} f
(x) } (w, \Delta) + o_x (\Delta) \label{quasidiff6} \end{equation} is correct
where the sets $\underline{\partial} f (x)$ and $ \overline{\partial} f (x) $
are called {\em the subdifferential} and {\em the
superdifferential} correspondingly \cite{demvas1}. The pair
$[\underline{\partial} f (x), \overline{\partial} f (x)]$ is called the
quasidifferential of $f(\cdot)$ at $x$.

We will call the function $f(\cdot)$ twice quasidifferentiable at
$x$ if the following decomposition
$$
f (x + \Delta) = f (x) + \max_ {[v, A] \in \underline {\partial} ^ 2 f
(x)} [(v, \Delta) + \frac {1} {2} (A \Delta, \Delta)]
$$
\begin{equation} + \min_ {[w, B] \in \overline {\partial} ^ 2 f (x)} [(w, \Delta) +
\frac {1} {2} (B \Delta, \Delta)] + o_x (\Delta ^ 2),
\label{quasidiff9b} \end{equation} is correct, where $ \Delta ^ 2 = \| \Delta
\| ^ 2 $, $ o_x (\Delta ^ 2) \rightarrow 0 $, $ \frac {o_x (\alpha ^ 2
\Delta^ 2)} {\alpha ^ 2} \rightarrow 0 $ as $ \Delta \rightarrow 0 $ and $ \alpha \rightarrow
+0 $.

We will call the sets $ \underline {\partial} ^ 2 f (x) $ and $\overline
{\partial} ^ 2 f (x) $ {\em the second subdifferential} and {\em the
second superdifferential} of the function $ f (\cdot) $ at the
point $ x $ correspondingly, which are the convex compact sets. We
will call the pair $[\underline {\partial} ^ 2 f (x), \overline {\partial} ^ 2
f (x)]$ the second quasidifferential of $f(\cdot)$ at $x$.


Demyanov V.F. and Rubinov A.M. have introduced \cite{demrub}
codifferentiable and twice codifferentiable functions. They called
$ f (\cdot): \mathbb{R}^ n \rightarrow \mathbb{R} $ {\em
codifferentiable} at a point $ x $ if there exist convex compact
sets $ \underline {d} f (x), \overline {d} f (x) $ from $
\mathbb{R}^ {n + 1} $, called the {\em hypodifferential} and {\em
hyperdifferential} respectively, for which the decomposition  \begin{equation}
f (x + \Delta) = f (x) + \max_ {[a, v] \in \underline {d} f (x)}
[a + (v, \Delta)] + \min_ {[b, w] \in \overline {d} f (x)} [b +
(w, \Delta)] + o_x (\Delta), \label {quasidiff1} \end{equation} is true and
{\em twice codifferentiable} at a point $ x $ if there exist
convex compacts $ \underline {d} ^ 2 f (x), \overline {d} ^ 2 f
(x) $ of $ \mathbb {R} ^ 1 \times \mathbb {R} ^ {n} \times \mathbb
{R} ^ {n \times n} $, called  {\em the second hypodifferential}
and {\em second hyperdifferential} respectively, for which the
representation $$ f (x + \Delta) = f (x) + \max_ {[a, v, A] \in
\underline {d} ^ 2 f (x)} [a + (v, \Delta) + \frac {1} {2} (A
\Delta, \Delta)] + $$ \begin{equation} + \min_ {[b, w, B] \in \overline {d} ^ 2
f (x)} [b + (w, \Delta)) + \frac {1} {2} (B \Delta, \Delta)] + o_x
(\Delta ^ 2), \label{quasidiff2} \end{equation} is true where $ \Delta ^ 2 =
\| \Delta \| ^ 2 $, $ o_x (\Delta) \rightarrow 0 $, $ o_x (\Delta ^ 2)
\rightarrow 0 $ with $ \Delta \rightarrow 0 $, $ \frac {o_x (\alpha \Delta)} {\alpha}
\rightarrow 0 $ and $ \frac {o_x (\alpha ^ 2 \Delta ^ 2)} {\alpha ^ 2} \rightarrow 0 $
with $ \alpha \rightarrow +0 $.

The pairs of sets $ D  f (x) = [\underline {d} f (x), \overline
{d} f (x)] $ and $ D ^ 2 f (x) = [\underline {d} ^ 2 f (x),
\overline {d} ^ 2 f (x)] $ according to the Demyanov's terminology
are called {\em the first} and {\em the second codifferentials} of
$ f $ at $ x $.  We suppose that the function $ o_x (\Delta^2)$ in
(\ref{quasidiff2}) is uniformly infinitesimal with respect to $
\Delta^2  $ for small $ \Delta.  $

The author's problem is to define and construct the continuous
first and second codifferential of  $ f (\cdot) $ at $ x $, using
subdifferentials of the first and second orders, introduced by the
author in \cite{pimfirstsecondsubdiff}. Let us denote the author's
subdifferential of the first order of $f(\cdot)$ at $x$ by $Df(x)$.


\begin{theorem} Any quasidifferentiable Lipschitz function $ f (\cdot):
\mathbb {R} ^ n \rightarrow \mathbb {R} $, the SVMs $\underline {\partial} f
(\cdot)$, $\overline {\partial} f (\cdot)$ of which  are upper
semi-continuous at $ x $ and, consequently, the SVM $ D f (\cdot)
$ is upper semicontinuous at $x$, is continuously codifferentiable
at $ x $. \label{homogthm5} \end{theorem}

\begin{theorem}
Any twice quasidifferentiable Lipschitz function $ f : \mathbb {R}
^ n \rightarrow \mathbb {R} $, the second quasidifferential $
[\underline{\partial}^2 f (\cdot), \overline {\partial}^2 f (\cdot)]$ of which
is upper semicontinuous at $ x $, is twice continuously
codifferentiable at $ x $.
\end{theorem}

Let us give an example of a Lipschitz quasidifferentiable function
$ f (\cdot) $ for which $ Df (\cdot) $ is not upper semicontinuous
and the function $ f (\cdot) $ is not continuously
codifferentiable.

\begin{example} \hspace {-2mm}. The graph of the function $ f: \mathbb
{R} \rightarrow \mathbb {R} $ consists of the segments,  located between
the curves $ -x ^ 2, + x ^ 2 $, with the slopes $ \pm 1 $. The
function $ f (\cdot) $  is not representable as the difference of
two convex functions in a neighborhood of the point zero, since $
\vee ^ a _0 f '= \infty $ for an arbitrary $ a
> 0. $ It is easy to see that $ \partial _ {Cl} f (0) = [-1, + 1], \,
D f (0) = \{0 \}.$ The SVM $ Df (\cdot) $ is not upper
semi-continuous at zero. The function $ f (\cdot) $ is
quasidifferentiable, but not continuously codifferentiable at
zero. \label{ex1322} \end{example}


% The figures and tables are drawn according to the standard class 'article'.
%\begin{figure}[htb]
%  \centering

% Two picture formats are supported:
%\includegraphics[width=0.7\linewidth]{figure.pdf} % Raster format
%\includegraphics[width=0.7\linewidth]{figure.png} % Vector and raster format
%
% Vector drawings can be drawn in Inkscape editor
% https://inkscape.org/ru/download/
% The usual format of the editor is SVG, so the drawings must be exported in
% PDF or PNG (with a resolution of minimum 150 dpi, and maximum of 300 dpi).
%  \begin{center}
%    \missingfigure[figwidth=0.7\linewidth]{Remove me from the article!} \end{center}
%  \caption{Caption of the figure}\label{fig:example}
%\end{figure}

% At the end of the text, acknowledgments are expressed, if you haven't
% made a footnote from the title. For example, we can write
%The research is carried on with support of RFBR (RNF, other funds), project No.~00-00-00000.

\begin{thebibliography}{8}

\bibitem{demvas1} Demyanov V.F., Vasilyev L.V. Nondifferentiable
optimization.  Nauka, Moscow, 1981. 384 P.

\bibitem{demrub} Demyanov V.F., Rubinov A.M. Foundation of
nonsmooth analysis and quasidifferentiable calculus. M.: Nauka.
1990. 432 p.

\bibitem{lowapp2} Proudnikov I.M. New constructions for local
approximation of Lipschitz functions.I.  Nonlinear analysis. 2003. Vol.
53, no. 3.  Pp. 373--390.
\bibitem{lowapp2a} Proudnikov I.M.  New constructions for local
approximation of Lipschitz functions. II.  Nonlinear Analysis 2007.
 V. 66, no. 7.  Pp. 1443--1453.

\bibitem{pimfirstsecondsubdiff}{Proudnikov I.M.} The Subdifferentials
of the First and Second Orders for Lipschitz Functions. J. of
Optimization Theory and Application. 2016. Vol. 171, no.3. Pp.
906--930.

\bibitem{demyanovdixon} {Demyanov V.F., Dixon L.C.W} (Eds)
Quasidifferential calculus. Mathematical Programming Study.  
Vol. 29.   221 p.
\bibitem{demyanovpallaschke} {Demyanov V.F., Pallaschke D.} (Eds)
Nondifferentiable optimization: Motivations and applications.
Lecture Notes in Economics and mathematical systems.   Berlin:
Springer -- Verlag. 1985.   Vol. 255.   349 p.
\bibitem{shapiro1} {Shapiro A.} Quasidifferential calculus and
first-order optimality conditions in nonsmooth optimization. In
\cite{demyanovdixon}. Pp. 176--202.
\bibitem{shapiro2} {Shapiro A.} On optimality conditions in
quasidifferentiable optimization. SIAM J. Control Optimiz.  
1984.   Vol. 28.   Pp. 61--617.
\end{thebibliography}





%\end{document}

%%% Local Variables:
%%% mode: latex
%%% TeX-master: t
%%% End:
