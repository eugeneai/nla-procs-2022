
\iffalse
\documentclass[12pt]{llncs}
\usepackage{iftex}
\usepackage{nla}
%\usepackage{showframe}
%
%
\begin{document}
%
\fi

\title{Analysis of the Controllability Criteria for Some Degenerate Four-level Quantum Systems\thanks{This work was supported by the Russian Science Foundation under grant № 22-11-00330, https://rscf.ru/en/project/22-11-00330/}}
%
\titlerunning{Analysis of the controllability criteria}  % abbreviated title (for running head)
%                                     also used for the TOC unless
%                                     \toctitle is used
%
\author{Anastasia A. Myachkova\and
Alexander N. Pechen}
%
\authorrunning{Anastasia A. Myachkova and Alexander N. Pechen} % abbreviated author list (for running head)
%
%%%% list of authors for the TOC (use if author list has to be modified)
\tocauthor{Anastasia A. Myachkova and Alexander N. Pechen}
%
\institute{Steklov Mathematical Institute of Russian Academy of Sciences, Moscow, Russia\\
\email{miachkova.aa@phystech.edu}, \email{apechen@gmail.com}}

\maketitle 

\begin{abstract}
In this talk we present our recent results on the analysis of controllability criteria for a class of four-level quantum systems with twice degenerate higher excited energy level and with one forbidden transition. For this purpose, we construct dynamical Lie algebras generated by all commutators of the free Hamiltonians and the interaction Hamiltonians. We show that for two special cases the quantum system is irreducible and controllable, and for the third case the system is reducible and hence uncontrollable. Controllability is proved by constructing a dynamical Lie algebra and showing that it has maximal rank. The reducibility and uncontrollability are proved by constructing a conservative operator. 
\keywords{quantum optimal control theory, controllability of quantum systems, dynamical Lie algebra, reducibility}
\end{abstract}

\section{The problem and the result}
Quantum control attracts high attention due to existing and prospective applications in quantum technologies~\cite{Moore2011,Koch2022}. An important problem in quantum control is to establish, for a given quantum control system, whether the system is controllable or not~\cite{Turinici2001,Albertini2003,BoscainCMP2015,PechenPRA2011}. Of special interest is controllability of systems that arise in the analysis of higher-order traps in quantum control landscapes~\cite{PechenPRL2011}. In this talk we discuss the results of the work~\cite{Myachkova}, which is devoted to study of controllability of special four-level degenerate quantum systems with twice degenerate one of the energy levels and with one forbidden transition. The free and interaction Hamiltonians are
\begin{equation}\label{mmmeq2}
H_0=\mathrm{diag}(a,b,c,c),\qquad
V=\begin{pmatrix}
0& 0 & v_{13} & v_{14} \\
0&  0 & v_{23}  & v_{24} \\
v^\ast_{13}& v^\ast_{23}& 0 &  v_{34} \\
v^\ast_{14}& v^\ast_{24}& v_{34}^\ast & 0 \\
\end{pmatrix},
\end{equation}
where real numbers $a,b$ and $c$ are not equal one to another, and matrix elements satisfy the additional constraint (star denotes complex conjugate)
\begin{equation}\label{Eq:Constraint}
v_{13}v^\ast_{23} +v_{14}v^\ast_{24}=0.
\end{equation}
Here $a$, $b$, $c$ are the energies of the levels, and $v_{ij}$ is the coupling strength between the $|i\rangle$ and $|j\rangle$ levels. 

The main characteristics considered when studying controllability of this quantum system are its connectivity and irreducibility. For degenerate quantum systems irreducibility is more demanding than connectivity and provides a more accurate criterion to characterize un\-cont\-rol\-la\-bi\-li\-ty. 
We investigate all three systems for irreducibility by looking for operators commuting with $H_0$ and $V$, and show that one of the systems is reducible and hence uncontrollability.
For the controllable cases, we numerically construct a basis of 15 matrices generated by all independent commutators of $iH_0$ and $iV$ forming the dynamical Lie algebra ${\mathfrak g}$. We show that $\mathfrak g=\mathfrak{su}(4)$ by decomposing standard generators of $\mathfrak{su}(4)$ in linear combinations of these obtained 15 commutators. This implies that the systems are controllable. Uncontrollability is proven by several different ways: (1) by showing that the system is reducible, (2) by numerically constructing basis of the dynamical Lie algebra which has 7 elements and hence the dynamical Lie algebra in this case has not maximal rank, and (3) by constructing a conserved nontrivial system observable.

\begin{thebibliography}{9}

\bibitem{Moore2011}
Moore K. W. , Pechen A.N., Feng X.-J., Dominy J., Beltrani V.J., Rabitz H.:
Why is chemical synthesis and property optimization easier than expected? Physical Chemistry Chemical Physics. 2011. Vol. 13, no. 21. Pp. 10048-–10070.

\bibitem{Koch2022} 
Koch C.P., Boscain U., Calarco T., Dirr G., Filipp S., Glaser S.J., Kosloff R., Montangero S., Schulte-Herbr\"{u}ggen T.,  Sugny D., Wilhelm F.K.:
Quantum optimal control in quantum technologies. Strategic report on current status, visions and goals for research in Europe
(preprint)

\bibitem{Turinici2001}
Turinici G., Rabitz H.:
Quantum wavefunction controllability
Chem. Phys. 2001. Vol. 267. Pp. 1--9.

\bibitem{Albertini2003}
Albertini F., D'Alessandro D.:
Notions of controllability for bilinear multilevel quantum systems
IEEE Trans. Automat. Control. 2003. Vol. 48, no. 8. Pp. 1399--1403.

\bibitem{BoscainCMP2015}
Boscain U., Gauthier J-P., Rossi F., Sigalotti M.:
Approximate controllability, exact controllability, and conical eigenvalue intersections for quantum mechanical systems. 
Comm. Math. Phys. 2015. Vol. 333. Pp. 1225--1239.

\bibitem{PechenPRA2011}
Pechen A.N.:
Engineering arbitrary pure and mixed quantum states
Phys. Rev. A. 2011. Vol.  84,  no. 4, 042106

\bibitem{PechenPRL2011}
Pechen A.N., Tannor D.J.:
Are there traps in quantum control landscapes?
Phys. Rev. Lett. 2011. Vol. 106, 120402 

\bibitem{Myachkova} Myachkova, A.A., Pechen, A.N.: 
Some controllable and uncontrollable degenerate four-level quantum systems (submitted). 

\end{thebibliography}
%\end{document}