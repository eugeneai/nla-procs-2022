\begin{englishtitle} % Настраивает LaTeX на использование английского языка
% Этот титульный лист верстается аналогично.
\title{Lower Bounds for Area Complexity of Decoder in Model of Cellular Circuits}
% First author
\author{Vadim Zizov 
}
\institute{CMC MSU M.V.Lomonosov, Moscow, Russia\\
  \email{vzs815@gmail.com}
}
% etc

\maketitle

\begin{abstract}
In general the cellular circuit of functional and commutation elements (CC) is a mathematical model of integrated circuits (IC), which takes into account the features of their
physical synthesis. The fundamental difference of this model from the wellstudied classes of boolean circuit (BC) is the presence of additional requirements to the geometry of the circuit, which take into account necessary routing resources when creating an IC. The subject of many
authors study became the complexity of implementation of decoder. Lower bounds for the area complexity of boolean circuits implementing decoder with repeating inputs are
shown in this paper.

\keywords{cellular circuit, boolean circuit, decoder, planar schemes, lower bounds}
\end{abstract}
\end{englishtitle}

\iffalse
%%%%%%%%%%%%%%%%%%%%%%%%%%%%%%%%%%%%%%%%%%%%%%%%%%%%%%%%%%%%%%%%%%%%%%%%
%
%  This is the template file for the 6th International conference
%  NONLINEAR ANALYSIS AND EXTREMAL PROBLEMS
%  June 25-30, 2018
%  Irkutsk, Russia
%
%%%%%%%%%%%%%%%%%%%%%%%%%%%%%%%%%%%%%%%%%%%%%%%%%%%%%%%%%%%%%%%%%%%%%%%%

%  Верстка статьи осуществляется на основе стандартного класса llncs
%  (Lecture Notes in Computer Sciences), который корректируется стилевым
%  файлом конференции.
%
%  Скомпилировать файл в PDF можно двумя способами:
%  1. Использовать pdfLaTeX (pdflatex), (LaTeX+DVIPS не работает);
%  2. Использовать LuaLaTeX (XeLaTeX будет работать тоже).
%  При использовании LuaLaTeX потребуются TTF- или OTF-шрифты CMU
%  (Computer Modern Unicode). Шрифты устанавливаются либо пакетом
%  дистрибутива LaTeX cm-unicode
%              (https://www.ctan.org/tex-archive/fonts/cm-unicode),
%  либо загрузкой и установкой в операционной системе, адрес страницы:
%              http://canopus.iacp.dvo.ru/%7Epanov/cm-unicode/
%  Второй вариант не будет работать в XeLaTeX.
%
%  В MiKTeX (дистрибутив LaTeX для ОС Windows):
%  1. Пакет cm-unicode устанавливается вручную в программе MiKTeX Console.
%  2. Для верстки данного примера, а именно, картинки-заглушки необходимо,
%     также вручную, загрузить пакет pgf. Этот пакет используется популярным
%     пакетом tikz.
%  3. Тест показал, что остальные пакеты MiKTeX грузит автоматически (если
%     ему разрешено автоматически грузить пакеты). Режим автозагрузки
%     настраивается в разделе Settings в MiKTeX Console.
%
%
%  Самый простой способ сверстать статью - использовать pdfLaTeX, но
%  окончательный вариант верстки сборника будет собран в LuaLaTeX,
%  так как результат получится лучшего качества, благодаря пакету microtype и
%  использованию векторных шрифтов OTF вместо растровых pdfLaTeX.
%
%  В случае возникновения вопросов и проблем с версткой статьи,
%  пишите письма на электронную почту: eugeneai@irnok.net, Черкашин Евгений.
%
%  Новые варианты корректирующего стиля будут доступны на сайте:
%        https://github.com/eugeneai/nla-style
%        файл - nla.sty
%
%  Дальнейшие инструкции - в тексте данного шаблона. Он одновременно
%  является примером статьи.
%
%  Формат LaTeX2e!

\documentclass[12pt]{llncs}  % Необходимо использовать шрифт 12 пунктов.

% При использовании pdfLaTeX добавляется стандартный набор русификации babel.
% Если верстка производится в LuaLaTeX, то следующие три строки надо
% закомментировать, русификация будет произведена в корректирующем стиле автоматом.
\usepackage{iftex}

\ifPDFTeX
\usepackage[T2A]{fontenc}
\usepackage[utf8]{inputenc} % Кодировка utf-8, cp1251 и т.д.
\usepackage[english,russian]{babel}
\fi

% Для верстки в LuaLaTeX текст готовится строго в utf-8!

% В операционной системе Windows для редактирования в кодировке utf-8
% можно использовать программы notepad++ https://notepad-plus-plus.org/,
% techniccenter http://www.texniccenter.org/,
% SciTE (самая маленькая по объему программа) http://www.scintilla.org/SciTEDownload.html
% Подойдет также и встроенный в свежий дистрибутив MiKTeX редактор
% TeXworks.

% Добавляется корректирующий стилевой файл строго после babel, если он был включен.
% В параметре необходимо указать russian, что установит не совсем стандартные названия
% разделов текста, настроит переносы для русского языка как основного и т.п.

\usepackage{todonotes} % Этот пакет нужен для верстки данного шаблона, его
% надо убрать из вашей статьи.
\usepackage{amssymb }

\usepackage[russian]{nla}

% Многие популярные пакеты (amsXXX, graphicx и т.д.) уже импортированы в корректирующий стиль.
% Если возникнут конфликты с вашими пакетами, попробуйте их отключить и сверстать
% текст как есть.
%
%


% Было б удобно при верстке сборника, чтобы названия рисунков разных авторов не пересекались.
% Чтоб минимизировать такое пересечение, рисунки можно поместить в отдельную подпапку с
% названием - фамилией автора или названием статьи.
%
% \graphicspath{{ivanov-petrov-pics/}} % Указание папки с изображениями в форматах png, pdf.
% или
% \graphicspath{{great-problem-solving-paper-pics/}}.


\begin{document}

% Текст оформляется в соответствии с классом article, используя дополнения
% AMS.
%
\fi

\title{Нижние оценки сложности длинных дешифраторов в модели клеточных схем}
% Первый автор
\author{В.~С.~Зизов   % \inst ставит циферку над автором.
}

% Аффилиации пишутся в следующей форме, соединяя каждый институт при помощи \and.
\institute{ВМК МГУ, Москва, Российская Федерация\\
  \email{vzs815@gmail.com}
}

\maketitle

\begin{abstract}
В общем случае клеточная схема (КС) представляет собой математическую модель интегральных схем (ИС), которая учитывает особенности их физического синтеза. Предметом изучения многих авторов стала сложность реализации дешифратора в различных классах схем. В настоящей работе устанавливаются асимптотически некоторые нижние оценки для площади КС, реализующих дешфираторы в модели с повторяющимися входами.
\keywords{плоские схемы, клеточные схемы, схемы из функциональных элементов, дешифратор, нижние оценки}
\end{abstract}

Модель клеточных схем (\textbf{КС}) впервые была предложена в 1967 году С.С. Кравцовым в работе \cite{Kravcov}, в которой для неё был получен порядок функции Шеннона. Функция Шеннона характеризует сложность самой <<сложной>> функции алгебры логики (ФАЛ) от $n$ переменных. Модель \textbf{КС} является математической моделью интегральных схем (ИС), учитывающей особенности физического синтеза. Наличие требований на геометрию схемы, обеспечивающих учёт необходимых трассировочных ресурсов при создании ИС, представляет собой принципиальное отличие от хорошо изученных классов схем из функциональных элементов (СФЭ).
В работе А. Альбрехта \cite{Albrecht} продемонстрировано, что функция Шеннона для \textbf{КС} имеет вид $\sigma2^n, \sigma = const$ при $n \to \infty$, но остается неизвестным точное значение $\sigma$ в настоящий момент.

Аналогичная математическая модель в зарубежных источниках была описана в 1980 году К.Д. Томпсоном в работе \cite{Thompson}. Для исследований, связанных с ИС, модель является основополагающей. Исследование \cite{Chazelle} показало, что модель плоских схем Томпсона является удовлетворительным приближением для ИС.%, по крайней мере для монокристаллических систем.
Более того, она остаётся точным приближением для небольших участков (отдельных компонентов) ИС в случаях, когда модель не может верно отразить все особенности проектируемых систем.

Ранее в работе \cite{Kazan} были показаны асимптотические верхние и нижние оценки для площади схем, реализующих дешифратор порядка $n$ без повторяющихся входов. Более того, данные оценки совпадают в первом члене разложения, имеют вид $n2^{n-1}\bigl(1 + O\Bigl(\frac{1}{n}\Bigr)\bigr)$. В настоящей работе устанавливаются некоторые асимптотические нижние оценки площади схем, реализующих дешифраторы порядка $n$ с повторяющимися входами.

В рамках модели клеточных схем были получены также точные по порядку роста нижние оценки сложности для некоторых специальных функций и систем булевых функций. Так, Н.А. Шкаликова установила \cite{Shkalikova} порядок роста вида $n2^n$ для сложности дешифратора, то есть системы из всех $2^n$ элементарных конъюнкций ранга $n$ от $n$ переменных. В работе Ю. Хромковича, С.А. Ложкина и др. \cite{Lozhkin} получены нижняя и верхняя оценки для одной конкретной булевой функции от $n$ переменных, с порядком роста $n^2$. В работе А.Ю. Яблонской \cite{Yablonskaja} была установлена асимптотика площади $C \frac{2^n}{log n}$ для площади \textbf{КС} с ограниченной высотой и кратными входами, реализующих функции из ненулевого инвариантного класса, где константа $C$ зависит от мощности этого класса.

\paragraph{Определение модели.} Предметом изучения настоящей работы являются клеточные схемы (\textbf{КС}), они же плоские прямоугольные схемы. \textbf{КС} является прямоугольной решёткой на плоскости, состоящей из клеток - единичных квадратов. Каждая клетка представляет собой элемент. Модель подробно рассматривается в работе \cite{Kazan}.

\paragraph{Длинные клеточные схемы} -- это выделенный класс клеточных схем, который характеризуется следующими свойствами. Данными \textbf{КС} реализуется одна существенная ФАЛ либо семейство ФАЛ, выходы которых расположены либо на одной, либо на двух заранее выбранных сторонах, и входы которых могут дублироваться сколь угодно большое число раз.

Определение. Сложностью системы ФАЛ $F$ от $n$ переменных в модели клеточных схем с повторяющимися входами называется минимальная площадь $A$ клеточной схемы $S$, реализующую данную систему ФАЛ $F$, в которой произвольные входы повторяются произвольное число раз, общим числом не превосходя $kn$.

$$  A^k(F(n)) = \min_{k} A(F(n)) = \min_{k} \min_{S: S \text{реал.} F(n)} A(S)$$

Теорема 1. В модели клеточных схем с повторяющимися входами при ограничении $kn = O(\frac{2^n}{n})$, т.е. $\exists C: kn \leqslant C\frac{2^n}{n}$ верно
$$  A^k(Q_n) \geqslant 2^{n}\log_2(n) - O(\frac{2^n \log^2(n)}{n}).$$

Теорема 2. В модели клеточных схем с повторяющимися входами при ограничении $kn = \Omega(2^n \log(n) \log\log(n))$, т.е. $\forall C : C2^n\log(n) \log\log(n) < kn$ верно
$$  A^k(Q_n) \geqslant \Omega(2^n \log(n)\log\log(n)).$$

% Современные издательства требуют использовать кавычки-елочки << >>.

% В конце текста можно выразить благодарности, если этого не было
% сделано в ссылке с заголовка статьи, например,
% Работа выполнена при поддержке РФФИ (РНФ, другие фонды), проект \textnumero~00-00-00000.
%

% Список литературы оформляется подобно ГОСТ-2008.
% Примеры оформления находятся по этому адресу -
%     https://narfu.ru/agtu/www.agtu.ru/fad08f5ab5ca9486942a52596ba6582elit.html
%

\begin{thebibliography}{99} % или {99}, если ссылок больше десяти.
\bibitem{Kravcov} Кравцов~С.С. О реализации функций алгебры логики в одном классе схем из функциональных и коммутационных элементов. // Проблемы кибернетики. М.:~Наука,~1967. Вып.~19. С.~285--292.

\bibitem{Albrecht} Альбрехт~А. О схемах из клеточных элементов // Проблемы кибернетики. М.:~Наука,~1975. Вып.~33. С.~209--214.

\bibitem{Thompson} Thompson~Clark~D. A complexity theory for VLSI (1980)

\bibitem{Chazelle} Chazelle~B., Louis~M. A model of computation for VLSI with related complexity results.   Journal ACM. 32~№3 (July~1985), С.~573--588.

\bibitem{Kazan} Ложкин~С.~А.,~Зизов~В.~С. Уточненные оценки сложности дешифратора в модели клеточных схем из функциональных и коммутационных элементов~// Учен. зап. Казан. ун-та. Сер. Физ.-матем. науки. 2020 Т.~162~№3. С.~322--334. DOI: 10.26907/2541-7746.2020.3.

\bibitem{Shkalikova} Шкаликова~Н.А. О реализации булевых функций схемами из клеточных элементов // Математические вопросы кибернетики. Вып. 2. М.: Наука, 1989. С. 177--197.

\bibitem{Lozhkin} Hromkovich~Yu., Lozhkin~S., Rybko~A., Sapozhenko~A., Shkalikova~N. Lower bounds on the area complexity of Boolean circuits. Theoretical Computer Science. 97 (1992) С.~285--300.

\bibitem{Yablonskaja} Яблонская~А.Ю. О сложности реализации булевых функций из инвариантных классов клеточными схемами ограниченной высоты с кратными входами // Вестник ННГУ. 2012. Т. 1б № 4. С. 225--231.

\end{thebibliography}

% После библиографического списка в русскоязычных статьях необходимо оформить
% англоязычный заголовок.




%\end{document}

%%% Local Variables:
%%% mode: latex
%%% TeX-master: t
%%% End:
