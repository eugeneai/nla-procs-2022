
\iffalse
%%%%%%%%%%%%%%%%%%%%%%%%%%%%%%%%%%%%%%%%%%%%%%%%%%%%%%%%%%%%%%%%%%%%%%%%
%
% This is the template file for the 6th International conference
% NONLINEAR ANALYSIS AND EXTREMAL PROBLEMS
% June 25-30, 2018
% Irkutsk, Russia
%
%%%%%%%%%%%%%%%%%%%%%%%%%%%%%%%%%%%%%%%%%%%%%%%%%%%%%%%%%%%%%%%%%%%%%%%%
% The preparation of the article is based on the standard llncs class
% (Lecture Notes in Computer Sciences), which is adjusted with style
% file of the conference.
%
% There are two ways of compilation of the file into PDF
% 1. Use pdfLaTeX (pdflatex), (LaTeX+DVIPS will not work);
% 2. Use LuaLaTeX (XeLaTeX will work too).
% When using LuaLaTeX You will need TTF or OTF CMU fonts
% (Computer Modern Unicode). The fonts are installed with 'cm-unicode' package in
% a distribution of LaTeX % (https://www.ctan.org/tex-archive/fonts/cm-unicode),
% either by downloading and installing these fonts system wide, the address of their page is
% http://canopus.iacp.dvo.ru/%7Epanov/cm-unicode/
% The second option won't work in XeLaTeX.
%
% For MiKTeX (LaTeX distribution for Windows),
%  1. Package 'cm-unicode' is installed manually with the MiKTeX administration Console.
%  2. For the compilation of this example, namely, the stub figure, one will also need to
% download package 'pgf' manually. This package uses in the popular
% package tikz.
%  3. Tests showed that the rest of the required packages MiKTeX loads automatically (if
%     it is allowed). The 'auto download' option is
%     configured in 'Settings' section in MiKTeX Console.
%
%
% The easiest way to compile an article is to use pdfLaTeX, but
% the final layout of the book will be compiled with LuaLaTeX,
% as a result will be of better quality thanks to the package 'microtype' and
% use vector OTF instead of standard raster fonts of pdfLaTeX.
%
% In the case of questions and problems with the article compilation,
% write letters to e-mail: eugeneai@irnok.net, Cherkashin Evgeny.
%
% New version of the correcting style file will be available at the website:
%     https://github.com/eugeneai/nla-style
%     file - nla.sty
%
% Further instructions are in the text body of the template. The template itself
% is an article example.
%
% The LaTeX2e format is used!

% 12 points font size is used.
\documentclass[12pt]{llncs}

% The correcting style file is added.
\usepackage{todonotes}

\usepackage{nla} % This package is needed for compiling
                 % this template, it should be removed
                 % from your article.

% Many popular packages (amsXXX, graphicx, etc.) are already imported in the style file.
% If there is a conflict with your packages, try disabling them and compile
% the text.
%
% It would be convenient in the layout of the proceedings if the file names
% of the figures of different authors do not clash.
% To minimize the clash, the drawings can be placed in a separate subfolder
% named after the author or the title of the paper.
%
% \graphicspath{{ivanov-petrov-pics/}} % specifies the folder with images in png, pdf formats.
% or
% \graphicspath{{great-problem-solving-paper-pics/}}.

\begin{document}

% Text should be formatted in accordance with the 'article' class, using extensions like
% AMS.
%

\fi

\title{Optimal Location of Rigid Inclusions  in Contact Problems for Inhomogeneous Two-dimensional Bodies\thanks{The research is
supported by Ministry of Science and Higher Education of the
Russian Federation within the framework of the base part of the
state task, project No.~FSRG-2020-0006.}}
\author{Nyurgun Lazarev
 % \and
%  Name FamilyName2\inst{2}
%  \and
%  Name FamilyName3\inst{1}
}
\institute{North-Eastern Federal University, Yakutsk, Russia\\
  \email{nyurgun@ngs.ru}
%  \and
%Affiliation, City, Country\\
%\email{email@example.com}
}
% etc

\maketitle

\begin{abstract}
The 2D-model of an elastic body with a finite set of rigid
inclusions is considered. We assume that the body can come in
frictionless contact on a part of its boundary with a rigid
obstacle. On the remaining part of the body's boundary a
homogeneous Dirichlet boundary condition is imposed. For a family
of corresponding variational problems, we analyze the dependence
of their solutions on locations of the rigid inclusions.
Continuous dependency of the solutions on location parameters is
established. The existence of a solution of the optimal control
problem is proven. For this problem, a cost functional is defined
by an arbitrary continuous functional on the solution space, while
the control is given by location parameters of the rigid
inclusions.

\keywords{variational inequality, optimal control problem
non-linear boundary conditions, rigid inclusion, location }
\end{abstract}

% at the end of the list, there should be no final dot
\section{The main results}


In the present study, we deal with an optimal problem for a
nonlinear mathematical model describing contact of an elastic body
with a finite set of volume rigid inclusions. For that problem the
cost functional is an arbitrary continuous functional defined on
the solution's space, while the location parameters of the rigid
inclusions serve as a control. More precisely, we assume that
inclusions can change their locations inside the domain of the
body so that the distance between the inclusion's boundary and the
boundary of the body or another inclusion is greater than or equal
to a given value. We prove the continuous dependence of the
solutions on the inclusion's location parameter
\cite{Laz-Rud2022}. The existence of the solution of the optimal
control problem is proven \cite{Laz-Rud2022}. The novelty of the
obtained results compared to previous investigations related to an
optimal location of a rigid inclusion consists in two
generalizations. The first generalization consists in an arbitrary
finite number of inclusions, since the previous works concern only
one inclusion. The second generalization is that the coordinates
of inclusions can vary in any strictly inner subdomain of the
body's domain, whereas in the previous formulations of
corresponding optimal problems the coordinate of an inclusion
varied only along a given smooth curve. These mentioned
differences lead to a significant generalization of the previous
results, since minimizing sequences of coordinates do not have to
satisfy the conditions for belonging to one smooth curve.

%% The figures and tables are drawn according to the standard class 'article'.
%\begin{figure}[htb]
%  \centering
%
%% Two picture formats are supported:
%%\includegraphics[width=0.7\linewidth]{figure.pdf} % Raster format
%%\includegraphics[width=0.7\linewidth]{figure.png} % Vector and raster format
%%
%% Vector drawings can be drawn in Inkscape editor
%% https://inkscape.org/ru/download/
%% The usual format of the editor is SVG, so the drawings must be exported in
%% PDF or PNG (with a resolution of minimum 150 dpi, and maximum of 300 dpi).
%  \begin{center}
%    \missingfigure[figwidth=0.7\linewidth]{Remove me from the article!} \end{center}
%  \caption{Caption of the figure}\label{fig:example}
%\end{figure}

% At the end of the text, acknowledgments are expressed, if you haven't
% made a footnote from the title. For example, we can write
%The research is carried on with support of RFBR (RNF, other funds), project No.~00-00-00000.

\begin{thebibliography}{9} % or {99}, if there is more than ten references.
\bibitem{Laz-Rud2022} Lazarev N., Rudoy E. Optimal location of a finite set of rigid inclusions in contact problems for inhomogeneous two-dimensional bodies.
J. Comput. Appl. Math. 2022. Vol.~403, Art no.~113710.


\end{thebibliography}
%\end{document}

%%% Local Variables:
%%% mode: latex
%%% TeX-master: t
%%% End:
