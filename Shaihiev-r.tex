\begin{englishtitle} % Настраивает LaTeX на использование английского языка
% Этот титульный лист верстается аналогично.
\title{Fluid Storage Control with a Proportional-integrally Differentiating Solver}
% First author
\author{E.~R.~Shaihiev\inst{1}
  \and
  A.~D.~Nizamova\inst{2}
 }
\institute{USATU, Ufa, Russia\\
  \email{erik08082002@mail.ru}
  \and
Mavlyutov Institute of Mechanics, Ufa Investigation Center, RAS, Russia\\
\email{adeshka@yandex.ru}}
% etc

\maketitle

\begin{abstract}
For water and oil the PID-controller coefficients responsible for the operation of the outlet valve were found, an interface was created for controlling, automating and optimizing a given process in the open software tool "WinCC OA" by SIEMENS.

\keywords{proportional-integrally differentiating solver, proportional-integrally differentiating coefficient, fluid} % в конце списка точка не ставится
\end{abstract}
\end{englishtitle}

\iffalse

%%%%%%%%%%%%%%%%%%%%%%%%%%%%%%%%%%%%%%%%%%%%%%%%%%%%%%%%%%%%%%%%%%%%%%%%
%
%  This is the template file for the 6th International conference
%  NONLINEAR ANALYSIS AND EXTREMAL PROBLEMS
%  June 25-30, 2018
%  Irkutsk, Russia
%
%%%%%%%%%%%%%%%%%%%%%%%%%%%%%%%%%%%%%%%%%%%%%%%%%%%%%%%%%%%%%%%%%%%%%%%%

%  Верстка статьи осуществляется на основе стандартного класса llncs
%  (Lecture Notes in Computer Sciences), который корректируется стилевым
%  файлом конференции.
%
%  Скомпилировать файл в PDF можно двумя способами:
%  1. Использовать pdfLaTeX (pdflatex), (LaTeX+DVIPS не работает);
%  2. Использовать LuaLaTeX (XeLaTeX будет работать тоже).
%  При использовании LuaLaTeX потребуются TTF- или OTF-шрифты CMU
%  (Computer Modern Unicode). Шрифты устанавливаются либо пакетом
%  дистрибутива LaTeX cm-unicode
%              (https://www.ctan.org/tex-archive/fonts/cm-unicode),
%  либо загрузкой и установкой в операционной системе, адрес страницы:
%              http://canopus.iacp.dvo.ru/%7Epanov/cm-unicode/
%  Второй вариант не будет работать в XeLaTeX.
%
%  В MiKTeX (дистрибутив LaTeX для ОС Windows):
%  1. Пакет cm-unicode устанавливается вручную в программе MiKTeX Console.
%  2. Для верстки данного примера, а именно, картинки-заглушки необходимо,
%     также вручную, загрузить пакет pgf. Этот пакет используется популярным
%     пакетом tikz.
%  3. Тест показал, что остальные пакеты MiKTeX грузит автоматически (если
%     ему разрешено автоматически грузить пакеты). Режим автозагрузки
%     настраивается в разделе Settings в MiKTeX Console.
%
%
%  Самый простой способ сверстать статью - использовать pdfLaTeX, но
%  окончательный вариант верстки сборника будет собран в LuaLaTeX,
%  так как результат получится лучшего качества, благодаря пакету microtype и
%  использованию векторных шрифтов OTF вместо растровых pdfLaTeX.
%
%  В случае возникновения вопросов и проблем с версткой статьи,
%  пишите письма на электронную почту: eugeneai@irnok.net, Черкашин Евгений.
%
%  Новые варианты корректирующего стиля будут доступны на сайте:
%        https://github.com/eugeneai/nla-style
%        файл - nla.sty
%
%  Дальнейшие инструкции - в тексте данного шаблона. Он одновременно
%  является примером статьи.
%
%  Формат LaTeX2e!

\documentclass[12pt]{llncs}  % Необходимо использовать шрифт 12 пунктов.

% При использовании pdfLaTeX добавляется стандартный набор русификации babel.
% Если верстка производится в LuaLaTeX, то следующие три строки надо
% закомментировать, русификация будет произведена в корректирующем стиле автоматом.
\usepackage{iftex}

\ifPDFTeX
\usepackage[T2A]{fontenc}
\usepackage[utf8]{inputenc} % Кодировка utf-8, cp1251 и т.д.
\usepackage[english,russian]{babel}
\fi

% Для верстки в LuaLaTeX текст готовится строго в utf-8!

% В операционной системе Windows для редактирования в кодировке utf-8
% можно использовать программы notepad++ https://notepad-plus-plus.org/,
% techniccenter http://www.texniccenter.org/,
% SciTE (самая маленькая по объему программа) http://www.scintilla.org/SciTEDownload.html
% Подойдет также и встроенный в свежий дистрибутив MiKTeX редактор
% TeXworks.

% Добавляется корректирующий стилевой файл строго после babel, если он был включен.
% В параметре необходимо указать russian, что установит не совсем стандартные названия
% разделов текста, настроит переносы для русского языка как основного и т.п.

\usepackage{todonotes} % Этот пакет нужен для верстки данного шаблона, его
                       % надо убрать из вашей статьи.

\usepackage[russian]{nla}

% Многие популярные пакеты (amsXXX, graphicx и т.д.) уже импортированы в корректирующий стиль.
% Если возникнут конфликты с вашими пакетами, попробуйте их отключить и сверстать
% текст как есть.
%
%


% Было б удобно при верстке сборника, чтобы названия рисунков разных авторов не пересекались.
% Чтоб минимизировать такое пересечение, рисунки можно поместить в отдельную подпапку с
% названием - фамилией автора или названием статьи.
%
% \graphicspath{{ivanov-petrov-pics/}} % Указание папки с изображениями в форматах png, pdf.
% или
% \graphicspath{{great-problem-solving-paper-pics/}}.


\begin{document}

% Текст оформляется в соответствии с классом article, используя дополнения
% AMS.
%
\fi

\title{Контроль хранилища жидкости при помощи ПИД-регулятора}
% Первый автор
\author{Э.~Р.~Шайхиев\inst{1}  % \inst ставит циферку над автором.
  \and  % разделяет авторов, в тексте выглядит как запятая.
% Второй автор
  А.~Д.~Низамова\inst{2}
  \and
} % обязательное поле

% Аффилиации пишутся в следующей форме, соединяя каждый институт при помощи \and.
\institute{УГАТУ, Уфа, Россия \\
  \email{erik08082002@mail.ru}
  \and   % Разделяет институты и присваивает им номера по порядку.
ИМех УФИЦ РАН, Уфа, Россия\\
  \email{adeshka@yandex.ru}
% \and Другие авторы...
}

\maketitle

\begin{abstract}
Для воды и нефти найдены коэффициенты ПИД-регулятора, отвечающие за работу выходного клапана, создан интерфейс для контролирования, автоматизации и оптимизации заданного процесса в открытом программном инструменте "WinCC OA” компании SIEMENS.

\keywords{ПИД-регулятор, ПИД-коэффициент, жидкость} % в конце списка точка не ставится
\end{abstract}

%\section{Основные результаты} % не обязательное поле

Рассматривается гидродинамическая система уравнений для модели вязкой жидкости, в которой определяются параметры установившегося прямолинейного потока несжимаемой жидкости.

Для регулирования выходного потока по мере приближения уровня жидкости к заданному уровню для формирования входного сигнала с точным и качественным переходным процессом в автоматических системах управления используют пропорционально-интегрально-дифференцирующий (ПИД) регулятор.

ПИД-регулятор в управляющем контуре использует обратную связь. ПИД-закон регулирования обеспечивает достаточно высокую точность поддержания уровня жидкости. Уровень оттока жидкости рассчитывается по формуле
$$
u(t)=K_p\cdot e(t)+K_i\int_0^te(\tau)d\tau-R_d\cdot\frac{de(t)}{dt},
$$
где пропорциональный $K_p$, интегральный $K_i$, и дифференциальный $K_d$  ПИД-коэффициенты регулирования.

Слагаемое $K_p\cdot e(t)$ прямо пропорционально ошибке $e(t)$ разности установки и измененного значения уровня. Использовние только пропорциональной части зачастую приводит к ошибке между установкой и фактическим значением.

Слагаемое $K_i\int_0^te(\tau)d\tau$ учитывает прошлые значения ошибки и интегрирует их, то есть при помощи интегральной составляющей устраняет остаточную ошибку, возникающую при использовании одной пропорциональной составляющей.

Слагаемое $-R_d\cdot\frac{de(t)}{dt}$ пропорционально скорости изменения уровня с обратным знаком и должно препятствовать резким скачкам уровня, а именно, упреждает ошибку и эффективно стремится уменьшить ее.

Для достижения хорошего качества регулирования необходимо правильно подобрать коэффициенты к установке. Настройка ПИД-регулятора зачастую осуществляется  методом проб и ошибок.  

Для создания программного интерфейса будем пользоваться программой SIMATIC WinCC Open Architecture от компании SIEMENS, находящуюся в открытом доступе. ПИД-регулирование выполняется с использованием глобального скрипта Model.ctl.

С помощью программы выполняется имитация поведения системы резервуаров, контроллеров и прочего.

\begin{thebibliography}{9} % или {99}, если ссылок больше десяти.

\bibitem{Shaihiev_Sherman} Sherman~F.S. Viscous Flow. McGraw-Hill series in mechanical engineering. McGraw-Hill,~1990.

\bibitem{Shaihiev_Murt}	Муртазина~Р.Д., Кудашева~Е.Г., Низамова~А.Д., Сидельникова~Н.А. Дифференциальные уравнения в частных производных второго порядка. Устойчивость течения жидкостей в канале с линейным профилем температуры. М.:~Русайнс,~2021.


\bibitem{Shaihiev_Niza}	Nizamova~A.D., Murtazina~R.D., Kireev~V.N., Urmancheev~S.F. Features of Laminar-Turbulent Transition for the Coolant Flow in a Plane Heat-Exchanger Channel. Lobachevskii Journal of Mathematics. 2021. Vol.~42. \textnumero~9. Pp.~2211--2215.


\end{thebibliography}

% После библиографического списка в русскоязычных статьях необходимо оформить
% англоязычный заголовок.



%\end{document}

%%% Local Variables:
%%% mode: latex
%%% TeX-master: t
%%% End:
