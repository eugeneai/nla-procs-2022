\begin{englishtitle}
\title{On the Spectrum of One Class of Integral-Functional Operators in Solving  Nonlinear Volterra  Loaded Equations}
% First author
\author{Nikolay Sidorov, 
             Lev Sidorov}
\institute{ISU, Irkutsk, Russia\\
  \email{sidorovisu@gmail.com, lev.ryan.lev@gmail.com}}
% etc

\maketitle

\begin{abstract}
	

\keywords{Integral-functional operator, eigenfunctions, integer analytic functions, Volterra equations}

~\\

The eigenvalues and eigenfunctions are constructed for the following operators:
$$ Ax = \int\limits_0^t K(x,s) x(s)dx + a(t)x_{\alpha},$$
where  $x_{\alpha} = x ( \alpha (t))$  or  $x_{\alpha} = \int\limits_a^bx(t)d\alpha(t)$  is Stieltjes functional  (load), $a(t)$ and  $\alpha(t)$ 
are given functions.


%В ряде простых случаев собственные функции таких операторов легко построить в классе целых аналитических функций.

\begin{example}
	For operators $Ax = \int\limits_0^t x(s)ds + x\left(\displaystyle\frac{t}{2}\right)$  eigenvalue $\lambda = 1$.  Corresponding eigenfunction  $\phi(t)$  for  $ t \in \mathbb{R}$  is constructed as following series
	
	$$\sum_{n=0}^{\infty} a_n\cdot t^n$$
	
	$$a_{0} = 1, a_{n} = \frac{2^{n}}{n(2^{n}-1)}a_{n-1}, n = 1, 2, ...$$
\end{example}


The presented analytical method for constructing eigennumbers and eigenfunctions of the operator $A$ is used in constructing branching solutions of the nonlinear Volterra integral equations considered in [1].

\begin{thebibliography}{99}
	\bibitem{1}
	% Format for Journal Reference
	Sidorov N.A., Sidorov D.N.  {\it Nonlinear Volterra Equations with Loads and Bifurcation Parameters: Existence Theorems and Construction of Solutions.} Diff Equat. 2021.  Vol. 57. 1640--1651.
	% Format for books
	
	% etc
\end{thebibliography}

\end{abstract}
\end{englishtitle}

\iffalse

\documentclass[12pt]{llncs}
\usepackage[T2A]{fontenc}
\usepackage[utf8]{inputenc}
\usepackage[english,russian]{babel}
\usepackage[russian]{nla}

%\usepackage[english,russian]{nla}

% \graphicspath{{pics/}} %Set the subfolder with figures (png, pdf).

%\usepackage{showframe}
\begin{document}
%\selectlanguage{russian}
\fi

\title{О роли спектра одного класса интегрально-функциональных операторов в решении нелинейных уравнений  Вольтерра с нагрузками}
% Первый автор
\author{Н.~А.~Сидоров 
  \and
% Второй автор
  Л.~Д.~Сидоров 
} % обязательное поле
\institute{Иркутский государственный университет (ИГУ), Иркутск, Россия\\
  \email{sidorovisu@gmail.com, lev.ryan.lev@gmail.com}}

\maketitle

\begin{abstract}

\keywords{ Интегрально-функциональный оператор, собственные функции, целые аналитические функции, уравнения Вольтерра }
\end{abstract}

Строятся собственные числа и собственные функции операторов вида:
$$ Ax = \int\limits_0^t K(x,s) x(s)dx + a(t)x_{\alpha},$$
где $x_{\alpha} = x ( \alpha (t))$  или $x_{\alpha} = \int\limits_a^bx(t)d\alpha(t)$ -- функционал Стилтьеса, называемый в приложениях нагрузкой, $a(t)$ и $\alpha(t)$ -- заданные функции.

В ряде простых случаев собственные функции таких операторов легко построить в классе целых аналитических функций.

\begin{example}
	 У оператора $Ax = \int\limits_0^t x(s)ds + x\left(\displaystyle\frac{t}{2}\right)$ собственное число $\lambda = 1$, а соответствующая собственная функция $\phi(t)$ при $ t \in \mathbb{R}$ строится в виде сходящегося ряда:
	 
	 	$$\sum_{n=0}^{\infty} a_n\cdot t^n$$
	 	
    	$$a_{0} = 1, a_{n} = \frac{2^{n}}{n(2^{n}-1)}a_{n-1}, \,\, n = 1, 2, ...$$
\end{example}


Излагаемый аналитический метод построения собственнных чисел и собственных функций оператора $A$ используется при построении разветвляющихся решений нелинейных интегральных уравнений Вольтерра, рассмотренных в работе [1].

%Sidorov, N.A., Sidorov, D.N. Nonlinear Volterra Equations with Loads and Bifurcation Parameters: Existence Theorems and Construction of Solutions. Diff Equat 57, 1640–1651 (2021). https://doi.org/10.1134/S0012266121120107

%Aida-zade, Kamil R., and Vagif M. Abdullayev. "Solution to a class of inverse problems for a system of loaded ordinary differential equations with integral conditions." Journal of Inverse and Ill-posed Problems 24.5 (2016): 543-558.

%Abdullayev, V. M., & Aida-zade, K. R. (2017). Optimization of loading places and load response functions for stationary systems. Computational Mathematics and Mathematical Physics, 57(4), 634-644.

%Kozhanov, A. I. (2004). Nonlinear loaded equations and inverse problems. Zhurnal Vychislitel'noi Matematiki i Matematicheskoi Fiziki, 44(4), 694-716.
% Список литературы.
\begin{thebibliography}{99}
\bibitem{sid1}
% Format for Journal Reference
Sidorov N.A., Sidorov D.N.  {  Nonlinear Volterra Equations with Loads and Bifurcation Parameters: Existence Theorems and Construction of Solutions.} Diff Equat. 2021.  Vol. 57. Pp. 1640--1651.
% Format for books

% etc
\end{thebibliography}






%\end{document}

%%% Local Variables:
%%% mode: latex
%%% TeX-master: t
%%% End:
