\iffalse
%%%%%%%%%%%%%%%%%%%%%%%%%%%%%%%%%%%%%%%%%%%%%%%%%%%%%%%%%%%%%%%%%%%%%%%%
%
% This is the template file for the 6th International conference
% NONLINEAR ANALYSIS AND EXTREMAL PROBLEMS
% June 25-30, 2018
% Irkutsk, Russia
%
%%%%%%%%%%%%%%%%%%%%%%%%%%%%%%%%%%%%%%%%%%%%%%%%%%%%%%%%%%%%%%%%%%%%%%%%
% The preparation of the article is based on the standard llncs class
% (Lecture Notes in Computer Sciences), which is adjusted with style
% file of the conference.
%
% There are two ways of compilation of the file into PDF
% 1. Use pdfLaTeX (pdflatex), (LaTeX+DVIPS will not work);
% 2. Use LuaLaTeX (XeLaTeX will work too).
% When using LuaLaTeX You will need TTF or OTF CMU fonts
% (Computer Modern Unicode). The fonts are installed with 'cm-unicode' package in
% a distribution of LaTeX % (https://www.ctan.org/tex-archive/fonts/cm-unicode),
% either by downloading and installing these fonts system wide, the address of their page is
% http://canopus.iacp.dvo.ru/%7Epanov/cm-unicode/
% The second option won't work in XeLaTeX.
%
% For MiKTeX (LaTeX distribution for Windows),
%  1. Package 'cm-unicode' is installed manually with the MiKTeX administration Console.
%  2. For the compilation of this example, namely, the stub figure, one will also need to
% download package 'pgf' manually. This package uses in the popular
% package tikz.
%  3. Tests showed that the rest of the required packages MiKTeX loads automatically (if
%     it is allowed). The 'auto download' option is
%     configured in 'Settings' section in MiKTeX Console.
%
%
% The easiest way to compile an article is to use pdfLaTeX, but
% the final layout of the book will be compiled with LuaLaTeX,
% as a result will be of better quality thanks to the package 'microtype' and
% use vector OTF instead of standard raster fonts of pdfLaTeX.
%
% In the case of questions and problems with the article compilation,
% write letters to e-mail: eugeneai@irnok.net, Cherkashin Evgeny.
%
% New version of the correcting style file will be available at the website:
%     https://github.com/eugeneai/nla-style
%     file - nla.sty
%
% Further instructions are in the text body of the template. The template itself
% is an article example.
%
% The LaTeX2e format is used!

% 12 points font size is used.
\documentclass[12pt]{llncs}

% The correcting style file is added.
\usepackage{todonotes}

\usepackage{nla} % This package is needed for compiling
                 % this template, it should be removed
                 % from your article.

% Many popular packages (amsXXX, graphicx, etc.) are already imported in the style file.
% If there is a conflict with your packages, try disabling them and compile
% the text.
%
% It would be convenient in the layout of the proceedings if the file names
% of the figures of different authors do not clash.
% To minimize the clash, the drawings can be placed in a separate subfolder
% named after the author or the title of the paper.
%
% \graphicspath{{ivanov-petrov-pics/}} % specifies the folder with images in png, pdf formats.
% or
% \graphicspath{{great-problem-solving-paper-pics/}}.

\begin{document}

% Text should be formatted in accordance with the 'article' class, using extensions like
% AMS.
%
\fi
\title{A Note on Differential-algebraic Equations with Hysteresis Phenomena}
% First author
\author{Pavel Petrenko
}
\institute{IDSCT SB RAS, Irkusk, Russia\\
  \email{petrenko.pavels@gmail.com}}

% etc

\maketitle

\begin{abstract}
In this note, we investigate some class of differential-algebraic equations (DAE) with hysteresis phenomena. We consider non-stationary DAE with non-linearity of hysteresis type (modeled by a sweeping process). For such a DAE, we design an equivalent structural form (in the sense of solutions), and prove a necessary and sufficient condition for the existence and uniqueness of a solution to an initial value problem.

\keywords{differential-algebraic equations, hysteresis, sweeping process}
\end{abstract}

% at the end of the list, there should be no final dot
\section{The main results}

Consider the following system of ordinary differential equations (ODE) paired with a sweeping process \cite{pppKM2000,pppMoreau1977} of a rate independent hysteresis type  (modeled by the play operator \cite{pppBS1996,pppKr1991}):
\begin{align} \label{ppppss1}
 & A(t)\dot{x}(t)=B(t)x(t)+C(t)y(t), \ \ \ x(t_0)=x_0,   
\\[0.5em] & 
\begin{array}{c}
  -\ddot{y}(t)\in \mathcal{N}_{Q(t)}\big(\dot{y}(t)\big), \\[0.5em] 
 {y}(t_0)=y_0, \ \  \  \dot{y}(t_0)=y_1\in Q(t_0).  
  \end{array} \label{ppppss2}
\end{align}
Here, $A(t), B(t), C(t) \in \mathbb R^{(n\times n)}$ are given matrices, wherein $\det A(t) \equiv 0$, $T=[t_0,t_1]$ is a given time interval, $x(t), y(t) \in {\mathbb R}^n$ are unknown functions; $Q(t)=x(t)-Z$ is a ``moving set'', where $Z$ is a given closed convex subset of ${\mathbb R}^n$, and $\mathcal{N}_{Q}\big(q\big)$ denotes the normal cone to a closed convex set $Q$ at a point $q$.

By a solution of (\ref{ppppss1}), (\ref{ppppss2}) we mean a pair of absolutely continuous ($W^{1,1}$) functions $(x, y)$ satisfying (\ref{ppppss1}), (\ref{ppppss2}) for almost all (a.e.) $t \in T$ with respect to (w.r.t.) the usual Lebesgue measure.


Let's consider time-varying ODE system not resolved with respect to the derivative
\begin{equation}\label{pppdau}
A(t)x'(t)=B(t)x(t) + U(t)f(t), \;\;\; t \in T \subset {\mathbb R},
\end{equation}
where $A(t), B(t), U(t) \in \mathbb R^{(n\times n)}$ are given matrices, wherein $\det A(t) \equiv 0$, $f(t) \in {\mathbb R}^n$ is some continuous on $T$ function; $x(t) \in {\mathbb R}^n$ is unknown function of a state. Such systems are called differential-algebraic equations (DAE). 

%Under some assumptions \cite{pppSCH2008} there exists a linear differential operator
\begin{lemma}{\rm \cite{pppSCH2008}}
Let $A(t), B(t), C(t) \in \mathbb C^r(T),$ ${\rm rank} D_{r,z}(t) = \rho = {\rm const}$, there is a resolving minor in matrix $D_{r,x}(t)$. Then there exists linear differential operator
$$
{\mathcal R}=R_0(t) +R_1(t){{d}\over{dt}}+\ldots +R_r(t)\left( {{d}\over{dt}}\right) ^r
$$
with continuous on $T$ coefficients
$R_j(t)\; (j=\overline{0,r})$, that reduces system (\ref{pppdau}) into the form
$$
\dot{x}_1(t) = J_1(t)x_1(t)+{\mathcal H}(t) \overline{u}(t),
$$
$$
x_2(t) = J_2(t)x_1(t)+{\mathcal G}(t) \overline{u}(t),
$$
where $\left( x_1(t), x_2(t) \right)=Qx(t)$, $\overline{u}(t) = (u(t), \dot{u}(t),\ldots, u^{(r)}(t))$, r is unresolvability index for DAE (\ref{pppdau}). Matrices $D_{r,x}, D_{r,z}$, $J_1(t), J_2(t)$, ${\mathcal H}(t), {\mathcal G}(t)$ are determined by formulas cited in \cite{pppSCH2008}.
\end{lemma}




The following proposition gives a necessary and sufficient condition for the existence of a unique solution to system (\ref{ppppss1}), (\ref{ppppss2}). %(or rather (\ref{ppppss3})--(\ref{ppppss5})).
\begin{theorem}
Let all the assumptions of Lemma 1 are satisfied. 
Then problem (\ref{ppppss1}), (\ref{ppppss2}) solvable on $T$ iff
%\begin{equation}\label{ppppss55}
%\left(\begin{array}{cc}
%0 & 0\\ 0 & E_{n-k}
%\end{array} \right)Qx_0=
%\left(\begin{array}{c} 
%0 \\
% J_2(t_0)z_1(t_0) + {\mathcal{G}(t_0)}\overline{y}(t_0)
%\end{array} \right).
%\end{equation}
\begin{equation}\label{pss55}
z_2(t_0) = 
J_2(t_0)z_1(t_0) + {\mathcal{G}(t_0)}\overline{y}(t_0).
\end{equation}
Moreover, if a solution to problem (\ref{ppppss1}), (\ref{ppppss2}) exists, then it is unique. Here
$(z_1(t), z_2(t)) = Qx(t), \; z_1(t) \in {\mathbb R}^k, z_2(t) \in {\mathbb R}^{n-k}$, $\overline{y}(t) = (y(t), \dot{y}(t), \ldots, y^{(r)}(t))$.
\end{theorem} 
% The figures and tables are drawn according to the standard class 'article'.


% Two picture formats are supported:
%\includegraphics[width=0.7\linewidth]{figure.pdf} % Raster format
%\includegraphics[width=0.7\linewidth]{figure.png} % Vector and raster format
%
% Vector drawings can be drawn in Inkscape editor
% https://inkscape.org/ru/download/
% The usual format of the editor is SVG, so the drawings must be exported in
% PDF or PNG (with a resolution of minimum 150 dpi, and maximum of 300 dpi).


% At the end of the text, acknowledgments are expressed, if you haven't
% made a footnote from the title. For example, we can write
%The research is carried on with support of RFBR (RNF, other funds), project No.~00-00-00000.

\begin{thebibliography}{9} % or {99}, if there is more than ten references.


\bibitem{pppKM2000} Kunze  M., Marques  M.D.M. An Introduction to Moreau’s Sweeping Process. In  Brogliato B. (eds) Impacts in Mechanical Systems. Lecture Notes in Physics. Vol.~551. Springer, Berlin, Heidelberg, 2000.
 
\bibitem{pppMoreau1977} Moreau, J.-J. Evolution problem associated with a moving convex set in a Hilbert space. J. Differential Eq.~1977. Vol.~26. Pp.~347--374.  


\bibitem{pppBS1996} Brokate  M.,  Sprekels  J. Hysteresis and Phase Transitions. Ser. Appl. Math. Sci. Vol.~121.  Springer-Verlag, New York, 1996. 



\bibitem{pppKr1991} Krej\u{c}\'{\i} P..  Vector hysteresis models. European J. Appl. Math.~1996. Vol.~2. Pp.~281--292.


\bibitem{pppSCH2008} Scheglova  A.   Controllability of nonlinear algebraic differential systems. Automat. Telemekh.~2008. Vol.~10, Pp.~57--80.



\end{thebibliography}
%\end{document}

%%% Local Variables:
%%% mode: latex
%%% TeX-master: t
%%% End: