\iffalse

%%%%%%%%%%%%%%%%%%%%%%%%%%%%%%%%%%%%%%%%%%%%%%%%%%%%%%%%%%%%%%%%%%%%%%%%
%
% This is the template file for the 6th International conference
% NONLINEAR ANALYSIS AND EXTREMAL PROBLEMS
% June 25-30, 2018
% Irkutsk, Russia
%
%%%%%%%%%%%%%%%%%%%%%%%%%%%%%%%%%%%%%%%%%%%%%%%%%%%%%%%%%%%%%%%%%%%%%%%%
% The preparation of the article is based on the standard llncs class
% (Lecture Notes in Computer Sciences), which is adjusted with style
% file of the conference.
%
% There are two ways of compilation of the file into PDF
% 1. Use pdfLaTeX (pdflatex), (LaTeX+DVIPS will not work);
% 2. Use LuaLaTeX (XeLaTeX will work too).
% When using LuaLaTeX You will need TTF or OTF CMU fonts
% (Computer Modern Unicode). The fonts are installed with 'cm-unicode' package in
% a distribution of LaTeX % (https://www.ctan.org/tex-archive/fonts/cm-unicode),
% either by downloading and installing these fonts system wide, the address of their page is
% http://canopus.iacp.dvo.ru/%7Epanov/cm-unicode/
% The second option won't work in XeLaTeX.
%
% For MiKTeX (LaTeX distribution for Windows),
%  1. Package 'cm-unicode' is installed manually with the MiKTeX administration Console.
%  2. For the compilation of this example, namely, the stub figure, one will also need to
% download package 'pgf' manually. This package uses in the popular
% package tikz.
%  3. Tests showed that the rest of the required packages MiKTeX loads automatically (if
%     it is allowed). The 'auto download' option is
%     configured in 'Settings' section in MiKTeX Console.
%
%
% The easiest way to compile an article is to use pdfLaTeX, but
% the final layout of the book will be compiled with LuaLaTeX,
% as a result will be of better quality thanks to the package 'microtype' and
% use vector OTF instead of standard raster fonts of pdfLaTeX.
%
% In the case of questions and problems with the article compilation,
% write letters to e-mail: eugeneai@irnok.net, Cherkashin Evgeny.
%
% New version of the correcting style file will be available at the website:
%     https://github.com/eugeneai/nla-style
%     file - nla.sty
%
% Further instructions are in the text body of the template. The template itself
% is an article example.
%
% The LaTeX2e format is used!

% 12 points font size is used.
\documentclass[12pt]{llncs}

% The correcting style file is added.
\usepackage{todonotes}

\usepackage{nla} % This package is needed for compiling
                 % this template, it should be removed
                 % from your article.

% Many popular packages (amsXXX, graphicx, etc.) are already imported in the style file.
% If there is a conflict with your packages, try disabling them and compile
% the text.
%
% It would be convenient in the layout of the proceedings if the file names
% of the figures of different authors do not clash.
% To minimize the clash, the drawings can be placed in a separate subfolder
% named after the author or the title of the paper.
%
% \graphicspath{{ivanov-petrov-pics/}} % specifies the folder with images in png, pdf formats.
% or
% \graphicspath{{great-problem-solving-paper-pics/}}.

\begin{document}

% Text should be formatted in accordance with the 'article' class, using extensions like
% AMS.
%
\fi

\title{A note on differential-algebraic equations with hysteresis phenomena\thanks{The research is supported by the Ministry of Education and Science of the Russian Federation (state registration No.~121041300060-4.}}
% First author
\author{Pavel Petrenko 
}
\institute{ISDCT SB RAS, Irkutsk, Russia\\
  \email{petrenko.pavels@gmail.com}}

% etc

\maketitle

\begin{abstract}
In this note, we investigate some class of differential-algebraic equations (DAE) with hysteresis phenomena. We consider non-stationary DAE with non-linearity of hysteresis type (modeled by a sweeping process). For such a DAE, we design an equivalent structural form (in the sense of solutions), and prove a necessary and sufficient condition for the existence and uniqueness of a solution to an initial value problem.

\keywords{differential-algebraic equations, hysteresis, sweeping process}
\end{abstract}

% at the end of the list, there should be no final dot
\section{The main results}

Consider the following system of ordinary differential equations (ODE) paired with a sweeping process \cite{KM2000,Moreau1977} of a rate independent hysteresis type  (modeled by the play operator \cite{BS1996,Kr1991,PSS2020}):
\begin{align} \label{pss1}
 & A(t)\dot{x}(t)=B(t)x(t)+C(t)y(t), \ \ \ x(t_0)=x_0,   
\\[0.5em] & 
\begin{array}{c}
  -\ddot{y}(t)\in \mathcal{N}_{Q(t)}\big(\dot{y}(t)\big), \\[0.5em] 
 {y}(t_0)=y_0, \ \  \  \dot{y}(t_0)=y_1\in Q(t_0).  
  \end{array} \label{pss2}
\end{align}
Here, $A(t), B(t), C(t) \in \mathbb R^{(n\times n)}$ are given matrices, wherein $\det A(t) \equiv 0$, $T=[t_0,t_1]$ is a given time interval, $x(t), y(t) \in {\mathbb R}^n$ are unknown functions; $Q(t)=x(t)-Z$ is a ``moving set'', where $Z$ is a given closed convex subset of ${\mathbb R}^n$, and $\mathcal{N}_{Q}\big(q\big)$ denotes the normal cone to a closed convex set $Q$ at a point $q$.

By a solution of (\ref{pss1}), (\ref{pss2}) we mean a pair of absolutely continuous ($W^{1,1}$) functions $(x, y)$ satisfying (\ref{pss1}), (\ref{pss2}) for almost all (a.e.) $t \in T$ with respect to (w.r.t.) the usual Lebesgue measure.


Let's consider time-varying ODE system not resolved with respect to the derivative
\begin{equation}\label{dau}
A(t)x'(t)=B(t)x(t) + U(t)f(t), \;\;\; t \in T \subset {\mathbb R},
\end{equation}
where $A(t), B(t), U(t) \in \mathbb R^{(n\times n)}$ are given matrices, wherein $\det A(t) \equiv 0$, $f(t) \in {\mathbb R}^n$ is some continuous on $T$ function; $x(t) \in {\mathbb R}^n$ is unknown function of a state. Such systems are called differential-algebraic equations (DAE). 

%Under some assumptions \cite{SCH2008} there exists a linear differential operator
\begin{lemma}{\rm \cite{SCH2008}}
Let $A(t), B(t), C(t) \in \mathbb C^r(T),$ ${\rm rank} D_{r,z}(t) = \rho = {\rm const}$, there is a resolving minor in matrix $D_{r,x}(t)$. Then there exists linear differential operator
$$
{\mathcal R}=R_0(t) +R_1(t){{d}\over{dt}}+\ldots +R_r(t)\left( {{d}\over{dt}}\right) ^r
$$
with continuous on $T$ coefficients
$R_j(t)\; (j=\overline{0,r})$, that reduces system (\ref{dau}) into the form
$$
\dot{x}_1(t) = J_1(t)x_1(t)+{\mathcal H}(t) \overline{u}(t),
$$
$$
x_2(t) = J_2(t)x_1(t)+{\mathcal G}(t) \overline{u}(t),
$$
where $\left( x_1(t), x_2(t) \right)=Qx(t)$, $\overline{u}(t) = (u(t), \dot{u}(t),\ldots, u^{(r)}(t))$, r is unresolvability index for DAE (\ref{dau}). Matrices $D_{r,x}, D_{r,z}$, $J_1(t), J_2(t)$, ${\mathcal H}(t), {\mathcal G}(t)$ are determined by formulas cited in \cite{SCH2008}.
\end{lemma}




The following proposition gives a necessary and sufficient condition for the existence of a unique solution to system (\ref{pss1}), (\ref{pss2}). %(or rather (\ref{pss3})--(\ref{pss5})).
\begin{theorem}
Let all the assumptions of Lemma 1 are satisfied. 
Then problem (\ref{pss1}), (\ref{pss2}) solvable on $T$ iff
%\begin{equation}\label{pss55}
%\left(\begin{array}{cc}
%0 & 0\\ 0 & E_{n-k}
%\end{array} \right)Qx_0=
%\left(\begin{array}{c} 
%0 \\
% J_2(t_0)z_1(t_0) + {\mathcal{G}(t_0)}\overline{y}(t_0)
%\end{array} \right).
%\end{equation}
\begin{equation}\label{pss55}
z_2(t_0) = 
J_2(t_0)z_1(t_0) + {\mathcal{G}(t_0)}\overline{y}(t_0).
\end{equation}
Moreover, if a solution to problem (\ref{pss1}), (\ref{pss2}) exists, then it is unique. Here
$(z_1(t), z_2(t)) = Qx(t), \; z_1(t) \in {\mathbb R}^k, z_2(t) \in {\mathbb R}^{n-k}$, $\overline{y}(t) = (y(t), \dot{y}(t), \ldots, y^{(r)}(t))$.
\end{theorem} 
% The figures and tables are drawn according to the standard class 'article'.


% Two picture formats are supported:
%\includegraphics[width=0.7\linewidth]{figure.pdf} % Raster format
%\includegraphics[width=0.7\linewidth]{figure.png} % Vector and raster format
%
% Vector drawings can be drawn in Inkscape editor
% https://inkscape.org/ru/download/
% The usual format of the editor is SVG, so the drawings must be exported in
% PDF or PNG (with a resolution of minimum 150 dpi, and maximum of 300 dpi).


% At the end of the text, acknowledgments are expressed, if you haven't
% made a footnote from the title. For example, we can write
%The research is carried on with support of RFBR (RNF, other funds), project No.~00-00-00000.

\begin{thebibliography}{9} % or {99}, if there is more than ten references.


%\bibitem[Adly, Haddad and Thibault, 2014]{AHT2014}  Adly, S.,  Haddad, T. and  Thibault, L. (2014). Convex sweeping process in the framework of measure differential inclusions and evolution variational inequalities. {\em Mathematical Programming B}, Springer,  \textbf{148}(1), pp.~5--47.

%\bibitem[Acary, Bonnefon and Brogliato, 2011]{ABB2011}  Acary, V.,  Bonnefon, O. and  Brogliato, B. (2011). {\em Nonsmooth Modelling and Simulation for Switched Circuits.} Springer Science+Business Media B.V.  

%\bibitem[Brogliato, 1999]{Brog1999}  Brogliato, B. (1999). {\em Nonsmooth Mechanics. Models, Dynamics and Control}. Springer, London.
\bibitem{KM2000} Kunze  M., Marques  M.D.M. An Introduction to Moreau's Sweeping Process. In  Brogliato B. (eds) Impacts in Mechanical Systems. Lecture Notes in Physics. Vol.~551. Springer, Berlin, Heidelberg, 2000.
 
\bibitem{Moreau1977} Moreau  J.-J. Evolution problem associated with a moving convex set in a Hilbert space. J. Differential Eq.~1977. Vol.~26. Pp.~347--374.  

%\bibitem[Brokate and Krej\u{c}\'{\i},  Schnabel, 2004]{BrKrSch2004}  Brokate, M.,     Krej\u{c}\'{\i},  P. and Schnabel, H. (2004). On uniqueness in evolution quasivariational   inequalities.   {\em J. Convex Anal.}, \textbf{11}, pp.~111--130.

\bibitem{BS1996} Brokate  M.,   Sprekels  J. Hysteresis and Phase Transitions. Ser. Appl. Math. Sci. Vol.~121.  Springer-Verlag, New York, 1996. 

%\bibitem[Gantmacher, 1988]{G1988} Gantmacher F.R. (1988). {\em The theory of matrices}. Nauka, Moscow (in Russian).

%\bibitem[Goncharova and Staritsyn, 2017]{GS2017} Goncharova, E. and Staritsyn, M. (2017). Relaxation and Optimization of Impulsive Hybrid Systems: Inspired by Impact Mechanics. In {\em Proceedings of the 14th International Conference on Informatics in Control, Automation and Robotics}, \textbf{1}, pp.~474--485.

%\bibitem[Goncharova and Staritsyn, 2018]{GS2018} Goncharova, E. and Staritsyn, M. (2018). On BV-extension of Asymptotically Constrained Control-affine Systems and Complementarity Problem for Measure Differential Equations. {\em Discrete and Continuous Dynamical Systems-Series S.}, \textbf{11}(6), pp.~1061--1070. 
%DOI: 10.3934/dcdss.2018061. 

%\bibitem[Facchinei and Pang, 2003]{FP2003} Facchinei, F.  and Pang, J.S. (2003). {\em Finite-Dimensional Variational Inequalities and Complementarity Problems}. Springer, New York. 

%\bibitem[Heemels,   Schumacher and Weiland, 2000]{HSW2000} Heemels, W.P.M.H., Schumacher, J.M., Weiland, S. (2000). Linear complementarity systems. {\em SIAM J. Appl. Math.}, \textbf{60}, pp.~1234--1269.

%\bibitem[Karamzin, Oliveira, Pereira and Silva, 2015]{KOPS2015} Karamzin, D.Yu., Oliveira, V.A., Pereira, F.L., and Silva, G.N. (2015). On the properness of the extension of dynamic optimization problems to allow impulsive controls. {\em ESAIM: Control, Optimisation and Calculus of Variations}, \textbf{21}(3), pp.~857--875.

%\bibitem[Kopfova and Recupero, 2016]{KopRec2015} Kopfova, J. and Recupero, V. (2016). BV-norm continuity of sweeping processes driven by a set with constant shape.  ArXiv: 1512.08711  

\bibitem{Kr1991} Krej\u{c}\'{\i}.  Vector hysteresis models. European J. Appl. Math.~1996. Vol.~2. Pp.~281--292.




%\bibitem[Miller, 1996]{Miller1996} Miller, B. (1996). The generalized solutions of nonlinear optimization problems with impulse control. {\em SIAM J. Control Optim.}, \textbf{34}, pp.~142--440.

%\bibitem[Miller and Rubinovich, 2013]{MR2013} Miller, B.M. and Rubinovich E.Ya. (2013). Discontinuous solutions in the optimal control problems and their representation by singular space-time transformations. {\em Autom. Remote Control}, \textbf{74}, pp.~1969--2006.



%\bibitem[Pang and Stewart, 2008]{PS2008} Pang, J.S. and Stewart, D.E. (2008). Differential variational inequalities. {\em Math. Progam. Ser. A.} \textbf{113}, pp.~345--424.  

%\bibitem[Petrenko, 2018]{P2018} Petrenko, P.S. (2018). Robust controllability of nonstationary differential-algebraic Equations with unstructured uncertainty {\em Journal of Mathematical Sciences}, \textbf{239}(2), pp.~123--134.

%\bibitem[Recupero, 2011]{Rec2011} Recupero, V. (2011).  BV solutions of the rate independent inequalities.   {\em Ann. Scuola Norm. Sup. Pisa Cl. Sci.} \textbf{X}(5),  pp.~269--315.

%\bibitem[Recupero, 2015]{Rec2015} Recupero, V. (2015). BV continuous sweeping processes. {\em J. Differential Equation}, \textbf{259}, pp.~4253--4272.

%\bibitem[Samsonyuk, 2019]{SOp2019} Samsonyuk, O.N. (2019). The space-time representation for impulsive control problems with hysteresis.   {\em Communications in Computer and Information Science}, \textbf{974}, pp.~351--366. %DOI: 10.1007/978-3-030-10934-9_25.

%\bibitem[Samsonyuk and Timoshin, 2018]{SamT2018} Samsonyuk, O.N. and Timoshin, S.A. (2018). BV solutions of rate independent processes driven by impulsive controls. {\em IFAC-Papers OnLine}, \textbf{51}(32), pp.~361--366. 
%DOI: 10.1016/j.ifacol.2018.11.410.

%\bibitem[Samsonyuk and Timoshin, 2019]{SamT2019} Samsonyuk, O.N. and Timoshin, S.A. (2019). Optimal control problems with states of bounded variation and hysteresis. {\em Journal of Global Optimization}, \textbf{74}(3), pp.~565--596. %DOI: 10.1007/s10898-019-00752-7.

\bibitem{PSS2020} Petrenko P., Samsonyuk O., Staritsyn M. A note on Differential-Algebraic Systems with Impulsive and Hysteresis Phenomena. Cybernetics  and Physics.~2020. Vol.~9, no.~1. Pp.~51--56.

\bibitem{SCH2008} Scheglova, A. Controllability of nonlinear algebraic differential systems. Automat. Telemekh.~2008. Vol.~10. Pp.~57--80 (Russian).

 
%\bibitem[Sesekin and Zavalishchin, 1997]{SZ1997} Sesekin, A. and Zavalishchin, S. (1997). {\em Dynamic Impulse Systems: Theory and Applications}. Kluwer Academic Publishers, Dordrecht.

%\bibitem[Song,  Krauss,  Kumar and Dupont]{SKKD2001} Song, P., Krauss, P., Kumar, V., Dupont, P. (1999). Analysis of rigid-body dynamic models for simulation of systems with frictional contacts. {\em J. Appl. Mech.}, \textbf{68}, pp.~118--128.


% \bibitem[Staritsyn, 2018]{St2018} Staritsyn, M. (2018). Quasi-solutions of complementarity systems with measures and related mixed-constrained optimal impulsive control problems. In {\em Proc. of the 14th Viennese Conf. ``Optimal Control and Dynamic Games''} Vienna, Austria, July 3-6,  p.~160.

%\bibitem[Staritsyn and Goncharova, 2019]{GS2019} Staritsyn, M. and Goncharova, E. (2019). On complementarity measure-driven dynamical systems, {\em Lecture Notes in Electrical Engineering}, \textbf{495}, pp.~699--718. 

%\bibitem[Stewart, 2000]{Stew2000} Stewart, D.E. (2000). Rigid-body dynamics with friction and impact. {\em SIAM Rev.}, \textbf{42}, pp.~3--39.

\end{thebibliography}


%\end{document}

%%% Local Variables:
%%% mode: latex
%%% TeX-master: t
%%% End:
