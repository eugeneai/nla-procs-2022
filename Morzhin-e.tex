
\iffalse
\documentclass[12pt]{llncs}
\usepackage{todonotes}
\usepackage{nla}

\begin{document}
\fi

\title{On Optimizing Coherent and Incoherent Controls in 
Some Open Quantum Systems\thanks{The talk presents the results partially supported by the Russian Science Foundation under grant no. 22-11-00330 (https://rscf.ru/en/project/22-11-00330/), 
Ministry of Science and Higher Education of the Russian Federation (project no. 075-15-2020-788).}} 

\author{Oleg V. Morzhin 
}
\institute{Steklov Mathematical Institute of Russian Academy of Sciences,\\
	Department of Mathematical Methods for Quantum Technologies, Moscow, Russia  
  \and
National University of Science and Technology ``MISiS'', Moscow, Russia\\
\email{morzhin.oleg@yandex.ru}}  

\maketitle

\begin{abstract}
The talk is devoted to describing our various results for optimizing 
coherent and incoherent controls in some open quantum systems. 
Along the way, various important aspects of such optimization are noted. 

\keywords{quantum control, coherent control, incoherent control, optimization, qubit, two-level quantum system}

\sloppy

\end{abstract}

Quantum optimal control is an important interdisciplinary direction 
\cite{ccMoore2011,ccKoch_arXiv2022,ccDAlessandroBook2021,ccMorzhinRMS2019,ccMIPTCourseQCS}, 
where quantum-mechanical equations (Schr\"{o}dinger, von Neumann equations, etc.) 
are considered with controls, which represent various external actions including
laser light, for modelling such changes of a quantum object 
that some set of objective criteria is satisfied, e.g., time-minimal
steering to a given target density matrix, $\rho_0 \to \rho_{\rm target}$. 

A powerful method of {\it incoherent control} of open quantum systems, possibly with simultaneous
coherent control, was proposed and studied for various quantum systems in~\cite{ccPechenPRA2006,ccPechenPRA2011}. As the main result, approximate controllability in the set of all density matrices for generic quantum systems was shown~\cite{ccPechenPRA2011} as well as the ability to generate all-to-one Kraus maps previously studied for quantum control in~\cite{ccWuJPA2007}. Reachable sets for a qubit driven by coherent and incoherent controls are explicitly described using geometric control theory~\cite{ccLokutsievskiyJPA2021}. This method of incoherent control forms the base of this talk. 

For the open-loop control type, this talk describes various results for optimizing
coherent and  incoherent controls in some open quantum systems. The talk discusses results of the works~\cite{ccMorzhinLJM2019,ccMorzhinPhysPartNucl2020,ccMorzhinProcSteklov2021,ccMorzhin_arXiv2022}, etc., and describes using, e.g., 
the Pontryagin maximum principle, Gabasov second order necessary condition for 
optimality~\cite{ccMorzhinPhysPartNucl2020},
various optimization methods. The following important aspects are noted:
(a)~importance to take into account the specific of a certain class of quantum optimal control problems
for adaption or creation of such methods that allow to approximately solve these problems
(e.g., the fundamental result proposed in the article~\cite{ccPechenPRA2011}
for approximate steering $\rho_0 \to \rho_{\rm target}$, 
approximate controllability for $N$-level open quantum systems driven by coherent and incoherent controls  
is based on analysis of the matrix specific including eigenvalues of $\rho_{\rm target}$, while
the work~\cite[Sec.~4]{ccMorzhin_arXiv2022} modifies this steering method for $N=2$
by using a wider class of incoherent controls that 
complicates the method, but makes it possible to reduce the duration, 
at that one can choose between the original and modified methods taking into account his requirements); 
(b)~such possible ``underwater rocks'' for optimization methods that the situation when some control satisfies 
the stationarity condition / Pontryagin maximum principle, but this control is not globally 
optimal (e.g.,~\cite[Sec.~7]{ccMorzhinProcSteklov2021}, \cite[Sec.~3]{ccMorzhin_arXiv2022}), unreachability of $\rho_{\rm target}$ for a given final time;
(c)~analyzing profiles of control processes numerically obtained that are claim to be approximately 
global solutions (e.g., see Fig.~4--6 in \cite{ccMorzhinLJM2019}).

\begin{thebibliography}{99}   

\bibitem{ccMoore2011}
Moore K.W., Pechen A., Feng X.-J., Dominy~J., Beltrani~V.J., Rabitz~H. 
Why is chemical synthesis and property optimization easier than expected? 
Physical Chemistry Chemical Physics. 2011. Vol.~13, no~21. Pp.~10048--10070.

\bibitem{ccKoch_arXiv2022}
Koch C.P., Boscain U., Calarco~T., 
Dirr~G., Filipp~S., Glaser~S.J., Kosloff~R., Montangero~S., 
Schulte-Herbr\"{u}ggen~T., Sugny~D., Wilhelm~F.K. Quantum optimal control in quantum technologies. Strategic report on current status, visions and goals for research in Europe. 
2022. https://arxiv.org/abs/2205.12110

\bibitem{ccDAlessandroBook2021} 
D'Alessandro D. Introduction to Quantum Control and Dynamics. 2nd~ed. Boca Raton, Chapman and Hall/CRC, 2021.

\bibitem{ccMorzhinRMS2019}  
Morzhin O.V., Pechen A.N. Krotov method for optimal control of closed quantum systems. Russian Math. Surveys. 2019. Vol.~74, no~5. Pp.~851--908.

\bibitem{ccMIPTCourseQCS}
Pechen A.N., Morzhin O.V. Lecture course ``Control of quantum systems'',
MIPT chair ``Methods of Modern Mathematics'' based at Steklov Mathematical Institute.  
http://www.mathnet.ru/conf1849 (2020), http://www.mathnet.ru/conf1989 (2021).
 
\bibitem{ccPechenPRA2006}    
Pechen A., Rabitz H. Teaching the environment to control quantum 
systems. Phys. Rev.~A. 2006. Vol.~73, no~6. Art.~no~062102.  

\bibitem{ccPechenPRA2011} 
Pechen A. Engineering arbitrary pure and mixed quantum 
states. Phys. Rev.~A. 2011. Vol.~84, no.~4. Art.~no~042106. 

\bibitem{ccWuJPA2007}
Wu R., Pechen A., Brif C., Rabitz H. Controllability of open quantum systems 
with Kraus-map dynamics. J.~Phys.~A: Math. Theor.  2007. Vol.~40, no~21. Pp.~5681--5693.

\bibitem{ccLokutsievskiyJPA2021} 
Lokutsievskiy L., Pechen A. Reachable sets for two-level open quantum systems 
driven by coherent and incoherent controls. J.~Phys.~A: Math. Theor. 
2021. Vol.~54, no~39. Art. no~395304.

\bibitem{ccMorzhinLJM2019}
Morzhin O.V., Pechen A.N. Maximization of the overlap between 
density matrices for a two-level open quantum system driven 
by coherent and incoherent controls. Lobachevskii J. Math. 2019. Vol.~40, no~10.  Pp.~1532--1548.

\bibitem{ccMorzhinPhysPartNucl2020}
Morzhin O.V., Pechen A.N. Maximization of the Uhlmann--Jozsa 
fidelity for an open two-level quantum system with coherent and 
incoherent controls. Phys. Part. Nucl. 2020. Vol.~51, no~4. Pp.~464--469.

\bibitem{ccMorzhinProcSteklov2021}
Morzhin O.V., Pechen A.N. On reachable and controllability sets for
time-minimal control of an open two-level quantum system. Proc. Steklov Inst. Math. 2021. Vol.~313. Pp.~149--164. 

\bibitem{ccMorzhin_arXiv2022}
Morzhin O.V., Pechen A.N. On optimization of coherent and incoherent controls for two-level quantum systems.  2022. https://arxiv.org/abs/2205.02521 

\end{thebibliography}
%\end{document} 
