\begin{englishtitle} % Настраивает LaTeX на использование английского языка
% Этот титульный лист верстается аналогично.
\title{The First Initial-boundary Value Problem for Oskolkov System of Nonzero Order}
% First author
\author{A.O. Kondyukov}
\institute{The Yaroslav-the-Wise Novgorod State University (NovSU), 	Veliky Novgorod, Russia\\
  \email{s181303@std.novsu.ru}
  }
% etc

\maketitle

\begin{abstract}
The phase space of the first initial-boundary value problem for a system of partial differential equations modeling the flow of the incompressible viscoelastic Kelvin-Voigt fluid of nonzero order is described. The investigation is based  in frames of semi-linear sobolev type equations on the concept of relatively spectral limited operator and quasi-stationary trajectory for the corresponding Oskolkov system modeling the plane--parallel flow of the above fluid.

\keywords{sobolev type equations, phase space, quasi-stationary trajectories, Oskolkov systems, incompressible viscoelastic Kelvin-Voigt fluid} % в конце списка точка не ставится
\end{abstract}
\end{englishtitle}

\iffalse
%%%%%%%%%%%%%%%%%%%%%%%%%%%%%%%%%%%%%%%%%%%%%%%%%%%%%%%%%%%%%%%%%%%%%%%%
%
%  This is the template file for the 6th International conference
%  NONLINEAR ANALYSIS AND EXTREMAL PROBLEMS
%  June 25-30, 2018
%  Irkutsk, Russia
%
%%%%%%%%%%%%%%%%%%%%%%%%%%%%%%%%%%%%%%%%%%%%%%%%%%%%%%%%%%%%%%%%%%%%%%%%

%  Верстка статьи осуществляется на основе стандартного класса llncs
%  (Lecture Notes in Computer Sciences), который корректируется стилевым
%  файлом конференции.
%
%  Скомпилировать файл в PDF можно двумя способами:
%  1. Использовать pdfLaTeX (pdflatex), (LaTeX+DVIPS не работает);
%  2. Использовать LuaLaTeX (XeTeX не будет работать).
%  При использовании LuaLaTeX потребуются TTF- или OTF-шрифты CMU
%  (Computer Modern Unicode). Шрифты устанавливаются либо пакетом
%  дистрибутива LaTeX cm-unicode
%              (https://www.ctan.org/tex-archive/fonts/cm-unicode),
%  либо загрузкой и установкой в операционной системе, адрес страницы:
%  http://canopus.iacp.dvo.ru/%7Epanov/cm-unicode/
%
%  В MiKTeX (дистрибутив LaTeX для ОС Windows):
%  1. Пакет cm-unicode устанавливается вручную в программе MiKTeX Console.
%  2. Для верстки данного примера, а именно, картинки-заглушки необходимо,
%     также вручную, загрузить пакет pgf. Этот пакет используется популярным
%     пакетом tikz.
%  3. Тест показал, что остальные пакеты MiKTeX грузит автоматически (если
%     ему разрешено автоматически грузить пакеты). Режим автозагрузки
%     настраивается в разделе Settings в MiKTeX Console.
%
%
%  Самый простой способ сверстать статью - использовать pdfLaTeX, но
%  окончательный вариант верстки сборника будет собран в LuaLaTeX,
%  так как результат получится лучшего качества.
%
%  В случае возникновения вопросов и проблем с версткой статьи,
%  пишите письма на электронную почту: eugeneai@irnko.net, Черкашин Евгений
%
%  Новые варианты корректирующего стиля будут доступны на сайте:
%        https://github.com/eugeneai/nla-style
%        файл - nla.sty
%
%  Дальнейшие инструкции - в тексте данного шаблона. Он одновременно
%  является примером статьи.
%
%  Формат LaTeX2e!

\documentclass[12pt]{llncs}  % Необходимо использовать шрифт 12 пунктов.

% При использовании pdfLaTeX добавляется стандартный набор русификации babel.
% Если верстка производится в LuaLaTeX, то следующие три строки надо
% закомментировать, русификация будет произведена в корректирующем стиле автоматом.
\usepackage[T2A]{fontenc}
\usepackage[cp1251]{inputenc} % Кодировка utf-8, win1251 (cp1251) не тестировалась.
\usepackage[english,russian]{babel}

% Для верстки в LuaLaTeX текст готовится строго в utf-8!

% В операционной системе Windows для редактирования в кодировки utf-8
% можно использовать программы notepad++ https://notepad-plus-plus.org/,
% techniccenter http://www.texniccenter.org/,
% SciTE (самая маленькая по объему программа) http://www.scintilla.org/SciTEDownload.html
% Подойдет также и встроенный в свежий дистрибутив MiKTeX редактор
% TeXworks.

% Добавляется корректирующий стилевой файл строго после babel, если он был включен.
% В параметре необходимо указать russian, что установит не совсем стандартные названия
% разделов текста, настроит переносы для русского языка как основного и т.п.

%\usepackage{todonotes} % Удрать из вашей статьи, нужен для верстки данного шаблона.

\usepackage[russian]{nla}

% Многие популярные пакеты (amsXXX, graphicx и т.д.) уже импортированы в корректирующий стиль.
% Если возникнут конфликты с вашими пакетами, попробуйте их отключить и сверстать
% текст как есть.
%
%


% Было б удобно при верстке сборника, чтобы названия рисунков разных авторов не пересекались.
% Чтоб минимизировать такое пересечение, рисунки можно поместить в отдельную подпапку с
% названием - фамилией автора или названием статьи.
%
% \graphicspath{{ivanov-petrov-pics/}} % Указание папки с изображениями в форматах png, pdf.
% или
% \graphicspath{{great-problem-solving-paper-pics/}} % Указание папки с изображениями в форматах png, pdf.


\begin{document}

% Текст оформляется в соответствии с классом article, используя дополнения
% AMS.
%
\fi

\title{Первая начально-краевая задача для системы Осколкова ненулевого порядка}
% Первый автор
\author{А.~О.~Кондюков}  % \inst ставит циферку над автором.
  


% Аффилиации пишутся в следующей форме, соединяя каждый институт при помощи \and.
\institute{Новгородский государственный университет имени Ярослава Мудрого (НовГУ),\\ Великий Новгород, Российская Федерация\\
  \email{s181303@std.novsu.ru}}

\maketitle

\begin{abstract}
Описано фазовое пространство первой начально-краевой задачи для системы уравнений в частных производных, моделирующей  течение несжимаемой вязкоупругой жидкости Кельвина-Фойгта ненулевого порядка. Исследование проводится в рамках теории полулинейных уравнений соболевского типа на основе понятий относительно спектрально ограниченного оператора  и квазистационарной траектории для соответствующей системы Осколкова, моделирующей плоскопараллельное течение вышеуказанной жидкости.

\keywords{уравнения соболевского типа, фазовое пространство, квазистационарные траектории, системы Осколкова, несжимаемая вязкоупругая жидкость Кельвина--Фойгта.} % в конце списка точка не ставится
\end{abstract}


Система уравнений Осколкова

\begin{equation}
\left \{
\begin{array}{l}
(1- \lambda {\nabla}^2)v_{t}=\nu\nabla^{2}v-(v\cdot\nabla)v+\displaystyle\sum_{l=1}^K \beta_{l}\nabla^{2} w_{l}-\nabla p+f,  \\
\mathop{\mathop{0=\nabla  \cdot { v},\quad}\limits^{\ }}
\limits^{\ }\\
   {\displaystyle\mathop{\mathop{\dfrac{\partial w_{l}}{\partial t}=v+\alpha_{l}w_{l},~ \alpha_{l}\in {\mathbb{R_-}},~ \beta_{l}\in {\mathbb{R_+}},~ l=\overline{1,~K}
\quad}\limits^{\ }}\limits^{\ }}
\end{array} \right.
\end{equation}%(1)
моделирует динамику вязкоупругой несжимаемой жидкости Кельвина-Фойгта ненулевого порядка $ K $ \cite{rrr1}. Здесь $ v=(v_{1}, v_{2},\ldots ,v_{n}),~ v_{k}=v_{k}(x,t) $ --- вектор скорости жидкости, $p=p(x,t)$ --- функция давления, $ f=(f_{1},f_{2},\ldots ,f_{n}),~ f_{k}=f_{k}(x) $ -- вектор внешнего воздействия в точке $ (x,t)\in\Omega\times {\mathbb{R}},~ \Omega\subset {\mathbb{R}}^{n}$ --- ограниченная область с границей класса $ C^{\infty}. $ Параметры $ \lambda,~ \nu \in {\mathbb{R_+}} $ характеризуют упругие и вязкие свойства жидкости соответсвенно.

 %В случае жидкости Кельвина-Фойгта  нулевого порядка $(K=0)$ задача Коши-Дирихле для системы уравнений (1) в цилиндре $ \Omega\times {\mathbb{R}} $ изучена в различных аспектах, причем экспериментально подтверждено, что параметр $ \lambda $ может принимать отрицательные значения. Хорошо известно, что при некоторых отрицательных значениях параметра $ \lambda $ (а именно, $ \lambda^{-1}\in\sigma(\nabla^{2}) $) такая задача, вообще говоря, не разрешима. Поэтому возникает проблема описания множества корректности этой задачи, понимаемого нами как фазовое пространство \cite{rrr6}.
 
Пусть $ \Omega\subset {\mathbb{R}}^{2} $ --- ограниченная область с границей $ \partial\Omega $ класса $ C^{\infty} $. В цилиндре $ \Omega\times {\mathbb{R}} $ рассмотрим задачу Коши-Дирихле

\begin{equation}
\begin{array}{l}
\psi(x,y,0)=\psi_{0}(x,y), w_{l}(x,y,0)=w_{l0}(x,y) ~ \forall(x,y)\in\Omega \\
\mathop{\mathop{
\psi(x,y,t)=\nabla^{2}\psi(x,y,t)=0, w_{l}(x,y,t)=0 ~ \forall(x,y,t)\in\partial\Omega\times {\mathbb{R}}
\quad}\limits^{\ }}
\end{array}
\end{equation}%(2)
для системы уравнений
\begin{equation}
\left \{
\begin{array}{l}
(1- \lambda {\nabla}^2)\nabla^{2}\psi_{t}=\nu\nabla^{4}\psi-\dfrac{\partial(\psi,\nabla^{2}\psi)}{\partial(x,y)}+\displaystyle\sum_{l=1}^K\beta_{l}\nabla^{2}\left(\dfrac{\partial w_{l1}}{\partial y}-\dfrac{\partial w_{l2}}{\partial x}\right)+g ,  \\
{\displaystyle\mathop{\mathop{\dfrac{\partial w_{l1}}{\partial t}}=\frac{\partial\psi}{\partial y}+\alpha_{l} w_{l1},\quad}\limits^{\ }}\\
{\displaystyle\mathop{\mathop{\dfrac{\partial w_{l2}}{\partial t}}=-\frac{\partial\psi}{\partial x}+\alpha_{l} w_{l2},~ \alpha_{l}\in {\mathbb{R_-}},~ l=\overline{1,~K}, \quad}\limits^{\ }}\end{array}\right.
\end{equation}%(3)
которая получится из системы (1) при $ n=3 $, если положить $ v_{3}\equiv 0 $ и формулами $ v_{1}=\dfrac{\partial\psi}{\partial y},~ v_{2}=-\dfrac{\partial\psi}{\partial x} $ ввести функцию тока $ \psi=\psi(x,y,t), $ определенную с точностью до аддитивной постоянной.Таким образом, система (3) моделирует плоскопараллельное течение вязкоупругой несжимаемой  жидкости Кельвина --- Фойгта $K$--го порядка. Преимущество  задачи (2), (3) по сравнению с задачей Коши-Дирихле для уравнения (1) заключается в том, что фазовое пространство \cite{rrr6}  уравнения (3) может быть описано полностью при любых значениях параметра $ \lambda\in {\mathbb{R}}. $

Для системы (3) при $ K=0 $ задача Коши-Дирихле рассмотрена в \cite{rrr7}. При $ K\neq0 $  существование квазистационарных траекторий  в случае $ \lambda^{-1}\in\sigma(\nabla^{2}) $ и описание структуры фазового пространства приводится в \cite{rrr8}. Результаты этого исследовния дополняют результаты \cite{rrr9}.



% Современные издательства требуют использовать кавычки-елочки << >>.

% В конце текста можно выразить благодарности, если этого не было
% сделано в ссылке с заголовка статьи, например,
Автор выражает благодарность профессору Т. Г. Сукачевой за внимание и конструктивную критику.
%

% Список литературы оформляется подобно ГОСТ-2008.
% Примеры оформления находятся по этому адресу -
%     https://narfu.ru/agtu/www.agtu.ru/fad08f5ab5ca9486942a52596ba6582elit.html
%

\begin{thebibliography}{9} % или {99}, если ссылок больше десяти.
\bibitem{rrr1} Осколков А. П. Начально-краевые задачи для уравнений движения жидкостей Кельвина-Фойгта и Олдройта// Тр. Мат. института им. В. А. Стеклова. 1988. Т. 179. С. 126-164.

%\bibitem{rrr2} Свиридюк Г. А. Об одной модели динамики несжимаемой вязкоупругой жидкости// Известия вузов. Математика. 1988. №1. С. 74-79.

%\bibitem{rrr3} Осколков А. П. Об одной квазилинейной параболической системе с малым параметром, аппроксимирующей систему Навье-Стокса// Зап. научн. сем. ЛОМИ. 1980. Т. 96. С. 233-236.

%\bibitem{rrr4} Свиридюк Г. А. О многообразии решений одной задачи несжимаемой вязкоупругой жидкости // Диференц. уравнения. 1988. т. 24, № 10. С. 1846-1848.

%\bibitem{rrr5} Свиридюк Г. А., Сукачева Т. Г. Фазовые пространства одного класса операторных уравнений// Дифференц. уравнения. 1990. Т. 26. № 2. С. 250-258.

\bibitem{rrr6} Свиридюк Г. А., Сукачева Т. Г. Задача Коши для одного класса полулиненых уравнений типа Соболева// Сиб. мат. журн. 1990. Т. 31. № 5. С. 109-119.

\bibitem{rrr7} Свиридюк Г. А., Якупов М. М. Фазовое пространство начально-краевой задачи для системы  Осколкова// Диференц. уравнения. 1996. Т. 32. № 11. С. 1538-1543.

\bibitem{rrr8} A.O. Kondyukov, T.G. Sukacheva, Phase space of the initial-boundary value problem for the Oskolkov system of nonzero order , published in Zhurnal Vychislitel’noi Matematiki i Matematicheskoi Fiziki, 2015, Vol. 55, No. 5, pp. 823–829.

\bibitem{rrr9} Сукачева Т. Г. Об одной модели движения несжимаемой вязкоупругой жидкости Кельвина-Фойгта ненулевого порядка// Диференц. уравнения. 1997. Т. 33. № 4. С. 552-557.

\end{thebibliography}

% После библиографического списка в русскоязычных статьях необходимо оформить
% англоязычный заголовок.




%\end{document}

%%% Local Variables:
%%% mode: latex
%%% TeX-master: t
%%% End:
