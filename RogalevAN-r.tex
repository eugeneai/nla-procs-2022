\begin{englishtitle} % Настраивает LaTeX на использование английского языка
% Этот титульный лист верстается аналогично.
\title{Numerical Estimation of the Boundaries of the Reachability Sets of Controlled Systems Based on Symbolic Formulas\thanks{
Работа была выполнена при поддержке ФИЦ КНЦ СО РАН, проект \textnumero~0287-2021-0002.} }
% First author
\author{A.N. Rogalev  
  %\and
 % Name FamilyName2\inst{2}
  %\and
  %Name FamilyName3\inst{1}
}
\institute{Institute of computing modelling SB RAS, Krasnoyarsk, Russia\\
  \email{rogalyov@icm.krasn.ru}
%\email{email}
}
% etc
\maketitle
\begin{abstract}
Controlled systems described by ordinary differential equations with initial or boundary conditions are considered.  In the report  the inclusion of the reachability domain of control systems using a guaranteed method of estimating the sets of solutions of systems of ordinary differential equations on the basis of symbolic formulas for approximating the shift operator along the trajectory are building.

\keywords{Uncertain dynamic systems, Reachability, Symbolic formulas } % в конце списка точка не ставится
\end{abstract}
\end{englishtitle}

\iffalse

%%%%%%%%%%%%%%%%%%%%%%%%%%%%%%%%%%%%%%%%%%%%%%%%%%%%%%%%%%%%%%%%%%%%%%%%
%
%  This is the template file for the 6th International conference
%  NONLINEAR ANALYSIS AND EXTREMAL PROBLEMS
%  June 25-30, 2018
%  Irkutsk, Russia
%
%%%%%%%%%%%%%%%%%%%%%%%%%%%%%%%%%%%%%%%%%%%%%%%%%%%%%%%%%%%%%%%%%%%%%%%%

%  Верстка статьи осуществляется на основе стандартного класса llncs
%  (Lecture Notes in Computer Sciences), который корректируется стилевым
%  файлом конференции.
%
%  Скомпилировать файл в PDF можно двумя способами:
%  1. Использовать pdfLaTeX (pdflatex), (LaTeX+DVIPS не работает);
%  2. Использовать LuaLaTeX (XeTeX не будет работать).
%  При использовании LuaLaTeX потребуются TTF- или OTF-шрифты CMU
%  (Computer Modern Unicode). Шрифты устанавливаются либо пакетом
%  дистрибутива LaTeX cm-unicode
%              (https://www.ctan.org/tex-archive/fonts/cm-unicode),
%  либо загрузкой и установкой в операционной системе, адрес страницы:
%  http://canopus.iacp.dvo.ru/%7Epanov/cm-unicode/
%
%  В MiKTeX (дистрибутив LaTeX для ОС Windows):
%  1. Пакет cm-unicode устанавливается вручную в программе MiKTeX Console.
%  2. Для верстки данного примера, а именно, картинки-заглушки необходимо,
%     также вручную, загрузить пакет pgf. Этот пакет используется популярным
%     пакетом tikz.
%  3. Тест показал, что остальные пакеты MiKTeX грузит автоматически (если
%     ему разрешено автоматически грузить пакеты). Режим автозагрузки
%     настраивается в разделе Settings в MiKTeX Console.
%
%
%  Самый простой способ сверстать статью - использовать pdfLaTeX, но
%  окончательный вариант верстки сборника будет собран в LuaLaTeX,
%  так как результат получится лучшего качества.
%
%  В случае возникновения вопросов и проблем с версткой статьи,
%  пишите письма на электронную почту: eugeneai@irnko.net, Черкашин Евгений
%
%  Новые варианты корректирующего стиля будут доступны на сайте:
%        https://github.com/eugeneai/nla-style
%        файл - nla.sty
%
%  Дальнейшие инструкции - в тексте данного шаблона. Он одновременно
%  является примером статьи.
%
%  Формат LaTeX2e!

\documentclass[12pt]{llncs}  % Необходимо использовать шрифт 12 пунктов.

% При использовании pdfLaTeX добавляется стандартный набор русификации babel.
% Если верстка производится в LuaLaTeX, то следующие три строки надо
% закомментировать, русификация будет произведена в корректирующем стиле автоматом.
\usepackage[T2A]{fontenc}
\usepackage[cp1251]{inputenc} % Кодировка utf-8, win1251 (cp1251) не тестировалась.
\usepackage[english,russian]{babel}

% Для верстки в LuaLaTeX текст готовится строго в utf-8!

% В операционной системе Windows для редактирования в кодировки utf-8
% можно использовать программы notepad++ https://notepad-plus-plus.org/,
% techniccenter http://www.texniccenter.org/,
% SciTE (самая маленькая по объему программа) http://www.scintilla.org/SciTEDownload.html
% Подойдет также и встроенный в свежий дистрибутив MiKTeX редактор
% TeXworks.

% Добавляется корректирующий стилевой файл строго после babel, если он был включен.
% В параметре необходимо указать russian, что установит не совсем стандартные названия
% разделов текста, настроит переносы для русского языка как основного и т.п.

\usepackage{todonotes} % Удрать из вашей статьи, нужен для верстки данного шаблона.

\usepackage[russian]{nla}

% Многие популярные пакеты (amsXXX, graphicx и т.д.) уже импортированы в корректирующий стиль.
% Если возникнут конфликты с вашими пакетами, попробуйте их отключить и сверстать
% текст как есть.
%
%


% Было б удобно при верстке сборника, чтобы названия рисунков разных авторов не пересекались.
% Чтоб минимизировать такое пересечение, рисунки можно поместить в отдельную подпапку с
% названием - фамилией автора или названием статьи.
%
% \graphicspath{{ivanov-petrov-pics/}} % Указание папки с изображениями в форматах png, pdf.
% или
% \graphicspath{{great-problem-solving-paper-pics/}} % Указание папки с изображениями в форматах png, pdf.


\begin{document}

% Текст оформляется в соответствии с классом article, используя дополнения
% AMS.
%
\fi

\title{Численная оценка границ множеств достижимости управляемых систем на основе символьных формул}
% Первый автор
\author{А.~Н.~Рогалев  % \inst ставит циферку над автором.
} % обязательное поле

% Аффилиации пишутся в следующей форме, соединяя каждый институт при помощи \and.
\institute{ИВМ СО РАН, Красноярск, Россия\\
  \email{rogalyov@icm.krasn.ru}
 }

\maketitle

\begin{abstract}
В докладе описывается алгоритм построения включения областей достижимости управляемой системы, использующий символьные формулы, а также характеристики множеств достижимости. Оцениваются вычислительные затраты предложенного алгоритма в сравнении с затратами некоторых известных алгоритмов оценки множеств достижимости.  Приводятся примеры оценки множеств достижимости.

\keywords{Динамические системы с возмущениями, области достижимости, символьные формулы, граничные и внутренние точки } % в конце списка точка не ставится
\end{abstract}

\section{Основные результаты} % не обязательное поле

При решении многих задач движение  управляемой системы описывается  системой дифференциальных уравнений:
\begin{equation}\frac{dy}{dt} = f (t, y, u) ,\end{equation}
с заданным классом допустимых управлений $u \in U$ , с известными начальным и конечным состоянием управляемой системы, при выполнении условий, наложенных на правую часть системы дифференциальных уравнений.
Множеством достижимости   в момент времени  $t$ называется множество всех точек из фазового пространства, в которые можно перейти на отрезке времени $[t_{0},T ]$   из всех возможных точек начального множества фазовых состояний  по решениям системы (1) с начальным условием   и с допустимым управлением.  Понятие области достижимости позволяет решать различные конкретные задачи теории управления. Это подтверждает большое число публикаций, посвященных вопросам  оценивания множеств достижимости. В статье приводятся ссылки к нескольким статьям \cite{Kurzhansky}-\cite{Filippova}, сделать полный обзор всех работ, посвященных оцениванию множеств достижимости -- это задача, требующая существенно больших ресурсов и выходящая за рамки данных тезисов.

 В докладе описывается включение множеств достижимости  с помощью гарантированного метода, основанного на символьной формуле оператора сдвига вдоль  траектории и вычислении множеств значений этой формулы 
 на области всех управляющих воздействий \cite{Rogalev1}-\cite{Rogalev6}.
Символьная формула (аналитическое выражение) – это запись символьной  формы включающей знаки операций, имена функций и констант, вычисление по которой позволяет получить значение решения.
Для получения символьных формул использовалось последовательное исполнение метода хранения и переработки символьной информации при продвижении вдоль траектории решений на основе статичного хранения этой информации, работы с адресацией памяти с помощью функций поточной обработки. %\begin{figure}[htb]


% Рисунки и таблицы оформляются по стандарту класса article. Например,

%\begin{figure}[htb]
 % \centering
  % Поддерживаются два формата:
  %\includegraphics[width=0.7\linewidth]{figure.pdf} % Растровый формат
  %\includegraphics[width=0.7\linewidth]{figure.png} % Векторный и растровый формат
  %
  % Векторные рисунки можно рисовать в редакторе Inkscape
  % https://inkscape.org/ru/download/
  % Основной формат этого редактора - SVG, поэтому рисунки необходимо экспортировать в
  % PDF или PNG (с разрешением - минимум 150 dpi, максимум - 300dpi).
  %\begin{center}
   % \missingfigure[figwidth=0.7\linewidth]{Уберите меня из статьи!}
  %\end{center}
  %\caption{Заголовок рисунка}\label{fig:example}
%\end{figure}

% Современные издательства требуют использовать кавычки-елочки << >>.

% В конце текста можно выразить благодарности, если этого не было
% сделано в ссылке с заголовка статьи, например,
%Работа выполнена при поддержке РФФИ (РНФ, другие фонды), проект \textnumero~00-00-00000.
%

% Список литературы оформляется подобно ГОСТ-2008.
% Примеры оформления находятся по этому адресу -
%     https://narfu.ru/agtu/www.agtu.ru/fad08f5ab5ca9486942a52596ba6582elit.html
%

\begin{thebibliography}{9} % или {99}, если ссылок больше десяти.
%\bibitem{Gantmakher} Гантмахер~Ф.Р. Теория матриц. М.:~Наука,~1966.

%\bibitem{Kholl} Современные численные методы решения обыкновенных дифференциальных уравнений~/ Под~ред.~Дж.~Холл, Дж.~Уатт. М.:~Мир,~1979.

%\bibitem{Aleksandrov1} Александров~А.Ю. Об устойчивости сложных систем в критических случаях~// Автоматика и телемеханика. 2001. \textnumero~9. С.~3--13.

%\bibitem{Moreau1977} Moreau~J.-J. Evolution problem associated with a moving convex set in a Hilbert space~// J.~Differential~Eq. 1977. Vol.~26. Pp.~347--374.

%\bibitem{Semenov} Семенов~А.А. Замечание о вычислительной сложности известных предположительно односторонних функций~// Тр.~XII Байкальской междунар. конф. <<Методы оптимизации и их приложения>>. Иркутск, 2001. С.~142--146.
\bibitem{Kurzhansky} Куржанский~А.Б., Управление и наблюдение в условиях неопределенности. M.:~ Наука, 1977 .

\bibitem{Chernousjko}Черноусько~Ф.Л., Оценивание фазового состояния динамических систем, М.:~ Наука, 1988.

\bibitem{Tyatushkin} Тятюшкин~А. И.,  Моржин О. В.  Численное исследование множеств достижимости нелинейных управляемых дифференциальных систем // Автомат. и телемеханика. 2011. \textnumero 6. C.160--170.

\bibitem{Filippova}Филиппова~Т.Ф. Оценки множеств достижимости управляемых систем с нелинейностью и параметричским возмущениями // Труды института математики и механики УрО РАН. 2014. Т.20. \textnumero 4. С. 287--296.

%\bibitem{Rogalev1} Новиков~ В.А.,   Рогалев~ А.Н., Построение сходящихся верхних и нижних оценок решений систем обыкновенных дифференциальных уравнений // Журнал вычислительной математики и математической физики. %1993. 33(2). С. 219-–231.

\bibitem{Rogalev1} Рогалев~ А.Н. Гарантированные методы решения систем обыкновенных дифференциальных уравнений на основе преобразования символьных формул // Вычислительные технологии. 2003. Т. 8(5). С.102--116.

\bibitem{Rogalev2} Рогалев~А.Н. Символьные вычисления в гарантированных методах, выполненные на нескольких процессорах // Вестник НГУ. Серия: Информационные технологии.  2006.  № 1 (4).  С. 56--62

\bibitem{Rogalev3}
 Rogalev  A.N., Rogalev A.A., Feodorova N.A.  Numerical computations of the safe boundaries of complex technical systems and practical stability. J. Phys: Conf.  Ser. 2019.  Vol. 1399,  33112.

\bibitem{Rogalev4}
 Rogalev A.N., Rogalev A.A., Feodorova N.A.  Malfunction analysis and safety of mathematical models of technical systems. J. Phys.: Conf. Ser.  2020.  Vol. 1515,  022064

\bibitem{Rogalev5}
Rogalev A.N.   Set of Solutions of Ordinary Differential Equations in Stability Problems. Continuum Mechanics, Applied Mathematics and Scientific Computing: Godunov's Legacy, Springer Nature Switzerland AG, 2020,
 307-312
\bibitem{Rogalev6}
Rogalev A.N., Rogalev A.A., Feodorova N.A.
Mathematical modelling of risk and safe regions of technical systems and surviving trajectories,  J. Phys: Conf Ser.  2021. Vol. 1889, 022108

\end{thebibliography}

% После библиографического списка в русскоязычных статьях необходимо оформить
% англоязычный заголовок.




%\end{document}

%%% Local Variables:
%%% mode: latex
%%% TeX-master: t
%%% End:
