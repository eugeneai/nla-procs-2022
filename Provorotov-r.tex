\begin{englishtitle}
\title{Navier-Stokes Evolutionary System with Spatial Variable in a Network-like Domain}
% First author
\author{Vyacheslav Provotorov\inst{1},
Semen Podvalny\inst{2}}
\institute{Voronezh State University, Voronezh, Russia\\
  \email{wwprov@mail.ru}
  \and
Voronezh State Technical University, Voronezh, Russia\\
\email{spodvalny@yandex.ru}}
% etc

\maketitle

\begin{abstract}
The paper deals with the question of the existence of a weak solution of the Navier-Stokes $n$-dimensional system with distributed parameters in a coherent limited network-like domain, which is a geometric graph in the one-dimensional case ($n = 1 $). A space of admissible solutions is introduced and, using the Faedo-Galerkin method, the existence of a solution from the class of functions summable on a network-like domain is established.
Taking into account the specificity of the Faedo-Galerkin method, when constructing approximations of a solution in the form of cutoff function, it is shown that such solutions are elements of the space of functions with summable derivatives from a time variable; for the elements of such a space, the analogue of the energy balance equation correctly.
The obtained results are the basis for the study of various types of problems of optimal control of differential systems with a spatial variable changing in a network-like domain.



\keywords{Navier-Stokes system, network-like domain}
\end{abstract}
\end{englishtitle}

\iffalse
\documentclass[12pt]{llncs}
\usepackage[T2A]{fontenc}
\usepackage[utf8]{inputenc}
\usepackage[english,russian]{babel}
\usepackage[russian]{nla}

%\usepackage[english,russian]{nla}

% \graphicspath{{pics/}} %Set the subfolder with figures (png, pdf).

%\usepackage{showframe}
\begin{document}
%\selectlanguage{russian}
\fi

\title{Эволюционная система Навье-Стокса с пространственной переменной в сетеподобной области}
% Первый автор
\author{В.~В.~Провоторов\inst{1}
  \and
% Второй автор
  С.~Л.~Подвальный\inst{2}
} % обязательное поле
\institute{Воронежский государственный университет, Воронеж, Россия\\
  \email{wwprov@mail.ru}
  \and
Воронежский государственный технический университет, Воронеж, Россия\\
\email{spodvalny@yandex.ru}}
% Другие авторы...

\maketitle

\begin{abstract}
В работе рассматривается вопрос существования слабого решения $n$-мерной  системы Навье-Стокса с распределенными параметрами в связной ограниченной сетеподобной области, представляющей собой геометрический граф в одномерном случае ($n=1$). Вводится пространство допустимых решений и, используя метод Фаэдо-Галеркина, устанавливается существование решения из класса суммируемых на сетеподобной области функций. Учитывая специфику метода Фаэдо-Галеркина при построении приближений решения в виде <<срезок>>, показывается, что такие решения являются элементами пространства функций с суммируемыми производными по временной переменной; для элементов такого пространства справедлив аналог уравнения энергетического баланса.  Полученные результаты лежат в основе исследования разного типа задач оптимального управления дифференциальными системами с пространственной переменной, изменяющейся  в сетеподобной области.

\keywords{система Навье-Стокса, сетеподобная область}
\end{abstract}



Рассмотрим несколько упрощенное описание сетеподобной области изменения пространственных переменных системы Навье-Стокса. Обозначим через $\Im$ ограниченную область евклидова пространства $\mathbb{R}^n$ ($n\geq2$), имеющую сетеподобную структуру, аналогичную структуре графа [1]: $\Im=\left(\bigcup\limits_{m=1}^M\Im_m\right)\cup\left(\bigcup\limits_{l=1}^JS_l\right)$, где  $S_l$~---  поверхность, разделяющая смежные подобласти $\Im_m\subset\Im$ (для фиксированного $l$ число смежных областей не менее двух), $\partial\Im$~--- граница  $\Im$ ($S_l\cap\partial\Im=\emptyset$). Поверхность  примыкания $S_l$ смежных подобластей $\Im_m$ назовем узловым местом: $S_l=\bigcup\limits_{l^\prime} S_{l,\,l^\prime}$, $S_{l,\,l^\prime}\subset \partial \Im_{l^\prime}$; количество поверхностей $S_{l,\,l^\prime}$  на единицу меньше числа примыкающих подобластей, при этом $S^-_{l,\,l^\prime}$, $S^+_{l,\,l^\prime}$~--- поверхности, определяемые соответствующими нормалями $n^-_l$, $n^+_l$ к $S_{l,\,l^\prime}$.


Для вектор-функции $Y(x,t)=\{y_1(x,t),y_2(x,t),...,y_n(x,t)\}$ ($x=\{x_1,x_2,...,x_n\}$), определенной в области $\Im_T=\Im\times(0,T)$ ($T<\infty$), рассмотрим систему
\begin{equation}\label{eq1}
{{\begin{array}{*{20}c}
\frac{\partial Y}{\partial t}-\nu\triangle Y+\sum\limits_{i=1}^nY_i\frac{\partial Y}{\partial x_i}= f-grad\,p,
 \end{array} }}
\end{equation}
\begin{equation}\label{eq2}
{{\begin{array}{*{20}c}
div Y=0\left(\sum\limits_{i=1}^n \frac{\partial Y}{\partial x_i}=0\right).
 \end{array} }}
\end{equation}
В каждом узловом месте имеют место соотношения (условия примыкания [2])
\begin{equation}\label{eq3}
{{\begin{array}{*{20}c}
Y|_{S^-_{l,\,l^\prime}}=Y|_{S^+_{l,\,l^\prime}},\,\,\,
\sum \limits_{l}\frac{\partial Y}{\partial n^-_l}|_{S^-_{l,\,l^\prime}}+\sum \limits_{l}\frac{\partial Y}{\partial n^+_l}|_{S^+_{l,\,l^\prime}}=0,
 \end{array} }}
\end{equation}
начальное  и граничное условия имеют вид
\begin{equation}\label{eq4}
{{\begin{array}{*{20}c}
Y(x,0)=Y_0(x), \,\,x\in \Im, \,\,\,Y|_{\partial\Im}=0.
 \end{array} }}
\end{equation}
В приложениях ([2], $n=3$) начально-краевая задача (1)--(4) описывает динамику несжимаемой жидкости с вязкостью $\nu>0$ в сетеподобной области $\Im$: (1), (2)~--- система Навье-Стокса, $Y(x,t)$, $x,t\in \overline{\Im}_T$~--- вектор скоростей гидравлического потока,  (3)~--- условия перетекания жидкости в узловых местах $\xi_{\,l}$. Используя классические пространства  $L_2(\Omega)$ и $W^1_2(\Omega)$, вводятся необходимые пространства. Пусть $\mathcal{V}=\{v:\overline{\Im}\rightarrow \mathbb{R}^n,\,\, v|_{\overline{\Im}_m}\in C^\infty(\overline{\Im}_m),\,\,m=\overline{1,M}\}$, при этом $v\in \mathcal{V}$ удовлетворяет условию (3) и второму условию (4); замыкание $\mathcal{V}$ в пространстве $(W^1_2(\Im))^n$  обозначим через $H^1(\Im)$.

Пусть далее $\Re(\Im_{T})$~--- множество всех функций  $u(x,t)\in W^{1,0}(\Im_{T})$ ($W^{1,0}(\Im_{T})$~--- пространство функций $u(x,t)\in (L_{2}(\Im_{T}))^n$, для которых обобщенная производная  $\frac{\partial u(x,t)}{\partial x}\in (L_{2}(\Im_{T}))^n$), имеющих  конечную норму
$\|u\|_{\Im_{T}}= \mathop{max}\limits_{0\leq t\leq T}\left\|u(\cdot,t)\right\|_{(L_{2}(\Im))^n}+\left\|\frac{\partial u}{\partial x}\right\|_{(L_{2}(\Im_{T}))^n}$ и  принадлежащие $H^1(\Im)$ следы на сечениях $\Im_{T}$ плоскостью $t=t_0$, причем $u(x,t)$ непрерывны по $t$ в норме $H^1(\Im)$ на $[0,T]$. Пусть
$V_{\,0}^{1,0}(\Im_{T})$~--- замыкание $\Re(\Im_{T})$ в норме $\|u\|_{\Im_{T}}$.

\textbf{Определение.} \emph{Слабым  решением задачи} (1)--(4) \emph{называется пара $\{Y,p\}$, при этом функция  $Y(x,t)\in V_{\,0}^{1,0}(\Im_{T})$  удовлетворяет интегральному тождеству}
\[
{{\begin{array}{*{20}c}
(Y(x,t),\eta(x,t))-\int\limits_{\Im_t}Y(x,\tau)\frac{\partial\eta(x,\tau)}{\partial \tau}dxd\tau+
\nu\int\limits_0^t\rho(Y,\eta)d\tau+\int\limits_0^t\varrho(Y,Y,\eta)d\tau=
\\
=(Y_0(x),\eta(x,0))+\int\limits_{\Im_t}f(x,\tau)\eta(x,\tau)dxd\tau
\end{array} }}
\]
\emph{для любых $t\in[0,T]$ и любых $\eta(x,t)\in V_{\,0}^{1,0}(\Im_{T})$, а $p(x,t)$ принадлежит классу бесконечно дифференцируемых в $\Im_T$ функций с компактными носителями в $\Im_T$.}



\section{Основной результат} % не обязательное поле


\textbf{Теорема.} \emph{Существует по меньшей мере одно слабое   решение начально-краевой задачи } (1)--(4) \emph{при произвольном} (\emph{конечном}) $T>0$.

Используя метод Фаэдо-Галеркина со специальным базисом в $H^1(\Im)$~--- системой обобщенных собственных функций $\{U_i(x)\}_{i\geq1}$, удовлетворяющей тождеству $\sum\limits_{i=1}^n \int\limits_\Im\frac{\partial U(x)}{\partial x_i}\frac{\partial \phi(x)}{\partial x_i}dx= \lambda \int\limits_\Im U(x)\phi(x)dx$ $\forall \,\phi(x)\in H^1(\Im)$, строятся приближения $Y_m(x,t)$ слабого решения системы (1)--(4): $Y_m(x,t)= \sum\limits_{i=1}^m g_{i\,m}(t)U_i(x)$ (скалярные функции $g_{i\,m}(t)$ абсолютно непрерывны на $[0,T]$). Дальнейшее доказательство основано на выводе уравнения энергетического баланса, априорных оценках норм $Y_m(x,t)$, установлении слабой компактности последовательности $\{Y_m(x,t)\}$ и сходимости ее в $V_{\,0}^{1,0}(\Im_{T})$.



% Список литературы.
\begin{thebibliography}{99}
\bibitem{1}
% Format for Journal Reference
Podvalny S. L., Provotorov V. V., Podvalny E. S.  {\it The controllability of parabolic
systems with delay and distributed parameters on the graph}. Procedia Computer Sciense.
2017. Vol.~103. Pp.~324-330.
\bibitem{2}
% Format for Journal Reference
Artemov M A,  Baranovskii E S, Zhabko A P,  Provotorov V V  {\it On a 3D model of non-isothermal flows in a pipeline network}. Journal of Physics. Conference Series.
2019. Vol.~1203. Article ID 012094.
\end{thebibliography}






%\end{document}

%%% Local Variables:
%%% mode: latex
%%% TeX-master: t
%%% End:
